\documentclass[12pt]{amsart}

\usepackage{amsmath}
\usepackage{amssymb}
\usepackage{amsthm}
\usepackage{url}
\usepackage{hyperref}
\usepackage{mathtools}
\DeclarePairedDelimiter\ceil{\lceil}{\rceil}
\DeclarePairedDelimiter\floor{\lfloor}{\rfloor}

%\usepackage{psfig}

\begin{document}
\baselineskip=16pt
\textheight=8.5in
\parindent=0pt 
\def\sk {\hskip .5cm}
\def\skv {\vskip .08cm}
\def\cos {\mbox{cos}}
\def\sin {\mbox{sin}}
\def\tan {\mbox{tan}}
\def\intl{\int\limits}
\def\lm{\lim\limits}
\newcommand{\frc}{\displaystyle\frac}
\def\xbf{{\mathbf x}}
\def\fbf{{\mathbf f}}
\def\gbf{{\mathbf g}}

\def\dbA{{\mathbb A}}
\def\dbB{{\mathbb B}}
\def\dbC{{\mathbb C}}
\def\dbD{{\mathbb D}}
\def\dbE{{\mathbb E}}
\def\dbF{{\mathbb F}}
\def\dbG{{\mathbb G}}
\def\dbH{{\mathbb H}}
\def\dbI{{\mathbb I}}
\def\dbJ{{\mathbb J}}
\def\dbK{{\mathbb K}}
\def\dbL{{\mathbb L}}
\def\dbM{{\mathbb M}}
\def\dbN{{\mathbb N}}
\def\dbO{{\mathbb O}}
\def\dbP{{\mathbb P}}
\def\dbQ{{\mathbb Q}}
\def\dbR{{\mathbb R}}
\def\dbS{{\mathbb S}}
\def\dbT{{\mathbb T}}
\def\dbU{{\mathbb U}}
\def\dbV{{\mathbb V}}
\def\dbW{{\mathbb W}}
\def\dbX{{\mathbb X}}
\def\dbY{{\mathbb Y}}
\def\dbZ{{\mathbb Z}}

\def\eps{{\varepsilon}}
\def\la{{\langle}}
\def\ra{{\rangle}}
\def\summ{{\sum\limits}}

\bf\centerline{Homework \#9}\rm
\skv
\it\centerline{Due Sat, April 6th by 11:59pm}\rm
\skv

\bf\centerline{Reading and plan for the next week: }\rm
\skv
1. For this homework assignment read 7.2, 7.3, online lectures 17-19 from Spring 24 and online lectures 16-18 from Spring 20. Note that there are theorems and examples in Lecture 18 from Spring 20 which we have not discussed in class.  
\skv
2. Plan for next week. MDS codes and generalized Reed-Solomon (GRS) codes (online lectures 19 and 20 from Spring 20). In the book MDS codes are discussed in 5.4 and GRS codes are discussed in 9.1; however, my presentation of GRS codes will be closer to that of Chapter~6 of Yehuda Lindell's notes (see a link on the course webpage).
\skv

\skv
\bf\centerline{Problems: }\rm

\skv
{\bf 1.} Problem 3.23(b)(d)(e). {\bf Hint:} For two of the three parts use Theorem~18.1 from online Lecture~18 in Spring 20 (which has nothing to do with Theorem 18.1 from this semester's notes). For the third part you can perform factorization by ``brute force" (keep doing natural factorizations as long as you can and then try to show that the factors are irreducible). Note that the factorization from 3.23(b) will be needed in Problem~4 below, so make sure to compute it correctly.
\skv
\skv
{\bf 2.} Let $F$ be a field, and let $g(x)\in F[x]$ be a nonzero polynomial.  Let
$k=\deg g(x)$, and write $g(x)=\sum_{i=0}^k g_i x^i$. Recall that the {\bf reciprocal polynomial} $\bar g(x)$ is defined by 
$${\overline g}(x)=x^k g\left(\frac{1}{x}\right)=\sum_{i=0}^k g_{k-i} x^i.$$ 
\begin{itemize}
\item[(a)] Prove that if $g(x)$ has nonzero constant term ($g_0\neq 0$), then $\overline{{\overline g}}=g$, that is, the double
reciprocal is equal to $g$ itself. Show that this is not true if $g(x)$ has zero constant term.
\item[(b)] Prove that $\overline{g(x) h(x)}=\overline{g(x)}\cdot \overline{h(x)}$ for all nonzero $g(x),h(x)\in F[x]$.
\end{itemize}
\skv
{\bf 3.} Let $F$ be a field, $n\in\dbN$, and let $C\subseteq F^n$ be a nonzero cyclic code. 
Let $g(x)=\sum_{i=0}^k g_i x^i$ be the generator polynomial for $C$, and let
${\overline g}(x)=\sum\limits_{i=0}^k g_{k-i} x^i$ be its reciprocal polynomial. 
Let $g'(x)=\frac{{\overline g}(x)}{g_0}$ (note that $g'(x)$ is monic).
\begin{itemize}
\item[(a)] Prove that $\overline g(x)$ divides $x^n-1$ (in $F[x]$) and hence $g'(x)$ divides $x^n-1$ as well. {\bf Hint:} Use Problem~3.
\item[(b)] By (a) and the correspondence between cyclic codes and monic divisors of $x^n-1$, there exists a unique cyclic code $C'$
whose generator polynomial is $g'(x)$. Prove that $C'$ is equivalent to $C$.  {\bf Hint:} Use Theorem~7.3.1 from the book (= Theorem 19.1(3)
from class). It may be useful to start with the case $g_0=1$ (so that $g'=\overline g$). The proof in the general case is not that different from this special case. 
\end{itemize}
\skv

{\bf 4.} Problem 7.11. Also find (with proof) all monic divisors of $x^{15}-1$ such that the corresponding cyclic code is equivalent to the Hamming code $Ham(4,2)$.
\skv
\skv

{\bf 5.} Problem 7.15.
\skv
\skv


{\bf 6.} Problem 7.35. {\bf Hint:} The following results may be useful for (b): Unique factorization theorem in $F[x]$ (Fact 19.3 from class);
if $f(x)\in F[x]$ and $a\in F$, then $(x-a) \mid f(x)$ if and only if $a$ is a root of $f$.
\end{document}