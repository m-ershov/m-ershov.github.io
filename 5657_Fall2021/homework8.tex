\documentclass[12pt]{amsart}

\usepackage{amsmath}
\usepackage{amssymb}
\usepackage{amsthm}
\usepackage{amscd}
\usepackage{url}
\usepackage{hyperref}
\usepackage[all]{xy}
%\usepackage{psfig}

\begin{document}
\baselineskip=16pt
\textheight=8.5in
\parindent=0pt 
\def\Ker {{\rm Ker}}
\def\Im {{\rm Im}}
\def\Tr {{\rm Tr}}
\def\sk {\hskip .3cm}
\def\skv {\vskip .08cm}
\def\cos {\mbox{cos}}
\def\sin {\mbox{sin}}
\def\tan {\mbox{tan}}
\def\intl{\int\limits}
\def\lm{\lim\limits}
\newcommand{\frc}{\displaystyle\frac}
\def\xbf{{\mathbf x}}
\def\fbf{{\mathbf f}}
\def\gbf{{\mathbf g}}

\def\dbA{{\mathbb A}}
\def\dbB{{\mathbb B}}
\def\dbC{{\mathbb C}}
\def\dbD{{\mathbb D}}
\def\dbE{{\mathbb E}}
\def\dbF{{\mathbb F}}
\def\dbG{{\mathbb G}}
\def\dbH{{\mathbb H}}
\def\dbI{{\mathbb I}}
\def\dbJ{{\mathbb J}}
\def\dbK{{\mathbb K}}
\def\dbL{{\mathbb L}}
\def\dbM{{\mathbb M}}
\def\dbN{{\mathbb N}}
\def\dbO{{\mathbb O}}
\def\dbP{{\mathbb P}}
\def\dbQ{{\mathbb Q}}
\def\dbR{{\mathbb R}}
\def\dbS{{\mathbb S}}
\def\dbT{{\mathbb T}}
\def\dbU{{\mathbb U}}
\def\dbV{{\mathbb V}}
\def\dbW{{\mathbb W}}
\def\dbX{{\mathbb X}}
\def\dbY{{\mathbb Y}}
\def\dbZ{{\mathbb Z}}

\def\lam{{\lambda}}
\def\Ker{\mathrm {Ker}}
\def\End{\mathrm {End}}
\def\Hom{\mathrm {Hom}}
\def\Irr{\mathrm {Irr}}
\def\la{{\langle}}
\def\ra{{\rangle}}
\def\summ{{\sum\limits}}
\def\char{{\rm char}}

\bf\centerline{Homework \#8. Due Saturday, November 6th}\rm
\vskip .1cm

\bf\centerline{Reading: }\rm
\skv
1. For this homework assignment: online class notes (Lectures 16 and 17) and Steinberg, parts of 4.2-4.4. {\bf Note:} We have not yet covered the material from Lecture 17 in class (will do so on Tuesday, Nov 2); however, several problems in this assignment are based on that material, and you are strongly encouraged to start reading before we get to that material in class.
\skv

2. Next week we will state and start proving orthogonality relations for characters (Theorem 17.1 from online notes) --
see online Lectures 17, 18 and 19. The corresponding material in Steinberg is in sections 4.2, 4.3 and 4.4, but in a rather different order. 
\skv
\skv


\bf\centerline{Problems: }\rm
\skv
For problems (or their parts) marked with a *, a hint is given later in the assignment. Do not to look at the hint(s) until you seriously tried to solve the problem without it.
\skv
{\bf 1.*} Prove that $[S_n,S_n]=A_n$.
\skv
{\bf 2.} Let $G$ be a group, $N$ a normal subgroup of $G$ and let $\pi:G\to G/N$ be the natural projection. 
\begin{itemize}
\item[(a)] Given a representation $\rho: G/N\to GL(V)$ of $G/N$, define the representation $\widetilde \rho: G\to GL(V)$ of $G$
by $$\widetilde \rho(g)=\rho\circ \pi(g)=\rho(gN). \eqno (***)$$ Prove that $\widetilde \rho$ is irreducible $\iff$ $\rho$ is irreducible. Also
prove that two representations $\rho_1$ and $\rho_2$ of $G/N$ are equivalent $\iff$ the corresponding representations 
$\widetilde \rho_1$ and $\widetilde \rho_2$ of $G$ are equivalent.
\item[(b)] Now fix a field $F$. Let
\begin{itemize}
\item $\Irr(G)$ be the set of equivalence classes of irreducible representations of $G$ over $F$;
\item $\Irr(G/N)$ the set of equivalence classes of irreducible representations of $G/N$ over $F$;
\item $\Irr(G,N)$ the set of all $[\rho]\in \Irr(G)$ such that $N\subseteq \Ker\rho$.
\end{itemize}
(here $[\rho]$ is the equivalence class of the representation $\rho$). Define the map $\Phi:\Irr(G/N)\to \Irr(G)$ by $$\Phi([\rho])=[\widetilde \rho]$$ (where $\widetilde \rho$ is defined by (***)). Explain why $\Phi$
is well defined and injective (this follows immediately from (a)) and then prove that $\Im(\Phi)=\Irr(G,N)$.
\end{itemize}
{\bf Remark:} In Lecture~17 (online Lecture~14) we considered the map $\Phi$ in a special case when we proved that for any group $G$ there is a natural bijection between 1-dimensional representations of $G$ and 1-dimensional representations of its abelianization $G^{ab}$. However the case of 1-dimensional representations is slightly easier, as two 1-dimensional representations are equivalent if and only if they are equal, so in that case
one does not have to worry about the equivalence classes in the definition of $\Phi$.
\skv
{\bf 3.} Let $(\rho_1,V_1)$ and $(\rho_2,V_2)$ be representations of a group $G$ over the same field. Let 
$(\rho_1\oplus \rho_2, V_1\oplus V_2)$ and $(\rho_1\otimes \rho_2, V_1\otimes V_2)$ be their direct sum and tensor product, respectively. Prove that
\begin{itemize}
\item[(i)] $\chi_{\rho_1\oplus \rho_2}(g)=\chi_{\rho_1}(g)+\chi_{\rho_2}(g)$ for all $g\in G$
\item[(ii)] $\chi_{\rho_1\otimes \rho_2}(g)=\chi_{\rho_1}(g)\cdot\chi_{\rho_2}(g)$ for all $g\in G$
\end{itemize}
\skv
{\bf 4.} Online Lecture~17 states what should be the character of the unique (up to equivalence) 2-dimensional irreducible complex representation of $S_4$ based on our knowledge of the rest of the character table. Now prove this claim using Theorem~17.1. Include all the relevant calculations and try to make your argument as efficient as possible.
\skv
{\bf 5.} Compute the character table for a cyclic group of order $3$ (with full justification). 
\skv
{\bf 6.} Compute the character table for the alternating group $A_4$ (with detailed justification) and explicitly construct its irreducible complex representations. First prove that $[A_4,A_4]=V_4$, the Klein $4$-group. Recall that
$V_4=\{e, (12)(34), (13)(24), (14)(23)\}$.
\skv
{\bf Note:} The description of conjugacy classes of $A_n$ is similar to that of $S_n$. Note that since $A_n$
is normal in $S_n$, every conjugacy class of $S_n$ is either contained in $A_n$ or has empty intersection with $A_n$,
but a single conjugacy class of $S_n$ contained in $A_n$ may split into several conjugacy classes of $A_n$. In other words, two elements of $A_n$ may be conjugate in $S_n$, but not conjugate in $A_n$. The following result (which is not an official part of homework, but which are you encouraged to prove as an exercise) describe exactly when and what kind of splitting occurs:
\skv

Let $g\in A_n$ and let $K(g)$ be its conjugacy class in $S_n$. Let $C(g)$ be the centralizer of $g$ in $S_n$, that is,
$C(g)=\{x\in S_n:\, gx=xg\}$.

\begin{itemize}
\item[(a)] Suppose that $C(g)\not\subseteq A_n$, that is, $C(g)$ contains an odd permutation. Then $K(g)$ is a single conjugacy class of $A_n$, that is, any two elements of $K(g)$ are conjugate in $A_n$.
\item[(b)] Suppose that $C(g)\subseteq A_n$. Then $K(g)$ is a disjoint union of two conjugacy classes of $A_n$. One
of these is $K_{A_n}(g)$, the conjugacy class of $g$ in $A_n$, and the other is $K_{A_n}(xgx^{-1})$ where $x\in S_n$
is any odd permutation.
\end{itemize}
\skv

{\bf 7.} Let $G$ be a group and $(\rho,V)$ a representation of $G$. 
\begin{itemize}
\item[(a)] Prove that for any $v\in V$, the smallest $G$-invariant subspace of $V$ containing $v$ 
is the span of the set $\pi(G)v$ where $\pi(G)v=\{\pi(g)v: g\in G\}$.
\item[(b)] Now assume that $G$ is finite and $(\rho,V)$ is cyclic. Prove that $\dim(V)\leq |G|$.
\end{itemize}
\newpage
{\bf Hint for 1:} Prove the inclusions $A_n\subseteq [S_n,S_n]$ and $[S_n,S_n]\subseteq A_n$ separately. For the first one use the fact that $A_n$ is generated by $3$-cycles. For the second one use basic properties of the commutator subgroup discussed in class. 
\end{document}
