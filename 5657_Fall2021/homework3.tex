\documentclass[12pt]{amsart}

\usepackage{amsmath}
\usepackage{amssymb}
\usepackage{amsthm}
\usepackage{amscd}
\usepackage{url}
\usepackage{hyperref}
\usepackage[all]{xy}
%\usepackage{psfig}

\begin{document}
\baselineskip=16pt
\textheight=8.5in
\parindent=0pt 
\def\Ker {{\rm Ker}}
\def\Im {{\rm Im}}
\def\sk {\hskip .5cm}
\def\skv {\vskip .08cm}
\def\cos {\mbox{cos}}
\def\sin {\mbox{sin}}
\def\tan {\mbox{tan}}
\def\intl{\int\limits}
\def\lm{\lim\limits}
\newcommand{\frc}{\displaystyle\frac}
\def\xbf{{\mathbf x}}
\def\fbf{{\mathbf f}}
\def\gbf{{\mathbf g}}

\def\dbA{{\mathbb A}}
\def\dbB{{\mathbb B}}
\def\dbC{{\mathbb C}}
\def\dbD{{\mathbb D}}
\def\dbE{{\mathbb E}}
\def\dbF{{\mathbb F}}
\def\dbG{{\mathbb G}}
\def\dbH{{\mathbb H}}
\def\dbI{{\mathbb I}}
\def\dbJ{{\mathbb J}}
\def\dbK{{\mathbb K}}
\def\dbL{{\mathbb L}}
\def\dbM{{\mathbb M}}
\def\dbN{{\mathbb N}}
\def\dbO{{\mathbb O}}
\def\dbP{{\mathbb P}}
\def\dbQ{{\mathbb Q}}
\def\dbR{{\mathbb R}}
\def\dbS{{\mathbb S}}
\def\dbT{{\mathbb T}}
\def\dbU{{\mathbb U}}
\def\dbV{{\mathbb V}}
\def\dbW{{\mathbb W}}
\def\dbX{{\mathbb X}}
\def\dbY{{\mathbb Y}}
\def\dbZ{{\mathbb Z}}

\def\lam{{\lambda}}
\def\la{{\langle}}
\def\ra{{\rangle}}
\def\summ{{\sum\limits}}
\def\char{{\rm char}}

\bf\centerline{Homework \#3. Due Saturday, Sep 18}\rm
\vskip .1cm

1. Reading for this homework assignment: Friedberg-Insel-Spence 6.1, 6.3 + online class notes (Lectures 5,6)
\skv
\skv
2. Plan for the next week: Diagonalization in Inner Product Spaces (Friedberg-Insel-Spence 6.4, 6.5, online lectures 7,8)
\skv


\bf\centerline{Problems: }\rm
\skv
{\bf 1.} Let $V$ be a finite-dimensional vector space over a field $F$ of characteristic $2$ and $H$ a symmetric (=skew-symmetric since $\char F=2$) bilinear form on $V$. Prove that there exist subspaces $V_1$ and $V_2$ of $V$ such that
\begin{itemize}
\item[(a)] $V=V_1\oplus V_2$ and $V_1\perp V_2$ (that is, $H(v,w)=0$ for all $v\in V_1$ and $w\in V_2$).
\item[(b)] $H_{| V_1}$ is diagonalizable (that is, $[H_{| V_1}]_{\beta_1}$ is diagonal for some basis $\beta_1$ of $V_1$)
\item[(c)] $H_{| V_2}$ is alternating and non-degenerate (such a form is called symplectic).
\end{itemize}
{\bf Hint:} Combine the proofs of Theorems~4.2 and 5.1 from class.
\skv
{\bf 2.} As observed at the beginning of Lecture~5, if $V$ is a finite-dimensional vector space and $H$ is a bilinear form on $V$,
the following conditions are equivalent:

\begin{itemize}
\item[(i)] $H$ is left non-degenerate
\item[(ii)] $H$ is right non-degenerate
\item[(iii)] $[H]_{\beta}$ is invertible for some (hence any) basis $\beta$ of $V$.
\end{itemize}

Now assume that $\dim V$ is countably infinite and $\beta=\{v_1,v_2,\ldots\}$ is a basis of $V$. 

\begin{itemize}
\item[(a)] Find (with proof) conditions $(ND)_{left}$ and $(ND)_{right}$ on an infinite square matrix such that $H$ is left non-degenerate $\iff$ $[H]_{\beta}$ satisfies $(ND)_{left}$
and $H$ is right non-degenerate $\iff$ $[H]_{\beta}$ satisfies $(ND)_{right}$. {\bf Hint:} The conditions will be non-equivalent, but in the finite dimensional case they should both reduce to different (but similar) well-known characterizations of invertible matrices.

\item[(b)] Use (a) to find a bilinear form $H$ on $V$ which is left-nondenerate but not right-nondegenerate.
\end{itemize}






 


\skv
{\bf 3.} Let $V$ be an inner product space.
\begin{itemize}
\item[(a)] Prove the parallelogram law: $\|x+y\|^2+\|x-y\|^2=2(\|x\|^2+\|y\|^2)$ for all $x,y\in V$.
\item[(b)] Show that $\la x,y\ra$ can be expressed as a linear combination of squares of norms. In Lecture~6 we discussed how to do this for the real
inner product spaces. 
\end{itemize}
\skv
{\bf 4.} Let $V$ be a finite-dimensional complex inner product space and $A\in\mathcal L(V)$.
Prove that $\Im(A^*)=\Ker(A)^{\perp}$ (where the orthogonal complement is with respect to the inner product on $V$).
Here $A^*$ is the adjoint operator of $A$ (see the end of the online lecture 6 for the definition).
\skv
{\bf 5.} Let $V$ be an inner product space where $\dim V$ is finite or countable, $\beta$ an orthonormal basis of $V$ and $A\in\mathcal L(V)$. 
\begin{itemize} 
\item[(a)] Prove that if $A^*\in \mathcal L(V)$ is any operator such that $\la Ax,y\ra=\la x,A^*y\ra$ for all $x,y\in V$, then $[A^*]_{\beta}=[A]_{\beta}^*$ (where $[A]_{\beta}^*$ is the conjugate transpose of $A$).
In particular, this shows that the adjoint operator is unique (if exists).
\item[(b)] As we will prove in class, the adjoint $A^*$ always exists if $\dim V$ is finite. Now use (a) and a result from earlier homeworks to
show that if $V$ is countably-dimensional, then the adjoint $A^*$ may not exist.
\end{itemize}
\end{document}
