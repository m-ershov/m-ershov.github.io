\documentclass[12pt]{amsart}

\usepackage{amsmath}
\usepackage{amssymb}
\usepackage{amsthm}
\usepackage{url}
\usepackage{hyperref}
\usepackage{mathtools}
\DeclarePairedDelimiter\ceil{\lceil}{\rceil}
\DeclarePairedDelimiter\floor{\lfloor}{\rfloor}

%\usepackage{psfig}

\begin{document}
\baselineskip=16pt
\textheight=8.5in
\parindent=0pt 
\def\sk {\hskip .5cm}
\def\skv {\vskip .08cm}
\def\cos {\mbox{cos}}
\def\sin {\mbox{sin}}
\def\tan {\mbox{tan}}
\def\intl{\int\limits}
\def\lm{\lim\limits}
\newcommand{\frc}{\displaystyle\frac}
\def\xbf{{\mathbf x}}
\def\fbf{{\mathbf f}}
\def\gbf{{\mathbf g}}

\def\dbA{{\mathbb A}}
\def\dbB{{\mathbb B}}
\def\dbC{{\mathbb C}}
\def\dbD{{\mathbb D}}
\def\dbE{{\mathbb E}}
\def\dbF{{\mathbb F}}
\def\dbG{{\mathbb G}}
\def\dbH{{\mathbb H}}
\def\dbI{{\mathbb I}}
\def\dbJ{{\mathbb J}}
\def\dbK{{\mathbb K}}
\def\dbL{{\mathbb L}}
\def\dbM{{\mathbb M}}
\def\dbN{{\mathbb N}}
\def\dbO{{\mathbb O}}
\def\dbP{{\mathbb P}}
\def\dbQ{{\mathbb Q}}
\def\dbR{{\mathbb R}}
\def\dbS{{\mathbb S}}
\def\dbT{{\mathbb T}}
\def\dbU{{\mathbb U}}
\def\dbV{{\mathbb V}}
\def\dbW{{\mathbb W}}
\def\dbX{{\mathbb X}}
\def\dbY{{\mathbb Y}}
\def\dbZ{{\mathbb Z}}

\def\eps{{\varepsilon}}
\def\la{{\langle}}
\def\ra{{\rangle}}
\def\summ{{\sum\limits}}

\bf\centerline{Homework \#9}\rm
\skv
\it\centerline{Due Sun, April 5th by 23:59pm in filedrop}\rm
\skv

\bf\centerline{Reading and plan for the next week: }\rm
\skv
1. For this homework assignment read 7.2 and 7.3. 
\skv
2. Plan for next week: We will talk about some of the material in Chapter 8, but I am not yet sure about the order and whether we will follow the book or not. For instance, for the discussion of Reed-Solomon codes I currently plan to follow the presentation in Chapter~6 of Yehuda Lindell's notes (see a link on the course webpage).
\skv

\skv
\bf\centerline{Problems: }\rm
\skv
{\bf 1.} Let $F$ be a field and $R_n(F)=F[x]/(x^n-1)$.
\begin{itemize}
\item[(a)] Prove (rigorously) that for any $u(x),v(x)\in R_n(F)$, their product in $R_n(F)$ (which we denote by $u(x)\odot v(x)$)
is obtained from $u(x)v(x)$ (the product in the regular polynomial ring $F[x]$) by replacing $x^k$ by $x^{k\!\mod n}$ for each $k$ (as usual, $k\!\mod n$ is the principal remainder of dividing $k$ by $n$).
\item[(b)] Let $g(x)=\sum_{i=0}^{n-1}x^i$. Use (a) to prove that $\la g(x)\ra$, the principal ideal of $R_n(F)$ generated by $g(x)$, is equal to $\{a\cdot g(x): a\in F\}$, the set of scalar multiples of $g(x)$.
\end{itemize}

\skv
{\bf 2.} Problem 3.23(b)(d)(e). {\bf Hint:} For two of the three parts you can use a suitable theorem from Lecture~18. For the third part you can perform factorization by ``brute force" (keep doing natural factorizations as long as you can and then try to show that the factors are irreducible).
\skv
\skv

{\bf 3.} Problem 7.6. Make sure to include all your work.
\skv
\skv


{\bf 4.} Let $F$ be a field, $n\in\dbN$, let $C\subseteq F^n$ be a cyclic code and let $g(x)$ be the generator polynomial for $C$. Let
$k=\deg g(x)$, and write $g(x)=\sum_{i=0}^k g_i x^i$. Recall that the {\bf reciprocal polynomial} $\bar g(x)$ is defined by 
$${\overline g}(x)=x^k g\left(\frac{1}{x}\right)=\sum_{i=0}^k g_{k-i} x^i.$$  
Let $g'(x)=\frac{{\overline g}(x)}{g_0}$ (note that $g'(x)$ is monic).
\begin{itemize}
\item[(a)] Prove that $\overline g(x)$ divides $x^n-1$ (in $F[x]$) and hence $g'(x)$ divides $x^n-1$ as well.
\item[(b)] By (a) and the correspondence between cyclic codes and monic divisors of $x^n-1$, there exists a unique cyclic code $C'$
whose generator polynomial is $g'(x)$. Prove that $C'$ is equivalent to $C$.  {\bf Hint:} Use Theorem~7.3.1. It may be useful to start with the case $g_0=1$ (so that $g'=\overline g$). The proof in the general case is not that different from this special case. 
\end{itemize}
\skv

{\bf 5.} Problem 7.11. Also find (with proof) all monic divisors of $x^{15}-1$ such that the corresponding cyclic code is equivalent to the Hamming code $Ham(4,2)$.
\skv
\skv

{\bf 6.} Problem 7.15.
\skv
\skv


{\bf 7.} Problem 7.22. {\bf Hint:} There is a problem from Chapter 4 (which was previously assigned as a homework problem) that is very relevant.
\end{document}