\documentclass[12pt]{amsart}

\usepackage{amsmath}
\usepackage{amssymb}
\usepackage{amsthm}
%\usepackage{psfig}

\begin{document}
\baselineskip=16pt
%\textheight=9.6in
%\parindent=0pt
\def\sk {\hskip .5cm}
\def\skv {\vskip .12cm}
\def\cos {\mbox{cos}}
\def\sin {\mbox{sin}}
\def\tan {\mbox{tan}}
\def\intl{\int\limits}
\def\lm{\lim\limits}
\newcommand{\frc}{\displaystyle\frac}
\def\xbf{{\mathbf x}}
\def\fbf{{\mathbf f}}
\def\gbf{{\mathbf g}}

\def\dbA{{\mathbb A}}
\def\dbB{{\mathbb B}}
\def\dbC{{\mathbb C}}
\def\dbD{{\mathbb D}}
\def\dbE{{\mathbb E}}
\def\dbF{{\mathbb F}}
\def\dbG{{\mathbb G}}
\def\dbH{{\mathbb H}}
\def\dbI{{\mathbb I}}
\def\dbJ{{\mathbb J}}
\def\dbK{{\mathbb K}}
\def\dbL{{\mathbb L}}
\def\dbM{{\mathbb M}}
\def\dbN{{\mathbb N}}
\def\dbO{{\mathbb O}}
\def\dbP{{\mathbb P}}
\def\dbQ{{\mathbb Q}}
\def\dbR{{\mathbb R}}
\def\dbS{{\mathbb S}}
\def\dbT{{\mathbb T}}
\def\dbU{{\mathbb U}}
\def\dbV{{\mathbb V}}
\def\dbW{{\mathbb W}}
\def\dbX{{\mathbb X}}
\def\dbY{{\mathbb Y}}
\def\dbZ{{\mathbb Z}}

\def\la{{\langle}}
\def\ra{{\rangle}}

\def\Aut{{\rm Aut}}
\def\End{{\rm End}}
\def\Inn{{\rm Inn}}
\def\Ker{{\rm Ker}}
\def\Im{{\rm Im\,}}
\def\phi{{\varphi}}

\bf\centerline{Math 4452, Spring 2020. Midterm \#2}\rm
\skv
\bf\centerline{due Tuesday, April 14th, by 23:59pm in filedrop}\rm
\vskip .3cm
{\bf Directions: } Provide complete arguments
(do not skip steps). State clearly any result you are referring to. Partial credit for
incorrect solutions, containing steps in the right direction, may be given.
\vskip .1cm

{\bf Rules: } You are not allowed to discuss midterm problems with each other.
You may ask me any questions about the problems (e.g. if the formulation is unclear),
but as a rule I will only provide minor hints. You may freely use class notes (your own notes as well as notes posted on collab),
previous homework assignments, our main textbook ``Coding theory: a first course'' and lectures notes by J. Hall and Y. Lindell. The use of other books or other online resources is prohibited.

\skv
{\bf Scoring:} The best 4 out of 5 problems will count. Note that problems are not weighted equally! The maximum possible score is 44, but the score of 40 will count as 100\%. Solving any 4 problems completely correctly is sufficient to get 40 points.

\skv
{\bf 1.} (8 pts) In each part determine if a code with given parameters exists. Make sure to prove your answer.
\begin{itemize}
\item[(a)] binary $[12,11,2]$-linear code
\item[(b)] binary $[12,6,5]$-linear code
\item[(c)] binary $[12,5,6]$-linear code
\item[(d)] ternary $[12,6,6]$-linear code
\end{itemize}
\skv
{\bf 2.} (10 pts) Problem 5.18. {\bf Hint:} there is something very special about distance 3.
\skv
{\bf 3.} (12 pts) In HW\#6 we determined the distribution of weights in the extended binary Golay code $G_{24}$ assuming without proof that $G_{24}$
has exactly $759$ words of weight $8$. The goal of this problem is to prove the latter statement.

Parts (a)-(d) below deal with $G_{23}$, not $G_{24}$. For each $k\in\dbN$ let us denote by $n_k(G_{23})$ the number of words of weight $k$ in $G_{23}$.
\begin{itemize}
\item[(a)] According to Problem~4(c) in HW\#6 the following holds: For every $w\in \dbF_2^{23}$ with $wt(w)=4$ there exists $c\in G_{23}$
such that $wt(c)=7$ and $d(w,c)=3$. Explain why such $c$ must be unique.
\item[(b)] Use the result of (a) (as well as Problem~4(c) in HW\#6) to prove that $${7 \choose 4}\cdot n_{7}(G_{23})={23 \choose 4}.$$ Deduce that
$n_7(G_{23})=253$.
\item[(c)] Now prove that for every $w\in \dbF_2^{23}$ with $wt(w)=5$ there exists unique $c\in G_{23}$ such that either $wt(c)=7$ and $d(w,c)=2$ or $wt(c)=8$ and $d(w,c)=3$. 
\item[(d)] Use (c) to find a relation of the form $A n_7(G_{23}) + B n_8(G_{23})=C$ where $A,B,C$ are some (explicit) binomial coefficients. Use this relation to prove that $n_8(G_{23})=506$.
\item[(e)] Now use (b) and (d) to prove that $G_{24}$ has exactly $759$ words of weight $8$.
\end{itemize}
\skv
{\bf 4.} (10 pts) Problem 7.35.
\skv
{\bf 5.} (12 pts)
\begin{itemize}
\item[(a)] Factor $x^{24}-1$ as a product of monic irreducibles in $\dbF_2[x]$. Make sure to prove your answer.
\item[(b)] List all polynomials $g(x)\in \dbF_2[x]$ which are generators for binary cyclic codes of length $24$ and dimension $18$.
\item[(c)] Prove that there exists unique binary cyclic code of length $24$ which is self-dual. What is the generator polynomial for that code? You may use without proof that $\overline{u(x)\cdot v(x)}=\overline{u(x)}\cdot\overline{v(x)}$ for any polynomials
$u(x),v(x)\in F[x]$ ($F$ is any field) where $\overline{f(x)}$ is the reciprocal polynomial of $f(x)$.
\item[(d)] Use (c) to prove that the extended Golay code $G_{24}$ is NOT equivalent to a cyclic code.
\item[(e)] Find (with proof) $n\in\dbN$ such that there exists more than one binary cyclic self-dual code of length $n$.
\end{itemize}

\end{document}
