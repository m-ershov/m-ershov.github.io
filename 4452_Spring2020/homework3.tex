\documentclass[12pt]{amsart}

\usepackage{amsmath}
\usepackage{amssymb}
\usepackage{amsthm}
%\usepackage{psfig}

\begin{document}
\baselineskip=16pt
\textheight=8.5in
\parindent=0pt 
\def\sk {\hskip .5cm}
\def\skv {\vskip .08cm}
\def\cos {\mbox{cos}}
\def\sin {\mbox{sin}}
\def\tan {\mbox{tan}}
\def\intl{\int\limits}
\def\lm{\lim\limits}
\newcommand{\frc}{\displaystyle\frac}
\def\xbf{{\mathbf x}}
\def\fbf{{\mathbf f}}
\def\gbf{{\mathbf g}}

\def\dbA{{\mathbb A}}
\def\dbB{{\mathbb B}}
\def\dbC{{\mathbb C}}
\def\dbD{{\mathbb D}}
\def\dbE{{\mathbb E}}
\def\dbF{{\mathbb F}}
\def\dbG{{\mathbb G}}
\def\dbH{{\mathbb H}}
\def\dbI{{\mathbb I}}
\def\dbJ{{\mathbb J}}
\def\dbK{{\mathbb K}}
\def\dbL{{\mathbb L}}
\def\dbM{{\mathbb M}}
\def\dbN{{\mathbb N}}
\def\dbO{{\mathbb O}}
\def\dbP{{\mathbb P}}
\def\dbQ{{\mathbb Q}}
\def\dbR{{\mathbb R}}
\def\dbS{{\mathbb S}}
\def\dbT{{\mathbb T}}
\def\dbU{{\mathbb U}}
\def\dbV{{\mathbb V}}
\def\dbW{{\mathbb W}}
\def\dbX{{\mathbb X}}
\def\dbY{{\mathbb Y}}
\def\dbZ{{\mathbb Z}}

\def\la{{\langle}}
\def\ra{{\rangle}}
\def\summ{{\sum\limits}}

\bf\centerline{Homework \#3. Due Wednesday, February 5th, in class}\rm
\vskip .1cm
All reading assignments and references to exercises, definitions etc. are from our main book `Coding Theory: A First Course' by Ling and Xing 
\vskip .1cm

\bf\centerline{Reading and plan for the next week: }\rm
\skv
1. For this homework assignment read 4.2-4.6
\skv
\skv
2. Next week we will continue with the basic theory of linear codes (4.2-4.6). Then (probably next Wednesday) we will introduce binary Hamming codes (5.3.1, page 84). If there will be time left, we will go back to Chapter 4 and start discussing encoding and decoding for linear codes (4.7-4.8).
\skv

\skv
\bf\centerline{Problems: }\rm
\skv
{\bf 1.} Problem 4.3. {\bf Hint:} First count the number of ordered $k$-tuples $(v_1,\ldots, v_k)$ such that the vectors $v_1,\ldots, v_k$ are linearly independent. This can be done by using the argument from the proof of Theorem~4.1.15(ii).

{\bf 2.} Problem 4.14. If the code is linear, find its generator matrix and its parity-check matrix.

{\bf 3.} Problem 4.15.

{\bf 4.} Problem 4.20.

{\bf 5.} 
\begin{itemize}
\item[(a)] Let $C$ be a binary linear code of length $n$. Fix $1\leq i\leq n$, consider the $i^{\rm th}$ coordinates of all codewords in $C$. Prove that either all codewords have $0$ as their $i^{\rm th}$ coordinate or exactly half of all codewords have $0$ as their $i^{\rm th}$ coordinate. This is based on the same idea as 4.20(iii).
\item[(b)] In Homework~1 we proved that if $C$ is a binary code of length $n$ and distance $2$, then $|C|\leq 2^{n-1}$; thus, if in addition $C$ is linear,
then $\dim C\leq n-1$. Now prove that if $C$ is a binary $[n,n-1,2]$-linear code, then $C$ is the parity-check code. {\bf Hint:} Use induction on $n$ and part (a). If you need a more detailed hint, see next page.
\end{itemize}

{\bf 6.} Problem 4.22.

{\bf 7.} Problem 4.31.


\newpage
{\bf Hint for 5(b):} For the induction step take an arbitrary binary  $[n,n-1,2]$-linear code $C$, consider the set $C'=\{w\in\dbF_2^{n-1}: w0\in C\}$
(here $w0$ is the concatenation of $w$ and $0$), show that $C'$ is a linear code of Hamming distance $\geq 2$ (and length $n$) and apply part (a).
\end{document}



