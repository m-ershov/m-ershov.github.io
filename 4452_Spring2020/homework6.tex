\documentclass[12pt]{amsart}

\usepackage{amsmath}
\usepackage{amssymb}
\usepackage{amsthm}
%\usepackage{psfig}

\begin{document}
\baselineskip=16pt
\textheight=8.8in
\parindent=0pt 
\def\sk {\hskip .5cm}
\def\skv {\vskip .08cm}
\def\cos {\mbox{cos}}
\def\sin {\mbox{sin}}
\def\tan {\mbox{tan}}
\def\intl{\int\limits}
\def\lm{\lim\limits}
\newcommand{\frc}{\displaystyle\frac}
\def\xbf{{\mathbf x}}
\def\fbf{{\mathbf f}}
\def\gbf{{\mathbf g}}

\def\dbA{{\mathbb A}}
\def\dbB{{\mathbb B}}
\def\dbC{{\mathbb C}}
\def\dbD{{\mathbb D}}
\def\dbE{{\mathbb E}}
\def\dbF{{\mathbb F}}
\def\dbG{{\mathbb G}}
\def\dbH{{\mathbb H}}
\def\dbI{{\mathbb I}}
\def\dbJ{{\mathbb J}}
\def\dbK{{\mathbb K}}
\def\dbL{{\mathbb L}}
\def\dbM{{\mathbb M}}
\def\dbN{{\mathbb N}}
\def\dbO{{\mathbb O}}
\def\dbP{{\mathbb P}}
\def\dbQ{{\mathbb Q}}
\def\dbR{{\mathbb R}}
\def\dbS{{\mathbb S}}
\def\dbT{{\mathbb T}}
\def\dbU{{\mathbb U}}
\def\dbV{{\mathbb V}}
\def\dbW{{\mathbb W}}
\def\dbX{{\mathbb X}}
\def\dbY{{\mathbb Y}}
\def\dbZ{{\mathbb Z}}

\def\eps{{\varepsilon}}
\def\la{{\langle}}
\def\ra{{\rangle}}
\def\summ{{\sum\limits}}

\bf\centerline{Homework \#6. Due Thursday, March 5th, by 1pm in my mailbox }\rm
\vskip .1cm

\bf\centerline{Reading and plan for the next week: }\rm
\skv
1. For this homework assignment read 5.1-5.4.
\skv
2. Plan for next week: 5.7 (Griesmer bound) and 5.5 (Plotkin bound), in this order; perhaps a bit more on MDS codes (5.4). 
If time left, we may talk about Reed-Muller codes (6.2).
\skv

\skv
\bf\centerline{Problems: }\rm
\skv

{\bf 1.} Problem 5.3(ii). {\bf Hint:} PCM in the special case $\eps=1$ is given by Theorem~5.1.9 (note that this theorem is NOT proved in the book). To justify your answer for PCM it may be convenient to use Theorem~5.2 from class.
\skv

{\bf 2.} Problem 5.7. 
\skv

{\bf 3.}
\begin{itemize}
\item[(a)] Show that the all-one vector $(1, 1,\ldots , 1)$ of length $24$ lies in the extended binary Golay code $G_{24}$.
\item[(b)] Assume without proof that $G_{24}$ contains (exactly) $759$ words of weight $8$. Use this fact and (a) to prove that the distribution of weights in $G_{24}$ is given by Table~5.5 on page 109. 
\end{itemize}
\skv


{\bf 4.}
\begin{itemize}
\item[(a)] Prove that the Golay code $G_{23}$ is perfect by checking that it matches the sphere-packing bound.
\item[(b)] Use Problem~3(b) to prove that possible weights of elements of $G_{23}$ are $0,7,8,11,12,15,16$ and $23$. Make sure to prove that each of those numbers actually arises as the weight of some element of $G_{23}$.
\item[(c)] Let $w\in \dbF_2^{23}$ with $wt(w)=4$. Let $c_{w}\in G_{23}$ be the result of applying NND decoding (with respect to $G_{23}$) to $w$.
Use (a) and (b) to prove that $d(w,c_w)=3$ and $wt(c_w)=7$. 
\end{itemize}
\skv

{\bf 5.} Problem~5.19. {\bf Note:} Simplex codes $S(r,q)$ are defined at the end of 5.3.2, page 88.
\skv
{\bf 6.} Let $r\geq 2$ be an integer.
\begin{itemize}
\item[(a)] Assume that $d\geq 3$. Prove that there is no binary $[2^r,2^r-r,d]$-linear code.
\item[(b)] Given an example of a  binary $[2^r,2^r-r-1,4]$-linear code.
\end{itemize}
 \end{document}
