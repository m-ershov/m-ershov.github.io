\documentclass[12pt]{amsart}

\usepackage{amsmath}
\usepackage{amssymb}
\usepackage{amsthm}
\usepackage{url}
\usepackage{hyperref}
%\usepackage{psfig}

\begin{document}
\baselineskip=15pt
\textheight=8.4in
\parindent=0pt
\def\sk {\hskip .5cm}
\def\skv {\vskip .12cm}
\def\cos {\mbox{cos}}
\def\sin {\mbox{sin}}
\def\tan {\mbox{tan}}
\def\intl{\int\limits}
\def\lm{\lim\limits}
\newcommand{\frc}{\displaystyle\frac}
\def\xbf{{\mathbf x}}
\def\fbf{{\mathbf f}}
\def\gbf{{\mathbf g}}

\def\Ker{{\rm Ker\,}}
\def\phi{\varphi}

\def\dbA{{\mathbb A}}
\def\dbB{{\mathbb B}}
\def\dbC{{\mathbb C}}
\def\dbD{{\mathbb D}}
\def\dbE{{\mathbb E}}
\def\dbF{{\mathbb F}}
\def\dbG{{\mathbb G}}
\def\dbH{{\mathbb H}}
\def\dbI{{\mathbb I}}
\def\dbJ{{\mathbb J}}
\def\dbK{{\mathbb K}}
\def\dbL{{\mathbb L}}
\def\dbM{{\mathbb M}}
\def\dbN{{\mathbb N}}
\def\dbO{{\mathbb O}}
\def\dbP{{\mathbb P}}
\def\dbQ{{\mathbb Q}}
\def\dbR{{\mathbb R}}
\def\dbS{{\mathbb S}}
\def\dbT{{\mathbb T}}
\def\dbU{{\mathbb U}}
\def\dbV{{\mathbb V}}
\def\dbW{{\mathbb W}}
\def\dbX{{\mathbb X}}
\def\dbY{{\mathbb Y}}
\def\dbZ{{\mathbb Z}}

\def\Aut{{\rm Aut}}

\def\la{{\langle}}
\def\ra{{\rangle}}
\def\rk{{\rm rk}}


\bf\centerline{Homework Assignment \# 3. }\rm
\vskip .2cm
{\bf Plan for the week of Feb 15:} Classification of finitely generated modules over PIDs (12.1, online lecture 9). Rational Canonical Form (12.2, online lectures 10-11).
\vskip .1cm
Here and in all future assignments ``online'' refers to Algebra-II lectures posted on my Spring 2010 Algebra-II webpage
\vskip .1cm
\centerline{\url{http://people.virginia.edu/~mve2x/7752_Spring2010/}}
\vskip .1cm

{\bf Note on hints:} All hints are given at the end of the assignment, each on a separate page.
Problems (or parts of problems) for which hint is available are marked with *.

\vskip .3cm

\bf\centerline{Problems, due by 11:59pm on Friday, February 19th.}\rm
\vskip .1cm
{\bf Problem 1.} In class we briefly discussed a simple characterization
of $R$-modules where $R=\dbZ$ or $R=F[x]$ for some field $F$ (see online Lecture~1 and DF, pp. 340-341 for more details). More specifically, 
we explained why 
\begin{itemize}
\item there is a natural correspondence between $\dbZ$-modules and abelian groups
\item for any field $F$, there is a natural correspondence between $F[x]$-modules and pairs $(V,A)$ where $V$ is an $F$-vector space and 
$A:V\to V$ is an $F$-linear map.
\end{itemize}

State and prove similar characterizations for the $R$-modules in the following two cases:

\begin{itemize}
\item[(a)] $R=\dbZ/n\dbZ$ for some $n\in\dbN$
\item[(b)] $R=F[x,y]$ for some field $F$.
\end{itemize}

\vskip .1cm
{\bf Problem 2.} DF, Problem 8 on page 455.
\vskip .1cm

\vskip .1cm

{\bf Problem 3*.} Prove Theorem~N1 from Lecture 8: Let $R$ be a ring (with 1), let $M$ be an $R$-module
and $N$ its submodule. Prove that $M$ is Noetherian $\iff$
$N$ and $M/N$ are both Noetherian.
\skv

{\bf Problem 4.} Let $A$ be a ring (with 1). A subring $B$ of $A$ is called a
\emph{retract} if there exists a surjective ring homomorphism
$\phi: A\to B$ such that $\phi_{| B}=id_B$, that is, $\phi(b)=b$ for all $b\in B$.

Now let $M$ and $N$ be two $R$-modules. Prove that the tensor algebra
$T(M)$ is (naturally isomorphic to) a subalgebra of $T(M\oplus N)$ and that this subalgebra is a retract. Also prove the analogous statement about the symmetric algebras.

\skv
{\bf Problem 5:} \rm Define the rank of an $R$-module $M$, denoted by $\rk(M)$, to be the minimal number of generators (WARNING: this definition is different from the definition in DF; the two definitions coincide for free modules over commutative rings).
\begin{itemize}
\item[(a)*] Let $R$ be a PID,
$M$ be a finitely generated $R$-module and
$R/a_1R\oplus \ldots \oplus R/a_mR\oplus R^s$ its invariant factor
decomposition, that is, $a_1,\ldots, a_m$ are nonzero 
non-units and $a_1\mid a_2\mid\ldots\mid a_m$. Prove that
$\rk(M)=m+s$. {\bf Warning:} It is not true in general that
$\rk(P\oplus Q)=\rk(P)+\rk(Q)$. 

\item[(b)] Again let $R$ be a PID. Let $F$ be a free $R$-module of rank $n$ with basis 
$e_1,\ldots, e_n$, let $N$ be the submodule of $F$
generated by some elements $v_1,\ldots, v_n\in F$,
and let $A\in Mat_n(F)$ be the matrix such that
$$\left(\begin{array}{c} v_1 \\ \vdots \\ v_n\end{array}\right)=
A \left(\begin{array}{c} e_1 \\ \vdots \\ e_n\end{array}\right)$$ 
Find a simple condition on the entries of $A$ which holds
if and only if $\rk(F/N)=n$.
\end{itemize}

\newpage

{\bf Hint for 3:} The forward direction is easy. For the backwards direction,
observe that if $\{P_i\}$ is an ascending chain of submodules of $M$,
then $\{P_i\cap N\}$ is an ascending chain of submodules of $N$
and $\{(P_i+N)/N\}$ is an ascending chain of submodules of $M/N$.

\newpage


{\bf Hint for 5(a):} Let $p$ be a prime
dividing $a_1$. How is $M$ related to $M'=(R/pR)^{m+s}$
and what is $\rk(M')$ (and why)?
\end{document}
