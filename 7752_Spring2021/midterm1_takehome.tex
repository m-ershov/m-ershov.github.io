\documentclass[12pt]{amsart}

\usepackage{amsmath}
\usepackage{amssymb}
\usepackage{amsthm}
\usepackage{url}
\usepackage{hyperref}

%\usepackage{psfig}

\begin{document}
\baselineskip=16pt
%\textheight=9.6in
%\parindent=0pt
\def\sk {\hskip .5cm}
\def\skv {\vskip .12cm}
\def\cos {\mbox{cos}}
\def\sin {\mbox{sin}}
\def\tan {\mbox{tan}}
\def\intl{\int\limits}
\def\lm{\lim\limits}
\newcommand{\frc}{\displaystyle\frac}
\def\xbf{{\mathbf x}}
\def\fbf{{\mathbf f}}
\def\gbf{{\mathbf g}}

\def\dbA{{\mathbb A}}
\def\dbB{{\mathbb B}}
\def\dbC{{\mathbb C}}
\def\dbD{{\mathbb D}}
\def\dbE{{\mathbb E}}
\def\dbF{{\mathbb F}}
\def\dbG{{\mathbb G}}
\def\dbH{{\mathbb H}}
\def\dbI{{\mathbb I}}
\def\dbJ{{\mathbb J}}
\def\dbK{{\mathbb K}}
\def\dbL{{\mathbb L}}
\def\dbM{{\mathbb M}}
\def\dbN{{\mathbb N}}
\def\dbO{{\mathbb O}}
\def\dbP{{\mathbb P}}
\def\dbQ{{\mathbb Q}}
\def\dbR{{\mathbb R}}
\def\dbS{{\mathbb S}}
\def\dbT{{\mathbb T}}
\def\dbU{{\mathbb U}}
\def\dbV{{\mathbb V}}
\def\dbW{{\mathbb W}}
\def\dbX{{\mathbb X}}
\def\dbY{{\mathbb Y}}
\def\dbZ{{\mathbb Z}}

\def\la{{\langle}}
\def\ra{{\rangle}}

\def\Aut{{\rm Aut}}
\def\End{{\rm End}}
\def\Inn{{\rm Inn}}
\def\Ker{{\rm Ker}}
\def\Im{{\rm Im\,}}
\def\phi{{\varphi}}

\bf\centerline{Algebra-II, Spring 2021. Midterm \#1}\rm
\skv
\bf\centerline{due by 11:59pm on Tuesday Mar 16th}\rm
\vskip .3cm
{\bf Directions: } Provide complete arguments
(do not skip steps). State clearly any result you are referring to. Partial credit for
incorrect solutions, containing steps in the right direction, may be given.
\vskip .1cm

{\bf Rules: } You are not allowed to discuss midterm problems with each other.
You may ask me any questions about the problems (e.g. if the formulation is unclear),
but as a rule I will only provide minor hints. You may freely use the following resources:
\begin{itemize}
\item[(i)] the book by Dummit and Foote
\item[(ii)] your class notes (including notes from 7751)
\item[(iii)] your previous assignments (homeworks and midterms)
\item[(iv)] any materials posted on the Math 7751/7752 collab sites and any materials posted on \url{http://people.virginia.edu/~mve2x/}
\end{itemize}

The use of any other resources is prohibited and will be considered a violation of the UVA honor code.


\skv
{\bf Scoring:} The exam contains 5 problems, each of which is worth 12 points. If $s_1\geq s_2\geq s_3\geq s_4\geq s_5$ are
your scores on individual problems in decreasing order, your total will be $s_1+s_2+s_3+s_4+\max\{s_5-8,0\}$. Thus, the maximal
possible total is 52, but the score of 50 will count as 100\%.


\skv
\skv
{\bf 1.} $\empty$
\begin{itemize}
\item[(a)] Let $R$ be a commutative domain (with $1$)
and $F$ the field of fractions of $R$. Let $M$ and $N$ be $F$-modules,
and let $\phi:M\to N$ be an isomorphism of $R$-modules. Prove that
$\phi$ must be an isomorphism of $F$-modules.

\item[(b)] Give an example of finitely generated $\dbC$-modules
$M$ and $N$ and a map $\phi:M\to N$ such that $\phi$ is an isomorphism
of $\dbR$-modules but not an isomorphism of $\dbC$-modules.

\item[(c)] Let $M$ and $N$ be finitely generated $\dbC$-modules,
and suppose that $M$ and $N$ are isomorphic as $\dbR$-modules.
Prove that $M$ and $N$ are isomorphic as $\dbC$-modules.
\end{itemize} \skv
\newpage

{\bf 2.} Let $R$ be a commutative ring with $1$, and let
$M,N$ and $L$ be $R$-modules.
\begin{itemize}
\item[(a)] Suppose that $L$ is a quotient module of $M$. Prove that $L\otimes_R N$ is 
(isomorphic to) a quotient module of $M\otimes_R N$. You are NOT allowed to use Dummit and Foote for this question (class and online notes should be sufficient).
\item[(b)] Suppose that $M$ is finitely generated and $N$ is Noetherian.
Prove that the $R$-module $M\otimes_R N$ is Noetherian.
{\bf Hint:} Start with the case when $M$ is a free $R$-module.
\item[(c)] Suppose that $L$ is a submodule of $M$. Is it always true
(for any $R,M,N$ and $L$) that $M\otimes_R N$ contains a submodule 
isomorphic to $L\otimes_R N$? Prove or give a counterexample.
\end{itemize}

\skv{\bf 3.} Let $R$ be a commutative ring (with 1) and let $M$ and $N$
be $R$-modules.
\begin{itemize}
\item[(a)] Give an example showing that $T(M\oplus N)$ need not be isomorphic
to $T(M)\otimes T(N)$ as rings. {\bf Hint:} Look for an example where
$T(M)\otimes T(N)$ satisfies certain nice algebraic property, while
$T(M\oplus N)$ does not.
\item[(b)] Prove that $S(M\oplus N)$ is isomorphic to $S(M)\otimes S(N)$
as $R$-algebras. {\bf Hint:} Use the universal property of symmetric algebras
to construct a map in one direction and results from previous homeworks 
to construct a map in the opposite direction. Then prove that the two maps
are mutually inverse.
\item[(c)] Now assume that $R$ is a field and $M$ and $N$ are finite-dimensional
over $R$. Prove that $\bigwedge (M\oplus N)$ is isomorphic to
$\bigwedge(M)\otimes \bigwedge(N)$ as $R$-modules but not necessarily
as rings.
\end{itemize}
\skv
{\bf 4.} Let $V$ be a finite-dimensional vector space over a field $F$,
and let $T:V\to V$ be an $F$-linear map. Let $m$ be the number of invariant factors
of $T$.
\begin{itemize}
\item[(a)] Prove that $m=1$ if and only if there exists $v\in V$ such that
the smallest $T$-invariant subspace of $V$ containing $v$ is the entire $V$.
\item[(b)] Assume that $m=1$ and $a(x)=x^6-1$ is the (unique) invariant factor
of $T$. Find the number of distinct $T$-invariant subspaces of $V$
in each of the following 4 cases: $F=\dbR$, $F=\dbC$, $F=\dbF_2$ (field with
$2$ elements) and $F=\dbF_3$. 
 \end{itemize}
{\bf Hint:} Rephrase both (a) and (b) as questions about $F[x]$-modules.
In (b), the answers in the 4 cases are all finite and all different.
\skv
\skv
{\bf 5.} Let $F$ be an algebraically closed field, let $A\in Mat_n (F)$
for some $n\in\dbN$, and let $C$ be the centralizer of $A$ in $Mat_n (F)$.
Prove that $$dim_F(C)\geq n.$$ {\bf Hint:} First assume that $A$ is in Jordan
canonical form and has just one Jordan block; then consider the case when
$A$ is an arbitrary matrix in Jordan canonical form, and finally prove
the statement for general $A$.

\end{document}
