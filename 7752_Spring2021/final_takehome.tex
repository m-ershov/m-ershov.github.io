\documentclass[12pt]{amsart}

\usepackage{amsmath}
\usepackage{amssymb}
\usepackage{amsthm}
\usepackage{url}
\usepackage{hyperref}

%\usepackage{psfig}

\begin{document}
%\baselineskip=16pt
%\textheight=9.6in
\parindent=0pt
\def\sk {\hskip .5cm}
\def\skv {\vskip .12cm}
\def\cos {\mbox{cos}}
\def\sin {\mbox{sin}}
\def\tan {\mbox{tan}}
\def\intl{\int\limits}
\def\lm{\lim\limits}
\newcommand{\frc}{\displaystyle\frac}
\def\xbf{{\mathbf x}}
\def\fbf{{\mathbf f}}
\def\gbf{{\mathbf g}}

\def\dbA{{\mathbb A}}
\def\dbB{{\mathbb B}}
\def\dbC{{\mathbb C}}
\def\dbD{{\mathbb D}}
\def\dbE{{\mathbb E}}
\def\dbF{{\mathbb F}}
\def\dbG{{\mathbb G}}
\def\dbH{{\mathbb H}}
\def\dbI{{\mathbb I}}
\def\dbJ{{\mathbb J}}
\def\dbK{{\mathbb K}}
\def\dbL{{\mathbb L}}
\def\dbM{{\mathbb M}}
\def\dbN{{\mathbb N}}
\def\dbO{{\mathbb O}}
\def\dbP{{\mathbb P}}
\def\dbQ{{\mathbb Q}}
\def\dbR{{\mathbb R}}
\def\dbS{{\mathbb S}}
\def\dbT{{\mathbb T}}
\def\dbU{{\mathbb U}}
\def\dbV{{\mathbb V}}
\def\dbW{{\mathbb W}}
\def\dbX{{\mathbb X}}
\def\dbY{{\mathbb Y}}
\def\dbZ{{\mathbb Z}}

\def\la{{\langle}}
\def\ra{{\rangle}}

\def\Aut{{\rm Aut}}
\def\End{{\rm End}}
\def\Inn{{\rm Inn}}
\def\Gal{{\rm Gal}}
\def\Ker{{\rm Ker}}
\def\Im{{\rm Im\,}}
\def\phi{{\varphi}}

\bf\centerline{Algebra-II, Spring 2021. Final exam}\rm
\skv
\bf\centerline{due by 12pm on Saturday May 15th}\rm
\vskip .3cm
{\bf Directions: } Provide complete arguments
(do not skip steps). State clearly any result you are referring to. Partial credit for
incorrect solutions, containing steps in the right direction, may be given.
\vskip .1cm

{\bf Rules: } You are not allowed to discuss midterm problems with each other.
You may ask me any questions about the problems (e.g. if the formulation is unclear),
but as a rule I will only provide minor hints. You may freely use the following resources:
\begin{itemize}
\item[(i)] the book by Dummit and Foote
\item[(ii)] your class notes (including notes from 7751)
\item[(iii)] your previous assignments (homeworks and midterms)
\item[(iv)] any materials posted on the Math 7751/7752 collab sites and any materials posted on \url{http://people.virginia.edu/~mve2x/}
\end{itemize}

The use of any other resources is prohibited and will be considered a violation of the UVA honor code.


\skv
{\bf Scoring:} The exam contains 6 problems, each of which is worth 15 points. If $s_1\geq s_2\geq s_3\geq s_4\geq s_5\geq s_6$ are
your scores on individual problems in decreasing order, your total will be $s_1+s_2+s_3+s_4+s_5+\max\{s_6-10,0\}$. Thus, the maximal
possible total is 80, but the score of 75 will count as 100\%.


\skv

{\bf Problem 1: } Let $F$ be an algebraically closed field with ${\rm char}\, {F}\neq 2$. 
\begin{enumerate}
\item[(a)] Let $J$ be a Jordan block of size $n$ with eigenvalue $\lambda$ over $F$. Determine the Jordan canonical form of the matrix $J^2$. {\bf Hint: } Consider the cases $\lambda=0$ and $\lambda\neq 0$ separately.  
\item[(b)] Let $A\in Mat_4(F)$. Determine necessary and sufficient conditions for $A$ to have a square root, i.e. for there to exist a matrix $B\in Mat_4(F)$ such that $A=B^2$. State your answer in the form: $A$ has a square root $\iff$ $JCF(A)$ satisfies certain conditions. Make sure to prove your answer. 
\end{enumerate}

\skv
{\bf Problem 2: } Let $f(x)=(x^2-2)(x^3-3)\in\dbQ[x]$ and let $K\subseteq \dbC$  be the splitting field of $f(x)$ over $\dbQ$.
\begin{itemize}
\item[(a)] Prove that $\Gal(K/\dbQ)\cong S_3\times \dbZ/2\dbZ$
\item[(b)] Describe explicitly all subfields $L$ of $K$ with $[L:\dbQ]=6$ by giving explicit generators for each subfield and determine the corresponding subgroup of $\Gal(K/\dbQ)$ for each such $L$. Make sure to prove that you found all such subfields/subgroups.
%\item[(c)] Find a primitive element for $K$ over $\dbQ$ (and prove
%your answer).
\end{itemize}

\newpage
{\bf Problem 3: } Let $K/F$ be a finite Galois extension and $G=\Gal(K/F)$.
\begin{itemize}
\item[(a)] Assume  that $G$ is a simple group, let $\alpha\in K\setminus F$
and $\mu_{\alpha,F}(x)$ the minimal polynomial of $\alpha$ over $F$.
Prove that $K$ is a splitting field for $\mu_{\alpha,F}(x)$.

\item[(b)] Let $n=[K:F]$, and fix integers $m$ and $l$ with $ml=n$.
Find a condition on the \underline{subfield lattice of $K/F$} which is equivalent
to the following: $G$ can be written as a semidirect product $G=A\rtimes B$
for some subgroups $A$ and $B$ where $|A|=m$ and $|B|=l$. Make sure to prove your answer.


\skv
\noindent
{\bf Note:} your answer should be stated in terms of subfields of $K/F$ and the integers $m$ and $l$; you can refer to properties like `normal extension' or `Galois extension', but you cannot mention any groups in your characterization.
\end{itemize}
\skv
{\bf Problem 4: }In this problem $p$ denotes a fixed prime number. Let $P$ denote the set of all prime numbers. Recall from Lecture~23 that
a {\it supernatural number} is a formal product $\alpha=\prod\limits_{q\in P}q^{a_q}$ where each $a_q\in\dbZ_{\geq 0}\cup\{\infty\}$.
Note that natural numbers can be identified with supernatural numbers for which all $a_q$ are finite and only finitely many $a_q$ are nonzero.
The goal of this problem is to construct a natural bijection between supernatural numbers and subfields of $\overline {\dbF_p}$
(a fixed algebraic closure of $\dbF_{p}$).
\skv
For each $i\in\dbN$ let $F_i=\dbF_{p^i}\subseteq \overline{\dbF_p}$. Given a subfield $L$ of $ \overline{\dbF_p}$, define
$I(L)=\{i\in\dbN: F_i\subseteq L\}$.
\begin{itemize}
\item[(a)] Prove that for any subfield $L$ of $\overline{\dbF_p}$ we have $L=\bigcup\limits_{i\in I(L)} F_i$.
\item[(b)] Suppose that $I=I(L)$ for some subfield $L$ of $\overline{\dbF_p}$. Prove that $I$ satisfies the following two conditions:
\begin{itemize}
\item[(i)] $I$ is closed under divisors: for any $i\in I$, any positive divisor of $I$ also lies in $i$
\item[(ii)] $I$ is closed under least common multiples: for any $i,j\in I$ we have $LCM(i,j)\in I$
\end{itemize}
\item[(c)] Now let $I$ be a subset of $\dbN$ satisfying conditions (i) and (ii) from (b), and let $L=\bigcup\limits_{i\in I}F_i$. 
Prove that $L$ is a subfield and that $I(L)=I$ (it is clear that $I(L)$ contains $I$, but you need to prove the equality!)
\item[(d)] Deduce from (b) and (c) that there exists a bijection between subfields of $ \overline{\dbF_p}$ and subsets of $\dbN$
satisfying (i) and (ii)
\item[(e)] Now construct a natural bijection between subsets of $\dbN$ satisfying (i) and (ii) and supernatural numbers. Your bijection
should satisfy two conditions: 
\begin{itemize}
\item[(iii)] finite subsets of $\dbN$ correspond to natural numbers
\item[(iv)] if $I$ and $J$ are two subsets of $\dbN$ satisfying (i) and (ii) and $\alpha_I=\prod\limits_{q\in P}q^{a_q}$ and
$\beta_I=\prod\limits_{q\in P}q^{b_q}$ are the corresponding supernatural numbers, then $I\subseteq J$ $\iff$ $a_q\leq b_q$ for all $q\in P$.
\end{itemize}  
\end{itemize}  
{\bf Problem 5: } [DF, Problem 12, page 654(a)-(c)]. Let $K$ be a subfield of $\dbC$ maximal with respect to the property ``$\sqrt{2}\not\in K$.''
\begin{itemize}
\item[(a)] Show that such a field $K$ exists.
\item[(b)] Show that $\dbC$ is algebraic over $K$. {\bf Hint:} Assume that there exists $\alpha\in \dbC$ which is transcendental over $K$
and consider the extension $K(\alpha)/K$.
\item[(c1)] Let $L/K$ be a finite Galois extension, with $L\subseteq \dbC$. Prove that $\Gal(L/K)$ is cyclic of order $2^k$ for some $k\in\dbZ_{\geq 0}$. {\bf Hint:} What can you say about the subfield lattice of $L/K$ based on the definition of $K$? Use Galois correspondence to translate this property into a condition on $\Gal(L/K)$, and then show that the only groups satisfying that condition are cyclic groups of
$2$-power order.
\item[(c2)] Now use (c) to prove that any finite extension $L/K$, with $L\subseteq \dbC$, is Galois.      
\end{itemize}
\skv
{\bf Problem 6: } Let $F$ be a field, with ${\rm char\,}(F)\neq 2$ 
and $$A=F[x,y,z]/(x^2+y^2+z^2-4).$$ For every polynomial $p\in F[x,y,z]$ 
we denote by $\bar p$ the image of $p$ in $A$. It is easy to see that $A$ is a domain (you need not prove this), so we can consider its field of fractions
$K=Frac(A)$. 

 \begin{itemize}
 \item[(a)] Let $a=\bar x+\bar y+\bar z$, $b=\bar x\bar y+ \bar x\bar z + \bar y\bar z$
and $c=\bar x\bar y\bar z$, and let $L=F(a,b,c)$, the subfield of $K$ generated by $F,a,b$ and $c$. 
Prove that $K/L$ is a finite Galois extension and $\Gal(K/L)\cong S_3$. 
\skv
\noindent
{\bf Hint:} First find a faithful action of $S_3$ on $K$ such that $L\subseteq K^{S_3}$.
Then find a cubic polynomial $f(t)\in L[t]$ such that $K$ is a splitting field for $f$ over $L$. Then deduce (a) from these two results
(and some theorems proved in class)
\item[(b)] Prove that the extension $L/F$ is purely transcendental and find an explicit transcendence basis for it. 
\end{itemize} 

\end{document}
