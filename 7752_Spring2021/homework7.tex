\documentclass[12pt]{amsart}

\usepackage{amsmath}
\usepackage{amssymb}
\usepackage{amsthm}
\usepackage{url}
\usepackage{hyperref}
%\usepackage{psfig}

\begin{document}
\baselineskip=15pt
\textheight=8.4in
\parindent=0pt
\def\sk {\hskip .5cm}
\def\skv {\vskip .12cm}
\def\cos {\mbox{cos}}
\def\sin {\mbox{sin}}
\def\tan {\mbox{tan}}
\def\intl{\int\limits}
\def\lm{\lim\limits}
\newcommand{\frc}{\displaystyle\frac}
\def\xbf{{\mathbf x}}
\def\fbf{{\mathbf f}}
\def\gbf{{\mathbf g}}

\def\Ker{{\rm Ker\,}}
\def\phi{\varphi}

\def\dbA{{\mathbb A}}
\def\dbB{{\mathbb B}}
\def\dbC{{\mathbb C}}
\def\dbD{{\mathbb D}}
\def\dbE{{\mathbb E}}
\def\dbF{{\mathbb F}}
\def\dbG{{\mathbb G}}
\def\dbH{{\mathbb H}}
\def\dbI{{\mathbb I}}
\def\dbJ{{\mathbb J}}
\def\dbK{{\mathbb K}}
\def\dbL{{\mathbb L}}
\def\dbM{{\mathbb M}}
\def\dbN{{\mathbb N}}
\def\dbO{{\mathbb O}}
\def\dbP{{\mathbb P}}
\def\dbQ{{\mathbb Q}}
\def\dbR{{\mathbb R}}
\def\dbS{{\mathbb S}}
\def\dbT{{\mathbb T}}
\def\dbU{{\mathbb U}}
\def\dbV{{\mathbb V}}
\def\dbW{{\mathbb W}}
\def\dbX{{\mathbb X}}
\def\dbY{{\mathbb Y}}
\def\dbZ{{\mathbb Z}}

\def\Aut{{\rm Aut}}
\def\exp{{\rm exp}}
\def\det{{\rm det}}
\def\tr{{\rm tr}}

\def\la{{\langle}}
\def\ra{{\rangle}}
\def\rk{{\rm rk}}


\bf\centerline{Homework Assignment \# 7}\rm
\vskip .2cm
{\bf Plan for the week of Mar 22:} Primitive element theorem (online Lecture 18). Structure of finite fields (see the first page and a half of online Lecture 22). Galois extensions and Galois groups (online lecture 19, Section 14.1 in DF), maybe start Galois correspondence (online lecture 20, Section 14.2 in DF).
\vskip .1cm
Here and in all future assignments ``online'' or ``online notes'' refers to Algebra-II lectures posted on my Spring 2010 Algebra-II webpage
\vskip .1cm
\centerline{\url{http://people.virginia.edu/~mve2x/7752_Spring2010/}}
\vskip .1cm

%{\bf Note on hints:} All hints are given at the end of the %assignment, each on a separate page.
%Problems (or parts of problems) for which hint is available are %marked with *.

\vskip .3cm

\bf\centerline{Problems, due by 11:59pm on Friday, March 26th.}\rm
\vskip .1cm
\skv
{\bf Problem 1 (practice):} \rm Let $F$ be a field and $\Omega$ a subset of $F[x]$.
Use the existence and uniqueness of algebraic closures to prove that
there exists a unique splitting field for $\Omega$ over $F$ up to
$F$-isomorphism. {\bf Hint:} Use the fact that any splitting field 
$K$ for $\Omega$ lies in some algebraic closure of $F$ (this was proved at the beginning
of Lecture~12 on March 16).
\skv

{\bf Problem 2:} Let $K/F$ be a field extension, and let $K_1$ and $K_2$
be subfields of $K$ containing $F$ such that the extensions
$K_1/F$ and $K_2/F$ are normal. Prove that the extensions
$K_1 K_2/F$ and $K_1\cap K_2/F$ are also normal.
Here $K_1 K_2$ is the {\it composite field of $K_1$ and $K_2$} (see  pp. 528-529 of DF for the definition).
\skv

\skv
{\bf Problem 3:} \rm
Prove the interesting part of Corollary 18.7 form online notes: if $K/L/F$ is a tower of algebraic extensions and $K/L$ and $L/F$ are separable, then $K/F$ is separable (see online notes for a hint). Give a detailed argument.
\skv

{\bf Problem 4:} 
\begin{itemize}
\item[(a)] Prove that if  $K/\dbQ$ is any field extension,
then any field automorphism of $K$ must fix $\dbQ$ elementwise.

\item[(b)] Prove that $\Aut(\dbR)$, the group of field automorphism of $\dbR$, is trivial.
See [DF, Problem~7, p.567] for a sketch of the proof.
\end{itemize}
\skv
Before doing problems 5-7 below read \S~13.6 in DF.
\skv
{\bf Problem 5:} \rm 
Let $p$ be a prime, $n\geq 2$ an integer, $f(x)=x^n-p$, and let
$K\subset \dbC$ be the splitting field for $f(x)$ over $\dbQ$.
\begin{itemize}
\item[(a)] Prove that $K=\dbQ(\sqrt[n]{p},\omega_n)$ where
$\omega_n=e^{2\pi i/n}$.
\item[(b)] Prove that $[K:\dbQ]\leq n\phi(n)$ where $\phi$
is the Euler function.
\item[(c)] Assume that $n$ is prime. Prove that inequality in (a)
is equality.
\item[(d)] Let $p=3$ and $n=12$. Prove that inequality in (a)
is strict and find $[K:\dbQ]$. {\bf Hint:} Compute $\omega_{12}$
explicitly.
\end{itemize}
\skv
{\bf Problem 6:} Problem 4 in [DF, p.555]
\skv
{\bf Problem 7:} Problem 5 in [DF, p.555]
\end{document}
