\documentclass[11pt]{amsart}

\usepackage{amsmath}
\usepackage{amssymb}
\usepackage{amsthm}
\usepackage{url}
\usepackage{hyperref}

%\usepackage{psfig}

\begin{document}
\baselineskip=16pt
\textheight=8.5in
\parindent=0pt 
\def\sk {\hskip .5cm}
\def\skv {\vskip .08cm}
\def\cos {\mbox{cos}}
\def\sin {\mbox{sin}}
\def\tan {\mbox{tan}}
\def\intl{\int\limits}
\def\lm{\lim\limits}
\newcommand{\frc}{\displaystyle\frac}
\def\xbf{{\mathbf x}}
\def\fbf{{\mathbf f}}
\def\gbf{{\mathbf g}}

\def\dbA{{\mathbb A}}
\def\dbB{{\mathbb B}}
\def\dbC{{\mathbb C}}
\def\dbD{{\mathbb D}}
\def\dbE{{\mathbb E}}
\def\dbF{{\mathbb F}}
\def\dbG{{\mathbb G}}
\def\dbH{{\mathbb H}}
\def\dbI{{\mathbb I}}
\def\dbJ{{\mathbb J}}
\def\dbK{{\mathbb K}}
\def\dbL{{\mathbb L}}
\def\dbM{{\mathbb M}}
\def\dbN{{\mathbb N}}
\def\dbO{{\mathbb O}}
\def\dbP{{\mathbb P}}
\def\dbQ{{\mathbb Q}}
\def\dbR{{\mathbb R}}
\def\dbS{{\mathbb S}}
\def\dbT{{\mathbb T}}
\def\dbU{{\mathbb U}}
\def\dbV{{\mathbb V}}
\def\dbW{{\mathbb W}}
\def\dbX{{\mathbb X}}
\def\dbY{{\mathbb Y}}
\def\dbZ{{\mathbb Z}}

\def\la{{\langle}}
\def\ra{{\rangle}}
\def\summ{{\sum\limits}}
\def\eps{{\varepsilon}}
\def\lam{{\lambda}}

\bf\centerline{Homework \#2. Due on Thursday, September 11th, 11:59pm on Canvas}\rm
\vskip .1cm

\bf\centerline{Plan for next week:}\rm
\skv
Below Pugh stands for `Real Mathematical Analysis' by Charles Pugh, Rudin for `Principles of Mathematical Analysis' by Walter Rudin, Tao I for `Real Analysis I' by Terrence Tao and Tao II for `Real Analysis II' by Terrence Tao. Pugh, Tao I and Tao II are freely available via UVA subscription (use Springer Link).
\skv

1. On {\bf Mon, Sep 8}, we will first wrap up our discussion of countable and uncountable sets. Then we will define metric spaces as well as open and closed sets in metric spaces and start talking about their basic properties. The references for the metric space part are 2.1 and 2.3 in Pugh, 2.2 in Rudin and 1.1 and 1.2 in Tao II (note that subsections are not explicitly numbered in Rudin).

2. On {\bf Wed, Sep 10}, we will continue talking about open and closed sets (and some related concepts). If time allows, we will also talk about convergence of sequences in metric spaces. The references are 2.1 and 2.3 in Pugh, 3.1-3.3 in Rudin and 1.3 and 1.4 in Tao II.
\skv
\skv

\skv
\bf\centerline{Problems: }\rm
\skv
{\bf Note on hints and practice problems: } Most hints are given at the end of the assignment, each on a separate page.
Problems (or parts of problems) for which hint is available are marked with *. Parts of probelms marked as `practice'
should be completed, but need not included in your submission.
\skv
\hskip .5cm
The first three problems in this assignment discuss various parts of the construction of $\dbR$ as equivalence classes of Cauchy sequences of rational numbers. As in Lecture~2, $\dbR^*$ denotes the set of all Cauchy sequence or rational numbers and $\sim$ is the equivalence relation on $\dbR^*$ given by $$(x_n)\sim (y_n)\iff \lim\limits_{n\to\infty}(x_n-y_n)=0$$ and $\dbR=\dbR^*/\sim$ is the set of equivalence classes with respect to $\sim$
(one needs to prove that $\sim$ is an equivalence relation, but this is not part of any of the homework problems). Also recall that the equivalence
class of a sequence $(x_n)$ is denoted by $[x_n]$. 
\skv
{\bf 1.} Recall that the addition and multiplication on $\dbR$ are defined by $[x_n]+[y_n]=[x_n+y_n]$ and $[x_n] \cdot [y_n]=[x_n y_n]$. Prove that these operations are well defined (as we discussed in class, there are two separate things to check). You are allowed to assume any standard properties of limits within $\dbQ$ (that is, properties applying to sequences of rational numbers which converge to a rational number), but do not assume anything about Cauchy sequences.
\skv
{\bf 2.}
\begin{itemize}
\item[(a)] (practice) Prove Lemma~2.5 from online Lecture~2. {\bf Hint:} argue by contrapositive -- assume that neither (i) nor (ii) holds and deduce
that $\lim\limits_{n\to\infty}(x_n-y_n)=0$. 
\item[(b)] Use (a) to prove that $\dbR$ is an ordered field (Theorem~2.6)
\end{itemize}
\skv
{\bf 3.} The goal of this problem is to prove Theorem~2.10 from online Lecture~3 which asserts that $\dbR$ has the LUB property. 
\begin{itemize}
\item[(a)] For each $n\in\dbZ_{\geq 0}$ let $Z_n=\{z\in\dbQ: z=\frac{m}{2^n}: m\in\dbZ\}$. Use the well-ordering principle to show that any
subset of $Z_n$ which is non-empty and bounded below has the minimal element (the well-ordering principle asserts that any non-empty subset of
$\dbN$ has the minimal element).
\item[(b)] Now let $A$ be a subset of $\dbR$ which is non-empty and bounded above. For each $n\in\dbN$ let
$U_n=\{z\in Z_n: z\mbox{ is an upper bound for }A\}$. Prove that each $U_n$ is non-empty and bounded from below and deduce from (a)
that $U_n$ has the smallest element; denote it by $x_n$.
\item[(c)] Prove that for each $n\in\dbN$ we have $x_n-\frac{1}{2^n}<x_{n+1}\leq x_n$. Deduce that the sequence $(x_n)$ is Cauchy.
\item[(d)] Now let $x=[x_n]$, the equivalence class of the sequence $(x_n)$ from part (c). Prove that $x$ is the least upper bound for $A$.
{\bf Hint:} this can be proved directly from definition of the least upper bound, that is, first show that $x$ is an upper bound for $A$
and then show that if $y$ is any upper bound for $A$, then $x\leq y$. Because of how inequalities in $\dbR$ were defined, it may
be convenient to prove a non-strict inequality $u\leq v$ by contradiction, that is, show that the inequality $v>u$ cannot hold.
\end{itemize}
\skv
The next 3 problems deal with countable and uncountable sets.

\skv
{\bf Definition:} Two sets $X$ and $Y$ are said to have the same cardinality
if there is a bijection from $X$ to $Y$.
\skv
{\bf 4.} Let $A$ be an uncountable set and $B$ a countable subset of $A$.
\begin{itemize}
\item[(a)] Prove that $A\setminus B$ is uncountable.
\item[(b)*] Prove that $A$ and $A\setminus B$ have the same cardinality.
\end{itemize}
\skv
{\bf 5*.} Let $X$ and $Y$ be any sets, and define $X^Y$ to be the
set of all functions $f: Y\to X$. Prove that if $|X|\geq 2$, then
$Y$ and $X^Y$ do not have the same cardinality. 
\skv
{\bf 6.} A real number $\alpha$ is called algebraic if $\alpha$ is a root of a (nonzero) polynomial with 
{\bf integer} coefficients, that is, if there exist integers $c_0,\ldots,c_n$, not all $0$ such that $\sum\limits_{k=0}^n c_k \alpha^k=0$. Note that all rational numbers are algebraic (if $\alpha=\frac{p}{q}$, then $\alpha$ is a root of the polynomial $qx-p$), but many irrational numbers are algebraic as well (e.g. $\sqrt{2}$ is algebraic as $\sqrt{2}$ is a root of $x^2-2$). 
Prove that the set of all algebraic numbers is countable.
\skv
If you need a hint, see Problem~1.39 in Pugh or 2.2 in Rudin or Problem 6 in
\skv
\centerline{
\url{https://m-ershov.github.io/3000_Spring2018/homework11.pdf}
}
\skv
{\bf 7.} Let $a\leq b$ be real numbers and $X=C[a,b]$, the set of all continuous functions
from $[a,b]$ to $\dbR$. Define the functions $d_{unif}:X\times X\to\dbR_{\geq 0}$ and $d_{int}:X\times X\to\dbR_{\geq 0}$ 
by $$d_{unif}(f,g)=\max\limits_{x\in X} |f(x)-g(x)| \mbox{ and }d_{int}(f,g)=\int\limits_{a}^b |f(x)-g(x)|\,dx.$$
\begin{itemize}
\item[(a)] Prove that $(X,d_{unif})$ is a metric space (the metric $d_{unif}$ is called the {\bf uniform metric})
\item[(b)] (practice) Prove that $(X,d_{int})$ is a metric space (the metric $d_{int}$ is called the {\bf integral metric})
\end{itemize}
\skv

{\bf 8*.} Let $X=C[a,b]$ and $d=d_{unif}$ (as defined in Problem~7). Find an (infinite) sequence $f_1,f_2,\ldots$ of elements of $X$ such that $d(f_i,f_j)=1$ for all $i\neq j$.
\newpage
{\bf Hint for 4(b):} Choose any countably infinite subset $C$ of $A\setminus B$ and then
use things proved in class to show that the identity map 
$f:(A\setminus B)\setminus C\to (A\setminus B)\setminus C$ can be extended
to a bijection from $A\setminus B$ and $A$. Draw a picture!
\newpage
{\bf Hint for 5:} Note that if $X=\{0,1\}$ and $Y=\dbN$, then $X^Y$ is precisely the set of infinite sequences of $0$ and $1$,
so the assertion of the problem in this special case holds by Theorem~3.8 (online Lecture~3). To prove the general case imitate the proof of 
Theorem~3.8. See also Problem~1.38 in Pugh.
\newpage
{\bf Hint for 8:} If we replace $C[a,b]$ by $B[a,b]$, the set of all bounded functions from $[a,b]$ to $\dbR$ and define the metric $d$ on $B[a,b]$ by $$d(f,g)=\sup\limits_{x\in [a,b]}|f(x)-g(x)|,$$ then the analogous question would have a very simple answer, e.g. we could let $f_n=I_{1/n}$ where $I_c$ (for a fixed $c\in\dbR$) is the function defined by $I_c(c)=1$ and $I_c(x)=0$ for $x\neq c$. To solve  Problem~7 think of a suitable way to approximate $I_c$ by a continuous function. 
\end{document}