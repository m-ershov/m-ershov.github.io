\documentclass[11pt]{amsart}

\usepackage{amsmath}
\usepackage{amssymb}
\usepackage{amsthm}
\usepackage{url}
\usepackage{hyperref}

%\usepackage{psfig}

\begin{document}
\baselineskip=16pt
\textheight=8.5in
\parindent=0pt 
\def\sk {\hskip .5cm}
\def\skv {\vskip .08cm}
\def\cos {\mbox{cos}}
\def\sin {\mbox{sin}}
\def\tan {\mbox{tan}}
\def\intl{\int\limits}
\def\lm{\lim\limits}
\newcommand{\frc}{\displaystyle\frac}
\def\xbf{{\mathbf x}}
\def\fbf{{\mathbf f}}
\def\gbf{{\mathbf g}}

\def\dbA{{\mathbb A}}
\def\dbB{{\mathbb B}}
\def\dbC{{\mathbb C}}
\def\dbD{{\mathbb D}}
\def\dbE{{\mathbb E}}
\def\dbF{{\mathbb F}}
\def\dbG{{\mathbb G}}
\def\dbH{{\mathbb H}}
\def\dbI{{\mathbb I}}
\def\dbJ{{\mathbb J}}
\def\dbK{{\mathbb K}}
\def\dbL{{\mathbb L}}
\def\dbM{{\mathbb M}}
\def\dbN{{\mathbb N}}
\def\dbO{{\mathbb O}}
\def\dbP{{\mathbb P}}
\def\dbQ{{\mathbb Q}}
\def\dbR{{\mathbb R}}
\def\dbS{{\mathbb S}}
\def\dbT{{\mathbb T}}
\def\dbU{{\mathbb U}}
\def\dbV{{\mathbb V}}
\def\dbW{{\mathbb W}}
\def\dbX{{\mathbb X}}
\def\dbY{{\mathbb Y}}
\def\dbZ{{\mathbb Z}}

\def\la{{\langle}}
\def\ra{{\rangle}}
\def\summ{{\sum\limits}}
\def\eps{{\varepsilon}}
\def\lam{{\lambda}}

\bf\centerline{Homework \#7. Due on Thursday, October 23rd, 11:59pm on Canvas}\rm
\vskip .1cm

\bf\centerline{Plan for next week:}\rm
\skv
Below Pugh stands for `Real Mathematical Analysis' by Charles Pugh, Rudin for `Principles of Mathematical Analysis' by Walter Rudin, Tao I for `Real Analysis I' by Terrence Tao and Tao II for `Real Analysis II' by Terrence Tao. Pugh, Tao I and Tao II are freely available via UVA subscription (use Springer Link).
\skv

{\bf Plan for next week:} Uniform Convergence (4.1 in Pugh, 7.1-7.3 in Rudin and 3.1-3.3 in Tao II) and Banach's Contraction Mapping Theorem (4.5 in Pugh, 9.3 in Rudin and 6.6 in Tao II).
\skv

\skv
\bf\centerline{Problems: }\rm
\skv
{\bf Note on hints: } Most hints are given at the end of the assignment, each on a separate page.
Problems (or parts of problems) for which hint is available are marked with *. 
\skv
\skv
\skv
{\bf 1.} (practice) Let $X$ be a metric space and $Y$ a subset of $X$. Prove that if $X$ is complete and $Y$ is closed in $X$, then $Y$ is complete. {\bf Note:} In class we proved an analogous result with `complete' replaced by `compact' (Theorem~8.7). We also proved a related result involving completeness: if $Y$ is complete, then $Y$ is closed in $X$ (this is part of Theorem~10.1).
\skv
{\bf 2.} The goal of this problem is to fill in the details of the construction of the completion of a metric space discussed in Lecture~13. 

We start by recalling the notations introduced in class. Let $(X,d)$ be a metric space. Let $\Omega=\Omega(X)$ be the set of all Cauchy sequences
$(x_n)_{n\in\dbN} $ with $x_n\in X$ for each $n$. Define the relation $\sim$
on $\Omega$ by setting $$(x_n)\sim (y_n) \iff \lim_{n\to\infty} d(x_n,y_n)=0.$$ 
\begin{itemize}
\item[(a)] Prove that $\sim$ is an equivalence relation.
\end{itemize}
Now let $\widehat X=\Omega/\sim$, the set of equivalence classes with respect to
$\sim$. The equivalence class of a sequence $(x_n)$ will be denoted by $[x_n]$.
Given an element $x\in X$, we will denote
by $[x]\in \widehat X$ the equivalence class of the constant sequence all of whose 
elements are equal to $x$.   
\skv
Now define the function $D:\widehat X\times \widehat X\to \dbR_{\geq 0}$ by setting
$$D([x_n],[y_n])=\lim_{n\to\infty} d(x_n,y_n) \eqno (***)$$
\begin{itemize}
\item[(b)*] Prove that the limit on the right-hand side of (***) always exists and that the
function $D$ is well-defined (that is, if $[x_n]=[x'_n]$ and $[y_n]=[y'_n]$, then
$\lim_{n\to\infty} d(x_n,y_n)=\lim_{n\to\infty} d(x'_n,y'_n)$). 

\item[(c)] Prove that $(\widehat X, D)$ is a metric space.
\end{itemize}
\skv



{\bf 3.} (Distance from a point to a subset).
Let $(X,d)$ be a metric space. Given a point $x\in X$
and a subset $Z$ of $X$, we define $d(x,Z)$ (the distance from $x$ to $Z$)
by $$d(x,Z)=\inf\{d(x,z): z\in Z\}.$$
Note that one indeed has to take infimum, not minimum. There may be no
$z\in Z$ which is closest to $x$.
\begin{itemize}
\item[(a)] Prove that $d(x,Z)=0$ $\iff$ $x\in\overline Z$.
\item[(b*)] Prove that $d(x,Z)\geq d(y,Z)-d(x,y)$ for all $x,y\in X$ and 
$Z\subseteq X$
\item[(c)] Now fix $Z\subseteq X$, and define $d_Z:X\to \dbR$ by
$d_Z(x)=d(x,Z)$. Use (b) to prove that $d_Z$ is continuous.
\end{itemize}
\skv
{\bf 4*.} Again let $(X,d)$ be a metric space. Let $a\in X$ and
$K$ a compact subset of $X$. Prove that there exists $k\in K$ such that $d(a,k)=d(a,K)$
(this is equivalent to saying that the set $\{d(a,z): z \in K\}$
has the minimal element). 

{\bf 5.} 
$\empty$
\begin{itemize}
\item[(a)] Theorem~40 in Pugh states the following: Let $X,Y$ are metric metric spaces, assume that $X$ is a compact, and assume that
$f:X\to Y$ is continuous and bijective. Then $f^{-1}:Y\to X$ is also continuous. Give a short proof of this theorem by combining Corollary~7.4, Theorem~8.2, Theorem~8.7 and Theorem~11.1 from class (note that in Theorems~8.2 and 8.7 we can now replace
sequential compactness and covering compactness by just `compactness' as we already proved the equivalence of the two versions of compactness). The respective references in Pugh are Theorem~11, equivalence of (i) and (iii), Theorem~26, Theorem~32 and Theorem~36.

\item[(b)] Now give an example of metric spaces $X$ and $Y$ and 
a function $f:X\to Y$ such that $f$ is continuous and bijective, but $f^{-1}:Y\to X$
is not continuous (so that the assumption that $X$ is compact in (a) is essential).
\end{itemize}
\skv
Before solving Probelms~6 and 7 below read the subsection on uniform continuity at the end of Lecture~11.
\skv
{\bf 6.} (practice) Read (and understand) the proof of Theorem~42 in Pugh which asserts that if $f:X\to Y$ is continuous and $X$ is compact, then $f$ is uniformly continuous.

\skv
{\bf 7*.} Let $f:\dbR\to\dbR$ be a differentiable function.
\begin{itemize}
\item[(a)] Assume that $f'$ is bounded, that is,
there exists $M\in\dbR$ such that $|f'(x)|\leq M$ for all $x\in\dbR$.
Prove that $f$ is uniformly continuous.
\item[(b)] Now assume that $f'(x)\to \infty$ as $x\to \infty$.
Prove that $f$ is not uniformly continuous.
\end{itemize}
%Recall the concept of the {\bf completion} of a metric space introduced in Lecture~13.
%\skv
{\bf 8.} Consider functions $f_n: \dbR_{\geq 0}\to \dbR$
given by $f_n(x)=\frac{1}{nx+1}$. Let $0\leq a\leq b$
be real numbers. Prove that $\{f_n\}$ converges uniformly
on $[a,b]\iff$ $a>0$ or $a=b=0$. Include all the details!
\skv

\newpage
{\bf Hint for 2(b):} For the existence of the limit prove that the sequence $(d(x_n,y_n))$ is Cauchy
using the inequality $d(x,w)\leq d(x,y)+d(y,z)+d(z,w)$.

\newpage
{\bf Hint for 3(b):}  If $S$ is a subset of $\dbR$ bounded below and $\alpha\in\dbR$, how to show that
$\inf(S)\geq \alpha$?

\newpage
{\bf Hint for 4:} Apply Corollary 11.2 to a suitable function.

\newpage
{\bf Hint for 7:} Use the Mean Value Theorem.

\end{document}
