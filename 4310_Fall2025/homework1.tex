\documentclass[11pt]{amsart}

\usepackage{amsmath}
\usepackage{amssymb}
\usepackage{amsthm}
%\usepackage{psfig}

\begin{document}
\baselineskip=16pt
\textheight=8.5in
\parindent=0pt 
\def\sk {\hskip .5cm}
\def\skv {\vskip .08cm}
\def\cos {\mbox{cos}}
\def\sin {\mbox{sin}}
\def\tan {\mbox{tan}}
\def\intl{\int\limits}
\def\lm{\lim\limits}
\newcommand{\frc}{\displaystyle\frac}
\def\xbf{{\mathbf x}}
\def\fbf{{\mathbf f}}
\def\gbf{{\mathbf g}}

\def\dbA{{\mathbb A}}
\def\dbB{{\mathbb B}}
\def\dbC{{\mathbb C}}
\def\dbD{{\mathbb D}}
\def\dbE{{\mathbb E}}
\def\dbF{{\mathbb F}}
\def\dbG{{\mathbb G}}
\def\dbH{{\mathbb H}}
\def\dbI{{\mathbb I}}
\def\dbJ{{\mathbb J}}
\def\dbK{{\mathbb K}}
\def\dbL{{\mathbb L}}
\def\dbM{{\mathbb M}}
\def\dbN{{\mathbb N}}
\def\dbO{{\mathbb O}}
\def\dbP{{\mathbb P}}
\def\dbQ{{\mathbb Q}}
\def\dbR{{\mathbb R}}
\def\dbS{{\mathbb S}}
\def\dbT{{\mathbb T}}
\def\dbU{{\mathbb U}}
\def\dbV{{\mathbb V}}
\def\dbW{{\mathbb W}}
\def\dbX{{\mathbb X}}
\def\dbY{{\mathbb Y}}
\def\dbZ{{\mathbb Z}}

\def\la{{\langle}}
\def\ra{{\rangle}}
\def\summ{{\sum\limits}}
\def\eps{{\varepsilon}}
\def\lam{{\lambda}}

\bf\centerline{Homework \#1. Due on Thursday, September 4th, 11:59pm on Canvas}\rm
\vskip .1cm

\bf\centerline{Plan for next week:}\rm
\skv
Below Pugh stands for `Real Mathematical Analysis' by Charles Pugh, Rudin for `Principles of Mathematical Analysis' by Walter Rudin, Tao I for `Real Analysis I' by Terrence Tao and Tao II for `Real Analysis II' by Terrence Tao. Pugh, Tao I and Tao II are freely available via UVA subscription (use Springer Link)
\skv

1. On Mon, Sep 1, we will briefly go over the axiomatic definition of real numbers, followed by two ways to explicitly construct real
numbers -- Dedekind cuts and equivalence classes of Cauchy sequences of rational numbers. See 1.2 in Pugh and Chapter 1 in Rudin for Dedekind cuts and 5.1-5.3 in Tao I for Cauchy sequences.
\skv
\skv
2. On Wed, Sep 3, we will discuss some basic results about cardinalities of sets -- see 1.4 in Pugh, 2.1 in Rudin and 3.6 in Tao I. If time allows,
we will also start talking about metric spaces  -- 2.1 in Pugh, 2.2 in Rudin and 1.1 in Tao II. 
\skv

\skv
\bf\centerline{Problems: }\rm
\skv
{\bf 1.} Give a detailed and rigorous proof of the fact that 
$$\lim\limits_{n\to\infty}\frac{2n+3}{3n+4}=\frac{2}{3}$$
directly from the definition of limit of a sequence.
\skv

{\bf 2.} Define the function $f:\mathbb R\to\mathbb R$ by $f(x)=x^2$. Prove that $f$ is continuous at $x=45$ directly from the $\eps$-$\delta$
definition of continuity.
\skv

{\bf 3.} Let $\{a_n\}_{n=1}^{\infty}$ be a sequence of real numbers. Prove that $\{a_n\}$ converges if and if only if
both subsequences $\{a_{2n}\}$ and $\{a_{2n+1}\}$ converge and $\lm_{n\to\infty}a_{2n}=\lm_{n\to\infty}a_{2n+1}$.
\skv

{\bf 4.} Let $A$ and $B$ be non-empty subsets of $\dbR$ such that $a\leq b$ for all $a\in A$ and $b\in B$.
Prove that $A$ is bounded above, $B$ is bounded below and $\sup(A)\leq \inf(B)$. {\bf Hint:} First show that $\sup(A)\leq b$ for every $b\in B$.

\skv

{\bf 5.} This problem defines the upper limit (or limit superior) of a sequence of real numbers, denoted by $\limsup$. 
Unlike the usual limit,
$\limsup$ is defined for every sequence, but can equal $\infty$ or $-\infty$. Let $\{a_n\}_{n=1}^{\infty}$
be a sequence of real numbers. If $\{a_n\}$ is NOT bounded from above, we set $\limsup a_n =\infty$. For the rest of the problem we assume 
that $\{a_n\}$ is bounded from above. 

\hskip .6cm For each $n\in\dbN$ let $A_n=\{a_n, a_{n+1}, a_{n+2},\ldots\}$ be the set of all elements of this sequence starting with $a_n$ (thus $A_1\supseteq A_2\supseteq A_3\supseteq \ldots$). Define $s_n=\sup(A_n)$. For some explicit examples see 6.4 in Tao I.
\begin{itemize}
\item[(a)] Prove that the sequence $\{s_n\}$ is non-increasing, that is, $s_1\geq s_2\geq s_3\ldots$
\item[(b)] Deduce from (a) that the sequence $\{s_n\}$ either converges (to a real number) or tends to $-\infty$. We define $\limsup a_n$ (the upper limit of the original sequence) by $\limsup a_n=\lim s_n$. 
\item[(c)] Prove that $\limsup a_n=-\infty$ if and only if $a_n$ tends to $-\infty$.
\item[(d)] Assume that $a_n$ does NOT tend to $-\infty$, so that $\limsup a_n$ is a real number. Prove that $\limsup a_n$ is the largest
possible limit of a subsequence of $\{a_n\}$. {\bf Hint:} Let $L=\limsup a_n$. First show that there is a subsequence of $\{a_n\}$
which converges to $L$; then show that if $\{a_{n_k}\}$ is any convergent subsequence, then $\lim_{k\to\infty} a_{n_k}\leq L$.
\end{itemize}

\skv
{\bf 6.} Let $X$ be a set and $\{f_n:X\to\mathbb R\}_{n=1}^{\infty}$ a sequence of functions from $X$ to $\mathbb R$. Let $f:X\to\mathbb R$
be a function.
One says that
\begin{itemize}
\item[(a)] $f_n$ converges to $f$ {\it uniformly} if for every $\eps>0$ there exists $N\in\mathbb N$ (depending only on $\eps$) such that $|f_n(x)-f(x)|<\eps$
for all $n>N$ and all $x\in X$.
\item[(b)] $f_n$ converges to $f$ {\it pointwise} if for every $\eps>0$ and every $x\in X$ there exists $N\in\mathbb N$ (depending on $\eps$ and $x$) such that $|f_n(x)-f(x)|<\eps$ for all $n>N$.
\end{itemize}
Write down the {\bf negations} of these properties, that is, define what the following mean:
\begin{itemize}
\item[(c)] $f_n$ does not uniformly converge to $f$; 
\item[(d)] $f_n$ does not converge pointwise to $f$.
\end{itemize}
\skv
In both (c) and (d) the formulations should be similar to those in (a) and (b) above and should avoid expressions like ``it is not true that (...)''
where (...) is some complicated statement.
\skv

{\bf 7.} Deduce the Intermediate Value Theorem and Extreme Value Theorems directly from the following four results
(which will be proved later in the course):
\begin{itemize}
\item[(1)] Let $I=[a,b]$ be a closed bounded interval in $\dbR$, and consider $I$ as a metric space with the standard metric
($d(x,y)=|x-y|$). Then $I$ is compact and connected.
\item[(2)] Let $S\subseteq \dbR$ be a subset which is both compact and connected (again with respect to the standard metric).
Then $S=\emptyset$ or $S=[a,b]$ for some $a,b\in\dbR$ with $a\leq b$.
\item[(3)] Let $X$ and $Y$ be metric spaces and $f:X\to Y$ be a continuous function. If $X$ is connected, then $f(X)$ is connected (as usual $f(X)=\{f(x): x\in X\}$ is the image (=range) of $f$).
\item[(4)] Let $X$ and $Y$ be metric spaces and $f:X\to Y$ be a continuous function. If $X$ is compact, then $f(X)$ is compact.
\end{itemize}
{\bf Note:} The definitions of compactness and connectedness for metric spaces are given in Chapter~2 of both Pugh and Rudin, but they are not needed for this problem (all you need to know is that these are certain properties of metric spaces).

\skv
{\bf 8.} Let $f:[a,b]\to\mathbb R$ be a function, and suppose that there are only finitely many $x\in [a,b]$ such that $f(x)\neq 0$. Prove that
$f$ is (Riemann)-integrable on $[a,b]$ and that $\int\limits_{a}^b f(x)dx=0$. You are allowed to use the definition of the Riemann integral and its linearity properties:
\begin{itemize}
\item[(i)] If $f$ and $g$ are integrable on $[a,b]$, then $f+g$ is integrable on $[a,b]$ and 
$\int\limits_{a}^b (f(x)+g(x))dx=\int\limits_{a}^b f(x)dx+\int\limits_{a}^b g(x)dx$

\item[(ii)] If $f$ is integrable on $[a,b]$ and $\lam\in\mathbb R$, then $\lam f$ is integrable on $[a,b]$ and
$\int\limits_{a}^b \lam f(x) dx=\lam \int\limits_{a}^b f(x)dx$.
\end{itemize}
\end{document}



