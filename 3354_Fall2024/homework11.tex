\documentclass[11pt]{amsart}

\usepackage{amsmath}
\usepackage{amssymb}
\usepackage{amsthm}
\usepackage[hyphens]{url}
\usepackage{hyperref}
%\usepackage{psfig}

\begin{document}
\baselineskip=16pt
\textheight=8.5in
%\parindent=0pt 
\def\sk {\hskip .5cm}
\def\skv {\vskip .08cm}
\def\cos {\mbox{cos}}
\def\sin {\mbox{sin}}
\def\tan {\mbox{tan}}
\def\intl{\int\limits}
\def\lm{\lim\limits}
\newcommand{\frc}{\displaystyle\frac}
\def\xbf{{\mathbf x}}
\def\fbf{{\mathbf f}}
\def\gbf{{\mathbf g}}

\def\dbA{{\mathbb A}}
\def\dbB{{\mathbb B}}
\def\dbC{{\mathbb C}}
\def\dbD{{\mathbb D}}
\def\dbE{{\mathbb E}}
\def\dbF{{\mathbb F}}
\def\dbG{{\mathbb G}}
\def\dbH{{\mathbb H}}
\def\dbI{{\mathbb I}}
\def\dbJ{{\mathbb J}}
\def\dbK{{\mathbb K}}
\def\dbL{{\mathbb L}}
\def\dbM{{\mathbb M}}
\def\dbN{{\mathbb N}}
\def\dbO{{\mathbb O}}
\def\dbP{{\mathbb P}}
\def\dbQ{{\mathbb Q}}
\def\dbR{{\mathbb R}}
\def\dbS{{\mathbb S}}
\def\dbT{{\mathbb T}}
\def\dbU{{\mathbb U}}
\def\dbV{{\mathbb V}}
\def\dbW{{\mathbb W}}
\def\dbX{{\mathbb X}}
\def\dbY{{\mathbb Y}}
\def\dbZ{{\mathbb Z}}

\def\lam{{\lambda}}
\def\la{{\langle}}
\def\ra{{\rangle}}
\def\summ{{\sum\limits}}

\def\Aut{{\rm Aut\,}}
\def\Ker{{\rm Ker\,}}
\def\phi{{\varphi}}


\bf\centerline{Homework \#11. Due on Thursday, December 5th, 11:59pm on Canvas}\rm
\vskip .1cm

\bf\centerline{Reading: }\rm
\skv
1. For this assignment: Online lectures 22, 23 and parts of 25, 26. From Hungerford: 7.5, 8.1 and 8.2.

2. For the remaining classes: Lecture 23 for the class on Monday, Nov 25, Lecture 25 for Monday, Dec 2 and Lecture 25 for Wednesday, Dec 4.
From Hungerford: 8.4, 6.1 and 6.2 (in this order).
\skv
Online lectures are currently posted on the Spring 2016 webpage
\skv

\skv
\centerline{\url{https://m-ershov.github.io/3354_Spring2016/}}
\skv


\skv
\bf\centerline{Problems: }\rm
\skv
\skv
\bf{Problem 1: }\rm Let $G=D_8$, the octic group, and $H=\la r^2\ra=\{e,r^2\}$. By HW\#10, Problem~9, $H$ is equal to $Z(G)$, the center of $G$,
so it is normal and hence we can consider the quotient group $G/H$. 

\begin{itemize}
\item[(a)] Compute the multiplication table for $G/H$ and show details of your computation
(for a sample calculation see online Lecture~22). Make sure that in the multiplication table you do not use multiple names for the same element of $G/H$.
\item[(b)] Now prove that $G/H\cong \dbZ_2\times \dbZ_2$ without using your answer in (a). You may use the classification of groups of order 4.
\end{itemize}

\skv
\bf{Problem 2: }\rm Let $G=(\dbZ_{12},+)$ and $H=\la [4]\ra$, the cyclic subgroup generated by $[4]$.
\begin{itemize}
\item[(a)] Describe the elements of the quotient group $G/H$ and compute the ``multiplication'' table for $G/H$
(the word ``multiplication'' is in quotes because the group operation in $G$ is addition).
\item[(b)] Deduce from your computation in (a) that $G/H$ is isomorphic to $\dbZ_4$.
\item[(c)] Now give a different proof of the isomorphism $G/H\cong \dbZ_4$ using FTH. 
%In other words, construct a surjective homomorphism $\phi: G\to \dbZ_4$ such that $\Ker\phi=H$.
\end{itemize}
{\bf Problem 3:} Let $A$ and $B$ be a groups and $G= A\times B$ their direct product. Let
$\widetilde A=\{(a, e_B) : a\in A\}$ be the subset of $G$ consisting of all elements whose second
component is identity. Use FTH to prove that $\widetilde A$ is a normal subgroup of $G$
and the quotient group $G/\widetilde A$ is isomorphic to $B$.
\skv
{\bf Problem 4:} This problem deals with the group $\dbR/\dbZ$, the quotient of
the group $(\dbR,+)$ of reals with addition by the subgroup of integers.
Let $x\in\dbR$. Prove that $x+\dbZ$ (considered as an element of $\dbR/\dbZ$) has finite order if and only if $x\in\dbQ$.
\skv
\bf{Problem 5: }\rm Let $R$ be a commutative ring with $1$, and let $I$ be an ideal of $R$.
Prove that if $I$ contains an element $r\in R$ which is invertible (in $R$), then $I=R$.
\skv
\bf{Problem 6: }\rm Let $R=\dbZ[x]$ (polynomials with coefficients in $\dbZ$),
and let 
$$I=\{a_0+a_1x+\ldots+ a_n x^n:\mbox{ each }a_i\in\dbZ \mbox{ and }a_0\mbox{ is even. }\}$$
\begin{itemize}
%\item[(a)] Prove that $I$ is an ideal of $R$ directly from definition.
\item[(a)] Prove that $I$ is an ideal of $R$
\item[(b)] Now prove that $I$ is the minimal ideal  
of $R$ containing $2$ and $x$, that is, prove that any ideal containing $2$ and $x$ must contain $I$
\item[(c)] Prove that $I$ is a non-principal ideal, that is, 
$I\neq fR$ for any $f\in R$.
\end{itemize}
\skv
{\bf Problem 7:} Find all RING homomorphisms $\phi:\dbZ_{10}\to\dbZ_{10}$ (see the beginning of Lecture~26 for
the definition of a ring homomorphism). {\bf Hint:} If
$\phi:\dbZ_{10}\to\dbZ_{10}$ is a ring homomorphism, then $\phi$ is also a group homomorphism where we consider
$\dbZ_{10}$ as a group with addition. All group homomorphisms from $\dbZ_{10}$ to $\dbZ_{10}$ have been described
in HW\#9. Problem~4, and you only need to determine which of those homomorphisms are ring homomorphisms.
\skv
{\bf Problem 8:} This problem deals with polynomials with coefficients in $\dbZ_3$. For $i=0,1,2$
let $f(x)=x^2+[i]_3\in\dbZ_3[x]$, and let $R_i=\dbZ_{3}[x]/(f_i)$, the quotient of $\dbZ_{3}[x]$ by the principal ideal generated by $f_i$.

\begin{itemize}
\item[(a)] Compute the multiplication table for the ring $R_i$. You do not have to write brackets in your answer.
\item[(b)] Prove that the rings $R_1,R_2$ and $R_3$ are pairwise non-isomorphic. {\bf Hint:} For any two of these rings you can find a basic ring-theoretic property (which is preserved under homomorphisms) that holds for one of these rings and does not hold for the other.
\end{itemize}
\end{document}
