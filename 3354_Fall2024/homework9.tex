\documentclass[11pt]{amsart}

\usepackage{amsmath}
\usepackage{amssymb}
\usepackage{amsthm}
\usepackage[hyphens]{url}
\usepackage{hyperref}
%\usepackage{psfig}

\begin{document}
\baselineskip=16pt
\textheight=8.5in
%\parindent=0pt 
\def\sk {\hskip .5cm}
\def\skv {\vskip .08cm}
\def\cos {\mbox{cos}}
\def\sin {\mbox{sin}}
\def\tan {\mbox{tan}}
\def\intl{\int\limits}
\def\lm{\lim\limits}
\newcommand{\frc}{\displaystyle\frac}
\def\xbf{{\mathbf x}}
\def\fbf{{\mathbf f}}
\def\gbf{{\mathbf g}}

\def\dbA{{\mathbb A}}
\def\dbB{{\mathbb B}}
\def\dbC{{\mathbb C}}
\def\dbD{{\mathbb D}}
\def\dbE{{\mathbb E}}
\def\dbF{{\mathbb F}}
\def\dbG{{\mathbb G}}
\def\dbH{{\mathbb H}}
\def\dbI{{\mathbb I}}
\def\dbJ{{\mathbb J}}
\def\dbK{{\mathbb K}}
\def\dbL{{\mathbb L}}
\def\dbM{{\mathbb M}}
\def\dbN{{\mathbb N}}
\def\dbO{{\mathbb O}}
\def\dbP{{\mathbb P}}
\def\dbQ{{\mathbb Q}}
\def\dbR{{\mathbb R}}
\def\dbS{{\mathbb S}}
\def\dbT{{\mathbb T}}
\def\dbU{{\mathbb U}}
\def\dbV{{\mathbb V}}
\def\dbW{{\mathbb W}}
\def\dbX{{\mathbb X}}
\def\dbY{{\mathbb Y}}
\def\dbZ{{\mathbb Z}}

\def\lam{{\lambda}}
\def\la{{\langle}}
\def\ra{{\rangle}}
\def\summ{{\sum\limits}}

\def\Aut{{\rm Aut\,}}
\def\Ker{{\rm Ker\,}}
\def\phi{{\varphi}}


\bf\centerline{Homework \#9. Due on Thursday, November 7th, 11:59pm on Canvas}\rm
\vskip .1cm

\bf\centerline{Reading: }\rm
\skv
1. For this assignment: Online lectures 16, 17 (before Cayley's theorem) and the beginning of 18 (just the statement of Lagrange Theorem and its consequences). From Hungerford: 7.4 and beginning 7.5.

2. For next week's classes: Monday, Nov 4: online lecture 17; Wednesday, Nov 6: online lecture 19 and the beginning of lecture 18 (just 18.1).
From Hungerford: beginning of 7.5 for Monday's class and 8.1 for Wednesday's class.
\skv
Online lectures are currently posted on the Spring 2016 webpage
\skv

\skv
\centerline{\url{https://m-ershov.github.io/3354_Spring2016/}}
\skv


\skv
\bf\centerline{Problems: }\rm
\skv
\bf{Problem 1: }\rm Recall that by Lemma~17.2 in class (not in the online notes), for any homomorphism of groups $\phi:G\to H$
and any $g\in G$ we have
\begin{itemize}
\item[(i)] $o(\phi(g))\leq o(g)$;
\item[(ii)] If $o(g)$ is finite, then $o(\phi(g))$ divides $o(g)$.
\end{itemize}
Use (i) to give a short proof of Proposition~15.3 from online notes which asserts that isomorphisms preserve orders of elements,
that is, for any isomorphism of groups $\phi:G\to H$ and any $g\in G$ we have $o(\phi(g))=o(g)$.
\skv

\bf{Problem 2: }\rm 
\begin{itemize}
\item[(a)] Let $G$ be an abelian group and let $m$ be an integer. Prove that the map
$\phi:G\to G$ given by $\phi(x)=x^m$ is a homomorphism.
\item[(b)] Let $G=(\dbZ_{12},+)$.
Define the map $\phi:G \to G$ by $\phi([x])=3[x]=[3x]$.
Prove that $\phi$ is a homomorphism and compute its image and kernel. 
\end{itemize}
\skv
\bf{Problem 3: }\rm Let $G$ and $H$ be groups and $\phi:G\to H$ a homomorphism.
For each of the following statements, determine whether it is true (in general)
or false (in at least one case). If the statement is true, prove it; if it is false,
give a specific counterexample.
\begin{itemize}
\item[(a)] If $H$ is abelian, then $G$ is abelian
\item[(b)] If $G$ is abelian, then $H$ is abelian
\item[(c)] If $G$ is abelian, then $\phi(G)$ is abelian
\item[(d)] If $G$ is abelian, then $\Ker(\phi)$ is abelian
\end{itemize}
\skv
\bf{Problem 4: }\rm Let $G=\la x\ra$ be a cyclic group generated by some element $x$ and let $H$ be an arbitary group.
\begin{itemize}
\item[(a)] Prove that for any $h\in H$ there exists AT MOST one homomorphism $\phi:G\to H$ with the property that $\phi(x)=h$,
and if such $\phi$ exists, it is given by the formula $$\phi(x^k)=h^k\mbox{ for all }k\in\dbZ. \eqno (***)$$
In other words, a homomorphism from a cyclic group is uniquely determined by where it sends a generator (but there is no guarantee that every choice of the image of a generator can be extended to an homomorphism)


\item[(b)] Now prove that the map $\phi$ given by the formula (***) from (a) is a homomorphism if and only if it is well defined.
Note that $\phi$ may not be well defined since for a given $g\in G$ there may be more than one value of $k$ such that
$g=x^k$.
\item[(c)] Assume that $G$ is infinite. Prove that the map $\phi$ from (***) is always well defined.
\item[(d)] Now assume that $G$ is finite and let $n=|G|=o(x)$. Fix $h\in H$ and let $\phi$ be the corresponding map from (***).
Prove that the following are equivalent:
\begin{itemize}
\item[(i)] $\phi$ is well defined
\item[(ii)] $h^n=e$
\item[(iii)] $o(h)$ divides $n$.
\end{itemize}
\item[(e)] Now assume that $G=H=\dbZ_n$ for some $n\in\dbN$ (as usual the operation is addition). Use (d) to prove that
for any $m\in\dbZ$ there exists a unique homomorphism $\phi_m:\dbZ_n\to \dbZ_n$ such that $\phi_m([1])=[m]$ and write down the explicit formula
for it: $$\phi_m([k])= \ldots$$ 
Then prove that $\phi_m$ is an isomorphism $\iff$ $gcd(m,n)=1$.
 \end{itemize}
\skv
\bf{Problem 5: }\rm Let $G$ and $H$ be finite groups such
that $|G|$ and $|H|$ are coprime. Prove that
any homomorphism $\phi:G\to H$ must be trivial, that
is, $\phi(x)=e_H$ for all $x\in G$ where $e_H$ is the identity
element of $H$. \bf{Hint: }\rm
Use the Range-Kernel theorem (see online Lecture 16; in class we called it the Image-Kernel Theorem) and Lagrange theorem (see Lecture~18)
applied to a suitable subgroup. 
\skv
 
 \skv
{\bf Problem 6:} 
\begin{itemize}
\item[(a)] Let $f=\begin{pmatrix}1 & 2& 3& 4&5 &  6& 7\\
5 & 1& 3& 7&2 &  6& 4 \end{pmatrix}$ in two-line notation. Write $f$ as in disjoint cycle form.

\item[(b)] Write the following element of $S_9$ as a product of disjoint cycles: 
$$(1,2,4,6,7)(3,4,5,1,8)(9,2,3,5)$$
\end{itemize}
\skv

{\bf Problem 7:} List all elements of $S_3$ in disjoint cycle form and compute the multiplication table of $S_3$.
\skv

\skv
\bf{Problem 8: }\rm As proved in online Lecture~17, if $f\in S_n$ is written as a product of
disjoint cycles $f_1 f_2\ldots f_r$ where
$f_1$ has length $k_1$, $\ldots$, $f_r$ has length $k_r$,
then the order of $f$ is the least common multiple
of $k_1, k_2,\ldots, k_r$. Use this fact to find the smallest $n\in\dbN$ for which $S_n$
has an element of order 15 and prove your answer (include all the details).
\skv

\bf{Bonus problem: }\rm 
\begin{itemize}
\item[(a)] Let $G$ be a group and let $\Aut(G)$ be the set of
all automorphisms of $G$ (= isomorphisms from $G$ to $G$). Prove
that elements of $\Aut(G)$ form a group with respect to composition.
This group is called the {\it automorphism group of $G$}.
{\bf Hint:}  This follows from Problem~6 of HW\#8.
What is the identity element of $\Aut(G)$?

\item[(b)] Let $G=\dbZ_n$ (with addition). Use the result of Problem~4(e)
to prove that $\Aut(G)$ is isomorphic to $\dbZ_n^{\times}$ (with multiplication).
\bf{Hint: }\rm This problem is much easier than it seems.
Elements of $\Aut(G)$ are explicitly described in Problem 4(e).
Use it to find a natural bijective mapping between $\Aut(G)$
and $\dbZ_n^{\times}$; then show that your mapping is in fact 
an isomorphism.
\end{itemize}
\end{document}


