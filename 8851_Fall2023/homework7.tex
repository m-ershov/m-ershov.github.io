\documentclass[12pt]{amsart}

\usepackage{amsmath}
\usepackage{amssymb}
\usepackage{amsthm}
%\usepackage{psfig}

\newtheorem* {Definition}    {Definition}
\newtheorem {Theorem}    {Theorem}
\newtheorem {Lemma} [Theorem]   {Lemma}
\newtheorem {Corollary} [Theorem]   {Corollary}


\begin{document}
 \pagenumbering{gobble}
\baselineskip=16pt
\textheight=8.5in
\textwidth=6in
%\parindent=0pt 
\def\sk {\hskip .5cm}
\def\skv {\vskip .08cm}
\def\cos {\mbox{cos}}
\def\sin {\mbox{sin}}
\def\tan {\mbox{tan}}
\def\intl{\int\limits}
\def\lm{\lim\limits}
\newcommand{\frc}{\displaystyle\frac}
\def\xbf{{\mathbf x}}
\def\fbf{{\mathbf f}}
\def\gbf{{\mathbf g}}

\def\dbA{{\mathbb A}}
\def\dbB{{\mathbb B}}
\def\dbC{{\mathbb C}}
\def\dbD{{\mathbb D}}
\def\dbE{{\mathbb E}}
\def\dbF{{\mathbb F}}
\def\dbG{{\mathbb G}}
\def\dbH{{\mathbb H}}
\def\dbI{{\mathbb I}}
\def\dbJ{{\mathbb J}}
\def\dbK{{\mathbb K}}
\def\dbL{{\mathbb L}}
\def\dbM{{\mathbb M}}
\def\dbN{{\mathbb N}}
\def\dbO{{\mathbb O}}
\def\dbP{{\mathbb P}}
\def\dbQ{{\mathbb Q}}
\def\dbR{{\mathbb R}}
\def\dbS{{\mathbb S}}
\def\dbT{{\mathbb T}}
\def\dbU{{\mathbb U}}
\def\dbV{{\mathbb V}}
\def\dbW{{\mathbb W}}
\def\dbX{{\mathbb X}}
\def\dbY{{\mathbb Y}}
\def\dbZ{{\mathbb Z}}

\def\la{{\langle}}
\def\ra{{\rangle}}
\def\Ker{{\rm Ker}}
\def\rk{{\rm rk}}
\def\summ{{\sum\limits}}
\def\lra{\longrightarrow}
\def\str{\stackrel}

\def\Im{{\rm Im\,}}
\def\Gal{{\rm Gal\,}}
\def\phi{{\varphi}}
\def\eps{{\varepsilon}}
\newcommand{\lla}{\la\!\la}
\newcommand{\rra}{\ra\!\ra}

\bf\centerline{Math 8851. Homework \#7. To be completed by 5pm on Fri, Dec 1}\rm
\vskip .1cm
We will start by discussing how the Golod-Shafarevich criterion for pro-$p$ groups follows from the corresponding result for algebras. This reduction was discussed in class at the end of Lecture~24, but things got rushed at the end, and I had to omit some details.
\skv
First we fix some notations. Let $K$ be a field, $U=\{u_1,\ldots, u_d\}$ a finite set and $K\lla U\rra$ the algebra of power series over $K$ in non-commuting variables $u_1,\ldots, u_d$. Given $0\neq f\in K$, we define $\deg(f)$ to be the smallest degree of a monomial in $U$ which appears in the expansion of $f$ with nonzero coefficient. We also set $\deg(0)=\infty$. For each $n\in\dbZ_{\geq 0}$ let 
$$K\lla U\rra_n=\{f\in K\lla U\rra: \deg(f)\geq n\}.$$ Note that $K\lla U\rra_n$ is an ideal of $K\lla U\rra$.

Now let $R$ be a subset of $K\lla U\rra$ with $0\not\in R$. Let 
$I=((R))$ be the closed ideal of $K\lla U\rra$ generated by $R$ and $A=K\lla U\rra/I$, so we can think of elements of $U$ as generators and elements of $R$ as relators.
Let $R_n=\{r\in R: \deg(r)=n\}$, so $R=\sqcup_{n=0}^{\infty}R_n$. As discussed in class, we can assume that $R_0=\emptyset$ (otherwise $A=0$)
and each $R_n$ is finite (this takes a bit of work to justify).  

Let $r_n=|R_n|$ and $H_R(t)=\sum\limits_{n=0}^{\infty}r_n t^n\in\dbZ[[t]]$ the associated Hilbert series.
Let $\pi:K\lla U\rra\to A$ be the natural projection, $A_n=\pi(K\lla U\rra_n)$, $a_n=\dim_K (A_n/A_{n+1})$ and $H_A(t)=\sum\limits_{n=0}^{\infty}a_n t^n$.

\begin{Theorem}[Golod-Shafarevich inequality for filtered algebras] In the above notations we have the following inequality of power series:
$$\frac{H_A(t)(1-|U|t+H_R(t))}{1-t}\geq \frac{1}{1-t}\eqno (***)$$ 
\end{Theorem}
Recall that the inequality in (***) means that in each degree the coefficient of the power series on the left-hand side is $\geq$ the respective coefficient on the right-hand side.
\skv

\begin{Corollary} 
\label{GScondalg}
Suppose there exists $\tau\in (0,1)$ such that $1-|U|\tau+H_R(\tau)\leq 0$. Then the numerical series $H_A(\tau)$ diverges, so
in particular $A$ is infinite-dimensional.
\end{Corollary}

Suppose now that $G$ is a pro-$p$ group given by a pro-$p$ presentation $\la X|R_G\ra$
where $X=\{x_1,\ldots,x_d\}$ is finite (the set of relators $R_G$ may be infinite). Thus, $G$ is isomorphic to $F/N$ where $F=F_{\widehat p}(X)$ is the free pro-$p$ group
on $X$ and $N=\lla R_G\rra$ is the closed normal subgroup generated by $R_G$.

As above let $U=\{u_1,\ldots,u_d\}$ and consider the subgroup $\Gamma$ of 
$\dbF_p\lla U\rra^{\times}$ given by $$\Gamma=\overline{\la 1+u_1,\ldots, 1+u_d\ra},$$
the closed subgroup generated by $1+u_1,\ldots, 1+u_d$. As shown in Lecture~12,
$\Gamma$ is isomorphic to $F$ via the map $x_i\mapsto 1+u_i$ for $1\leq i\leq d$.
From now on we will identify $\Gamma$ with $F$ using this map. 
\skv

Next let $R_A=\{r-1: r\in R_G\}$ (viewed as a subset of $\dbF_p\lla U\rra$),
$I=((R_A))$ the closed ideal of $\dbF_p\lla U\rra$ and 
$A=\dbF_p\lla U\rra/I$. As explained in Lecture~23,
the embedding $\Gamma=F\to \dbF_p\lla U\rra^{\times}$ induces
a natural map $\phi:G\to A^{\times}$ such that $Span(\Im\phi)$ is dense in $A$, so in particular, $G$ is infinite whenever $A$ is infinite.
\skv
Now define the degree function $D$ on the free pro-$p$ group $F$ (still identified with
$\Gamma$) by
$$D(f)=\deg(f-1).$$
(Note that $D(f)>0$ for all $f$ since the power series expansion of $f$ always has constant term $1$).

Thus, if we set $H_{R_G}(t)=\sum_{i=1}^{\infty}t^{D(r)}$, then $H_{R_G}(t)=H_{R_A}(t)$
as formal power series. Hence Corollary~\ref{GScondalg} yields
the following:

\begin{Corollary} 
\label{GScondgrp}
Suppose that a pro-$p$ group $G$ has a pro-$p$ presentation $\la X|R_G\ra$ with $X$
is finite and there exists $\tau\in (0,1)$ such that $1-|X|\tau+H_{R_G}(\tau)\leq 0$. Then $G$ is infinite.
\end{Corollary}

We are now ready to formulate the first 2 problems.

\skv
\noindent
{\bf Problem 1.} Prove the following properties of the degree function $D$ on $F$:
\begin{itemize}
\item[(a)] $D(fg)\geq \min(D(f),D(g))$ for all $f,g\in F$;
\item[(b)] $D([f,g])\geq D(f)+D(g)$ for all $f,g\in F$;
\item[(c)] $D(f^p)=p\cdot D(f)$ for all $f\in F$.
\end{itemize}

\skv
\noindent
{\bf Problem 2.} Prove that $D(f)>1$ $\iff$ $f\in \Phi(F)$. {\bf Hint:} The backwards direction
follows from the explicit formula for $\Phi(F)$ and Problem 1. For the forward direction first show that any element of $F$ can be uniquely written
as $\prod_{i=1}^d x_i^{a_i} \cdot y$ where $0\leq a_i\leq p-1$ for each $i$ and
$y\in\Phi(F)$ (here $X=\{x_1,\ldots, x_d\}$ is the chosen free generating set for
$F$, as before).
\skv
Let us now recall an important application of Problem~2 briefly discussed at the end of Lecture~24.

As a consequence of Corollary~\ref{GScondalg} we obtained the following result:

\begin{Corollary} 
\label{GScondalg2}
Let $A=K \lla U\rra/((R))$ for some field $K$ and finite set $U$ and suppose that
$r_1=0$ (in addition to the original hypothesis that $r_0=0$), so all relations
in $R$ have degree $\geq 2$. If $|U|>0$ and $|R|\leq \frac{|U|^2}{4}$, then $A$ is infinite-dimensional.
\end{Corollary}

Now recall that if $\la X|R_G\ra$ is a pro-$p$ presentation of a finitely generated pro-$p$ group $G$ such that $|X|=d(G)$, then $R_G$ lies inside  $\Phi(F_{\widehat p}(X))$, the Frattini subgroup of $F_{\widehat p}(X)$ and so $D(r)\geq 2$ for all $r\in R_G$ by Problem~2. Hence the algebra $A$ corresponding to $G$ satisfies the hypotheses of Corollary~\ref{GScondalg2}, and we obtain
the following group-theoretic counterpart of Corollary~\ref{GScondalg2}:

\begin{Theorem} 
\label{GS2}
Let $G$ be a finitely presented pro-$p$ group and assume that
$d(G)>0$ (that is, $G$ is non-trivial) and $r(G)\leq \frac{d(G)^2}{4}$. Then $G$ is infinite.
\end{Theorem}
Recall that we used Theorem~\ref{GS2} to give a negative solution to the class field tower problem.

\skv
The next 2 problems deal with powerful pro-$p$ groups. Recall that a pro-$p$ group 
$G$ is {\it powerful} if $\overline{[G,G]}\subseteq \overline{G^p}$ for $p>2$ and
if $\overline{[G,G]}\subseteq \overline{G^4}$ for $p=2$.

More generally, a subgroup $N$ of $G$ is {\it powerfully embedded in $G$} (notation
$N$ p.e. $G$) if $\overline{[N,G]}\subseteq \overline{N^p}$ for $p>2$ and
if $\overline{[N,G]}\subseteq \overline{N^4}$ for $p=2$. Thus, $G$ is powerful
$\iff$ $G$ p.e. $G$.

Note that a subgroup $N$ of $G$ is normal $\iff$ $[N,G]\subseteq N$, so any closed powerfully
embedded subgroup is automatically normal.

\skv
\noindent
{\bf Problem 3.} Given $k,n\in\dbN$, let 
$$GL_n^k(\dbZ_p)=\{A\in GL_n^k(\dbZ_p) : A\equiv I\mod p^k\},$$ the $k^{\rm th}$
congruence subgroup of $GL_n(\dbZ_p)$.
\begin{itemize}
\item[(a)] Prove that $[GL_n^k(\dbZ_p),GL_n^m(\dbZ_p)]\subseteq SL_n^{k+m}(\dbZ_p)$
for all $k,m\in\dbN$; in particular, $[GL_n^k(\dbZ_p),GL_n^k(\dbZ_p)]\subseteq SL_n^{2k}(\dbZ_p)$.
\item[(b)] Assume that $p$ is odd. Prove that every $g\in GL_n^2(\dbZ_p)$ 
can be written as $h^p$ for some $h\in GL_n^1(\dbZ_p)$. {\bf Hint:} One way to prove this is as follows. We need to show that for every $A\in Mat_n(\dbZ_p)$ the equation
$(1+pX)^p=1+p^2A$ has a solution $X\in Mat_n(\dbZ_p)$. Expand the left-hand side
and prove that the equation has a solution mod $p^i$ for all $i\in\dbN$ by induction on $i$; then deduce that there is a solution in $Mat_n(\dbZ_p)$.
\item[(c)] Now prove that every $g\in GL_n^4(\dbZ_2)$ 
can be written as $h^4$ for some $h\in GL_n^2(\dbZ_p)$.
\item[(d)] Deduce from (a), (b) and (c) that $GL_n^k(\dbZ_p)$ is powerful
if $p>2$ (and $k$ is arbitrary) or $p=2$ and $k\geq 2$.
\end{itemize}
{\bf Remark:} The fact that the conclusions of (b) and (c) are seemingly stronger
than what is required to be powerful is not a coincidence. We will prove in class that if $G$ is a powerful pro-$p$ group, then every element of $G^p$ is a $p^{\rm th}$ power.

\skv
\noindent
{\bf Problem 4.} The following result was formulated in Lecture~25:

\begin{Lemma}
\label{lem:pe}
Let $G$ be a pro-$p$ group, $N$ a subgroup of $G$, and suppose that
$N$ p.e. $G$. Then $N^p$ p.e. $G$.
\end{Lemma}
Prove Lemma~\ref{lem:pe} for odd $p$ using the outline below. Note
that Lemma~~\ref{lem:pe} is proved in Chapter~2 of [DDMS] using the same method, but the steps are justified slightly differently there.
\begin{itemize}
\item[(a)] Prove that it is sufficient to prove Lemma~\ref{lem:pe}
for finite $p$-groups.
\item[(b)] Assume now that Lemma~\ref{lem:pe} is false for some
pair $(G,N)$ with $G$ finite and choose such pair with $|G|$
smallest possible. Show that we must have $(N^p)^p=\{1\}$.
\item[(c)] Let $G$ and $N$ be as in (b). Using the fact that inside a
finite $p$-group, any non-trivial normal subgroup contains a central element of order $p$, show that $[N^p,G]$ must be central
of order $p$, so in particular $[N^p,G]^p=[N^p,G,G]=\{1\}$ (assume
the opposite and reach a contradiction with the assumption that
$|G|$  is smallest possible).
\item[(d)] Recall the following formula from Lecture~26: for any
group $\Gamma$ and any $x,y\in \Gamma$ we have
$$(xy)^p=x^py^p [x,y]^{p\choose 2}z\eqno(***)$$ 
where $z$ lies in the normal subgroup generated by the length $3$ commutators $[[x,y],x]$ and
$[[x,y],y]$. Use this formula and the equalities $[N^p,G]^p=[N^p,G,G]=\{1\}$ from (c) to deduce that $[N^p,G]=\{1\}$,
thus reaching a contradiction. {\bf Hint:} You need to show that
$g^{-1}n^pg=n^p$ for all $n\in N$ and $g\in G$. Write 
$g^{-1}n^pg$ as $(n[n,g])^p$ and apply (***).  
\end{itemize}
\end{document}