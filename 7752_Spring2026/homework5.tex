\documentclass[11pt]{amsart}

\usepackage{amsmath}
\usepackage{amssymb}
\usepackage{amsthm}
%\usepackage{psfig}

\begin{document}
\baselineskip=15pt
\textheight=8.4in
\parindent=0pt
\def\sk {\hskip .5cm}
\def\skv {\vskip .12cm}
\def\cos {\mbox{cos}}
\def\sin {\mbox{sin}}
\def\tan {\mbox{tan}}
\def\intl{\int\limits}
\def\lm{\lim\limits}
\newcommand{\frc}{\displaystyle\frac}
\def\xbf{{\mathbf x}}
\def\fbf{{\mathbf f}}
\def\gbf{{\mathbf g}}

\def\Ker{{\rm Ker\,}}
\def\phi{\varphi}

\def\dbA{{\mathbb A}}
\def\dbB{{\mathbb B}}
\def\dbC{{\mathbb C}}
\def\dbD{{\mathbb D}}
\def\dbE{{\mathbb E}}
\def\dbF{{\mathbb F}}
\def\dbG{{\mathbb G}}
\def\dbH{{\mathbb H}}
\def\dbI{{\mathbb I}}
\def\dbJ{{\mathbb J}}
\def\dbK{{\mathbb K}}
\def\dbL{{\mathbb L}}
\def\dbM{{\mathbb M}}
\def\dbN{{\mathbb N}}
\def\dbO{{\mathbb O}}
\def\dbP{{\mathbb P}}
\def\dbQ{{\mathbb Q}}
\def\dbR{{\mathbb R}}
\def\dbS{{\mathbb S}}
\def\dbT{{\mathbb T}}
\def\dbU{{\mathbb U}}
\def\dbV{{\mathbb V}}
\def\dbW{{\mathbb W}}
\def\dbX{{\mathbb X}}
\def\dbY{{\mathbb Y}}
\def\dbZ{{\mathbb Z}}


\def\char{{\rm char}}
\def\wt{{\rm wt}}
\def\Aut{{\rm Aut}}
\def\Hom{{\rm Hom}}
\def\End{{\rm End}}
\def\GL{{\rm GL}}
\def\SL{{\rm SL}}
\def\SNF{{\rm SNF}}
\def\tr {{\rm tr}}
\def\rk{{\rm rk}}
\def\lam{{\lambda}}

\def\la{{\langle}}
\def\ra{{\rangle}}

\bf\centerline{Homework \# 5, to be submitted on Canvas by 11:59pm on Fri, Feb 27th.}\rm
\vskip .2cm
{\bf Plan for the next 2 classes (Feb 24 and 26):} Unitary and unitarizable representations. Proof of Maschke's theorem
over $\mathbb C$. Schur's Lemma and representations of abelian groups. Start talking about characters and character tables
(time permitting)

\vskip .1cm

%{\bf Note on hints:} Some hints are given at the end of the assignment, each on a separate page.
%Problems (or parts of problems) for which a hint is available at the end are marked with *.

\skv
{\bf Problem 1:} Let $F$ be a field. Use JCF to prove that any square matrix $A\in Mat_n(F)$ is similar to its transpose $A^T$. You
can assume without proof that any field can be embedded into an algebraically closed field (we will prove this later this semester).

\skv
{\bf Problem 2:} Let $G$ be a group, $F$ a field and $V$ a vector space over $F$. Prove (including all the details)
that there is a natural bijection between linear representations of $G$ of the form $(\rho,V)$ and $FG$-module
structures on $V$ which extend the given $F$-vector space structure on $V$.


\skv
{\bf Problem 3:} (Schur's Lemma) This problem collects several (related) results (parts (a),(b) and (d) below), each of which may be referred to as Schur's Lemma.
\begin{itemize}
\item[(a)] Let $R$ be a ring, $M$ and $N$ irreducible (left) $R$-modules and $f:M\to N$ a homomorphism of $R$-modules.
Prove that $f$ is either an isomorphism or the zero map.
\item[(b)] Let $R$ be a ring and $M$ an irreducible $R$-module. Prove that $\End_R(M)$, the ring of endomorphisms
of $M$ as an $R$-module, is a division ring (a division ring is defined in the same way as a field except that multiplication need not be commutative).
\item[(c)] Let $G$ be a group, $g\in Z(G)$ an element of the center of $G$, and $(\rho,V)$ a linear representation of $G$ over some field $F$.
Prove that for any $\lambda\in F$, the map $\rho(g)-\lam I:V\to V$ lies in $\End_{FG}(V)$ (where $V$ is a considered as an $FG$-module
via the correspondence from Problem~2).
\item[(d)] In the setting of (c), assume that $F$ is algebraically closed and $V$ is finite-dimensional and irreducible. Use (b) and (c)
to prove that $\rho(g)=\lam I$ for some $\lam\in F$. In other words, if we are given a finite-dimensional irreducible representation over an algebraically closed field, then any central element must act as a scalar operator. 
\end{itemize}
\skv
{\bf Problem 4:} (Lemma~11.2 from class) Let $R$ be a ring and $M$ an $R$-module. We will say that $M$ has {\it the complement property} if for every submodule $N$ of $M$ there exists a submodule $P$ such that $M=N\oplus P$. By Theorem~11.1 from class, $M$ has the complement
property if and only if $M$ is completely reducible. However, the proof of Theorem~11.1 relies on Lemma~11.2, and the main goal of this problem is to prove the latter, so we need this temporary terminology for this problem.

\begin{itemize}
\item[(a)] Suppose that $M=P\oplus Q$ for some submodules $P$ and $Q$. Prove that if $N$ is any submodule containing $P$, then
$N=P\oplus (N\cap Q)$.
\item[(b)] Deduce from (a) that if $M$ has the complement property, then so does any submodule of $M$.
\item[(c)] Now prove Lemma~11.2 from class which asserts that if $M$ has the complement property, then any
nonzero submodule $L$ of $M$ contains an irreducible submodule.
\end{itemize}
\skv
{\bf Hint for (c):} First use (b) to reduce (c) to the case $L=M=Rx$ for some nonzero $x\in M$. Use Zorn's Lemma to show 
that $M$ contains a maximal submodule $N$ not containing $x$. Now show that $M/N$ is irreducible and then use
the complement property of $M$ to deduce that it contains an irreducible submodule.

\skv
{\bf Problem 5:} Let $n\in \dbN$, $[n]=\{1,\ldots,n\}$ and $S_n$ the symmetric group on $[n]$. Let $F$ be any field
and $(\rho, V)$ the permutation representation of $S_n$ over $F$ corresponding to the defining action of $S_n$
on $[n]$. Up to equivalence, we can think of $V$ simply as $F^n$, in which case $\rho:S_n\to \GL(F^n)$
is given by $(\rho(\sigma))(e_i)=e_{\sigma(i)}$ where $\{e_1,\ldots, e_n\}$ is the standard basis of $F^n$.
\begin{itemize}
\item[(a)] Let $Z=F(e_1+\ldots+e_n)$ and $W=\{(x_1,\ldots, x_n)\in F^n :\sum x_i=0\}$. Prove that
$Z$ and $W$ are both subrepresentations of $V$.
\item[(b)] Prove that $V=W\oplus Z$ if and only if $\char(F)$ does not divide $n$.
\item[(c)]* Assume that $\char(F)$ does not divide $n!$. Prove that $W$ is an irreducible representation of $S_n$
(see a hint at the end of the assignment).
\end{itemize}


\skv
{\bf Problem 6:} Let $R$ be a commutative ring with $1$, let $M_1, M_2,N_1$ and $N_2$ be $R$-modules, and let
 $\phi:M_1\to M_2$ and $\psi: N_1\to N_2$ be homomorphisms of $R$-modules. 
\begin{itemize}
\item[(a)] Prove that there exists a unique linear map $\phi\otimes \psi: M_1\otimes N_1\to M_2\otimes N_2$ such that 
$(\phi\otimes \psi)(m\otimes n)=\phi(m)\otimes \psi(m)$ for all $m\in M_1$ and $n\in N_1$.
\item[(b)] Now assume that $R$ a field, $M_1,M_2,N_1,N_2$ are finite-dimensional (as vector spaces over $R$),
and choose bases $\alpha_i$ for $M_i$ and $\beta_i$ for $N_i$ for $i=1,2$. Note that
$\gamma_i=\alpha_i\otimes \beta_i=\{x_i\otimes y_i: x_i\in\alpha_i, y_i\in\beta_i\}$ is a basis for $M_i\otimes N_i$.
Consider the matrices $A=[\phi]_{\alpha_1}^{\alpha_2}$, $B=[\psi]_{\beta_1}^{\beta_2}$ and
$C=[\phi\otimes\psi]_{\gamma_1}^{\gamma_2}$. Prove that if $\gamma_1$ and $\gamma_2$ are ordered in a suitable way,
then $C$ is the Kronecker product of $A$ and $B$ as defined below.
\item[(c)] In the setting of (b), assume that $M_1=M_2$ and $N_1=N_2$. Prove that $$\tr(\phi\otimes \psi)=\tr(\phi)\tr(\psi).$$ 
{\bf Note:} If $V$ is a finite-dimensional vector space and $T:V\to V$ a linear map, we defines $\tr(T)=\tr([T]_{\beta})$
where $\beta$ is any basis of $T$. The right-hand side does not depend on $\beta$ since similar matrices have the same trace.
\end{itemize}
{\bf Kronecker product:} Let $R$ be a commutative ring, and let $A=(a_{ij})\in Mat_{k\times l}(R)$ and $B\in Mat_{m\times n}(R)$.  
The Kronecker product of $A$ and $B$ is a $km\times ln$-matrix over $R$ defined as a block matrix:
$$
\begin{pmatrix}a_11 B & \ldots &a_{1l}B\\
\vdots& \ddots & \vdots\\
a_k1 B & \ldots &a_{kl}B.
\end{pmatrix} 
 $$
 \newpage
 {\bf Hint for 6(c):}  Given $v\in F^n$, denote by $\wt(w)$ (the {\it weight} of $w$) the number of nonzero coordinates of $w$.
 Now show that if $w\in W$ and $\wt(w)>2$, then there exists $g\in G$ such that $w'=\sigma(g)w-w$ is nonzero
 and $\wt(w')<\wt(w)$. Deduce that every nonzero $S_n$-invariant subspace of $W$ contains a vector of weight $2$.
 It remains to show that if $w\in W$ and $\wt(w)=2$, then the smallest $G$-invariant subspace of $W$ 
 containing $w$ is the entire $W$, which can be proved by a direct computation.
\end{document}
