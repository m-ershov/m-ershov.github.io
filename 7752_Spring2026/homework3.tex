\documentclass[11pt]{amsart}

\usepackage{amsmath}
\usepackage{amssymb}
\usepackage{amsthm}
%\usepackage{psfig}

\begin{document}
\baselineskip=15pt
\textheight=8.4in
\parindent=0pt
\def\sk {\hskip .5cm}
\def\skv {\vskip .12cm}
\def\cos {\mbox{cos}}
\def\sin {\mbox{sin}}
\def\tan {\mbox{tan}}
\def\intl{\int\limits}
\def\lm{\lim\limits}
\newcommand{\frc}{\displaystyle\frac}
\def\xbf{{\mathbf x}}
\def\fbf{{\mathbf f}}
\def\gbf{{\mathbf g}}

\def\Ker{{\rm Ker\,}}
\def\phi{\varphi}

\def\dbA{{\mathbb A}}
\def\dbB{{\mathbb B}}
\def\dbC{{\mathbb C}}
\def\dbD{{\mathbb D}}
\def\dbE{{\mathbb E}}
\def\dbF{{\mathbb F}}
\def\dbG{{\mathbb G}}
\def\dbH{{\mathbb H}}
\def\dbI{{\mathbb I}}
\def\dbJ{{\mathbb J}}
\def\dbK{{\mathbb K}}
\def\dbL{{\mathbb L}}
\def\dbM{{\mathbb M}}
\def\dbN{{\mathbb N}}
\def\dbO{{\mathbb O}}
\def\dbP{{\mathbb P}}
\def\dbQ{{\mathbb Q}}
\def\dbR{{\mathbb R}}
\def\dbS{{\mathbb S}}
\def\dbT{{\mathbb T}}
\def\dbU{{\mathbb U}}
\def\dbV{{\mathbb V}}
\def\dbW{{\mathbb W}}
\def\dbX{{\mathbb X}}
\def\dbY{{\mathbb Y}}
\def\dbZ{{\mathbb Z}}

\def\Aut{{\rm Aut}}
\def\Hom{{\rm Hom}}
\def\GL{{\rm GL}}
\def\SL{{\rm SL}}
\def\rk{{\rm rk}}
\def\lam{{\lambda}}

\def\la{{\langle}}
\def\ra{{\rangle}}

\bf\centerline{Homework \# 3, to be submitted on Canvas by 11:59pm on Fri, Feb 6th.}\rm
\vskip .2cm
{\bf Plan for the next 2 classes (Feb 3 and 5):}  Canonical Forms of Linear Transformations (12.2-12.3 in DF; Lectures 6-8 from Spring 21 and Lecture 10-12 from Spring 10). 
\vskip .1cm

{\bf Note on hints:} Some hints are given at the end of the assignment, each on a separate page.
Problems (or parts of problems) for which a hint is available at the end are marked with *.

\vskip .1cm
\skv
{\bf Problem 1}. Let $R=\mathbb R[x]$, $F=R^3$ (the standard 3-dimensional $R$-module)
and $N$ the $R$-submodule of $F$
generated by $(1-x,1,0)$, $(-2,4-x,0)$ and $(1,-5,-x)$.
\begin{itemize}
\item[(a)] Find compatible bases for $F$ and $N$, that is, bases satisfying the conclusion
of the compatible bases theorem (AKA submodule structure theorem). {\bf Note:} an algorithm for computing such bases is given in Lecture~8
from Spring 2010.
\item[(b)] Describe the quotient module $F/N$ in IF and ED forms.
\end{itemize}
\skv
{\bf Problem 2}. Let $R$ be a PID. For an $R$-module $M$ denote by $d(M)$ the minimal number of generators of $M$.
\begin{itemize}
\item[(a)] Prove that if $M$ is a finitely generated $R$-module and $N$ is a submodule of $M$, then $d(N)\leq d(M)$
\item[(b)] Let $a\in R$ be a nonzero non-unit. Find (with proof) the number of submodules of $R/aR$ in terms of the prime decomposition of $a$\end{itemize}
\skv
{\bf Problem 3:} \rm 
\begin{itemize}
\item[(a)*] Let $R$ be a PID,
$M$ be a finitely generated $R$-module and
$R/a_1R\oplus \ldots \oplus R/a_mR\oplus R^s$ its invariant factor
decomposition, that is, $a_1,\ldots, a_m$ are nonzero 
non-units and $a_1\mid a_2\mid\ldots\mid a_m$. Prove that
$$d(M)=m+s.$$ 
{\bf Warning:} It is not true in general that $d(P\oplus Q)=d(P)+d(Q)$. 

\item[(b)] Again let $R$ be a PID. Let $F$ be a free $R$-module of rank $n$ with basis 
$e_1,\ldots, e_n$, let $N$ be the submodule of $F$
generated by some elements $v_1,\ldots, v_n\in F$,
and let $A\in Mat_n(F)$ be the matrix such that
$$\left(\begin{array}{c} v_1 \\ \vdots \\ v_n\end{array}\right)=
A \left(\begin{array}{c} e_1 \\ \vdots \\ e_n\end{array}\right)$$ 
Find a simple condition on the entries of $A$ which holds
if and only if $d(F/N)=n$.
\end{itemize}
\skv
{\bf Problem 4}. DF, Problem 2, page 469. By definition, the rank of an $R$-module is the smallest number of linearly independent elements.
\skv
{\bf Problem 5:} \rm Let $R$ be a PID, $M$ a finitely generated free $R$-module and $N$ a submodule of $M$.
Prove that the following are equivalent:
\begin{itemize}
\item[(1)] any basis of $N$ can be extended to a basis of $M$;
\item[(2)] some basis of $N$ can be extended to a basis of $M$;
\item[(3)] $M/N$ is free;
\item[(4)] $M/N$ is torsion-free.
\end{itemize}
\skv
{\bf Problem 6:} Given a commutative ring $R$ with $1$ and $R$-module $V$ and $W$, define $\Hom_R(V,W)$ to be the set of $R$-module
homomorphisms from $V$ to $W$. This is an $R$-module with addition defined by $(f+g)(v)=f(v)+g(v)$ for all $f,g\in \Hom_R(V,W)$
and $v\in V$ and scalar action defined by $(rf)(v)=f(rv)$ for all $f\in \Hom(V,W)$, $v\in V$ and $r\in R$. The module
$V^*=\Hom_R(V,R)$ is called the dual module of $V$. 
\begin{itemize}
\item[(1)] Let $V$ and $W$ be as above. Show that there is a natural homomorphism $\phi:V^*\otimes_R W\to \Hom_R(V,W)$
such that $(\phi(f\otimes w))(v)=f(v)w$ for all $f\in V^*$, $v\in V$ and $w\in W$.
\item[(2)] Assume that $W$ is a finitely generated free $R$-module. Prove that $\phi$ is an isomorphism. {\bf Hint:} Problem~2 from HW\#1 is relevant.
\item[(3)] Give examples showing that $\phi$ need not be surjective if either $W$ is not free or $W$ is free but not finitely generated.
\item[(4)] (bonus )Now give an example where $\phi$ is not injective.
 \end{itemize}
 \newpage

{\bf Hint for 3(a):} Let $p$ be a prime
dividing $a_1$. How is $M$ related to $M'=(R/pR)^{m+s}$
and what is $d(M')$ (and why)?
\end{document}
