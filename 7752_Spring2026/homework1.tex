\documentclass[11pt]{amsart}

\usepackage{amsmath}
\usepackage{amssymb}
\usepackage{amsthm}
%\usepackage{psfig}

\begin{document}
\baselineskip=15pt
\textheight=8.4in
\parindent=0pt
\def\sk {\hskip .5cm}
\def\skv {\vskip .12cm}
\def\cos {\mbox{cos}}
\def\sin {\mbox{sin}}
\def\tan {\mbox{tan}}
\def\intl{\int\limits}
\def\lm{\lim\limits}
\newcommand{\frc}{\displaystyle\frac}
\def\xbf{{\mathbf x}}
\def\fbf{{\mathbf f}}
\def\gbf{{\mathbf g}}

\def\Ker{{\rm Ker\,}}
\def\phi{\varphi}

\def\dbA{{\mathbb A}}
\def\dbB{{\mathbb B}}
\def\dbC{{\mathbb C}}
\def\dbD{{\mathbb D}}
\def\dbE{{\mathbb E}}
\def\dbF{{\mathbb F}}
\def\dbG{{\mathbb G}}
\def\dbH{{\mathbb H}}
\def\dbI{{\mathbb I}}
\def\dbJ{{\mathbb J}}
\def\dbK{{\mathbb K}}
\def\dbL{{\mathbb L}}
\def\dbM{{\mathbb M}}
\def\dbN{{\mathbb N}}
\def\dbO{{\mathbb O}}
\def\dbP{{\mathbb P}}
\def\dbQ{{\mathbb Q}}
\def\dbR{{\mathbb R}}
\def\dbS{{\mathbb S}}
\def\dbT{{\mathbb T}}
\def\dbU{{\mathbb U}}
\def\dbV{{\mathbb V}}
\def\dbW{{\mathbb W}}
\def\dbX{{\mathbb X}}
\def\dbY{{\mathbb Y}}
\def\dbZ{{\mathbb Z}}

\def\Aut{{\rm Aut}}

\def\la{{\langle}}
\def\ra{{\rangle}}

\bf\centerline{Homework \# 1, to be submitted on Canvas by 11:59pm on Fri, Jan 23rd.}\rm
\vskip .2cm
{\bf Plan for the next 2 classes (Jan 20 and 22):} Tensor products of modules (continued) and tensor products of algebras (10.4).  
Start nodules over PIDs (12.1).
\vskip .1cm

{\bf Note:} All rings are assumed to have $1$. DF = Dummit and Foote.
\vskip .1cm

{\bf Problem 1.} Let $R$ be a commutative ring.
An $R$-module $M$ is called {\it torsion} if for any $m\in M$
there exists nonzero $r\in R$ such that $rm=0$. An $R$-module $M$ is called {\it divisible} 
if for any nonzero $r\in R$ we have $rM=M$. In other words, $M$ is divisible if
for any $m\in M$ and nonzero $r\in R$ there exists $x\in M$ such that $rx=m$.

\begin{itemize}
\item[(a)] Suppose that $M$ is a torsion $R$-module and $N$ is a divisible $R$-module.
Prove that $M\otimes_R N =\{0\}$.

\item[(b)] Let $M=\dbQ/\dbZ$ considered as a $\dbZ$-module. Prove that
$M\otimes _{\dbZ} M=\{0\}$.
\end{itemize}

{\bf Problem 2.} Let $R$ be a commutative ring, $\{N_{\alpha}\}$ a collection
of $R$-modules and $M$ another $R$-module.

\begin{itemize}
\item[(a)] ([DF, problem 14, p.376]) Prove that $M\otimes (\oplus N_{\alpha})\cong \oplus (M\otimes N_{\alpha})$ as $R$-modules (tensor products are over $R$).

\item[(b)] ([DF, problem 15, p. 376]) Show by example that $M\otimes (\prod N_{\alpha})$ need not be isomorphic to $\prod (M\otimes N_{\alpha})$. {\bf Hint:} Use the result of one of the previous problems on p. 376.
\vskip .1cm
\end{itemize}

\bf{Problem 3. }\rm Let $R$ be a commutative domain, and let $M$ be a free 
$R$-module with basis $e_1,\ldots, e_k$. Prove that the element
$e_1\otimes e_2+e_2\otimes e_1\in M\otimes M$ is not representable as a simple tensor
$m\otimes n$ for some $m,n\in M$.
\skv
\bf{Problem 4. }\rm Problem 16 on p.376 of DF.
\skv
\bf{Problem 5. }\rm Problem 17 on pp.376-377 of DF. 
\skv
\bf{Problem 6. }\rm Problem 21 on p.377 of DF.
\skv

\bf{Problem 7. }\rm 
\begin{itemize}
\item[(a)] Let $V$ be a finite-dimensional vector space over $\dbC$
(complex numbers). Note that $V$ can also be considered as a vector space over $\dbR$,
but $\dim_{\dbR} (V)=2 \dim_{\dbC}(V)$. Prove that $V\otimes_{\dbC} V$ is not isomorphic
to $V\otimes_{\dbR} V$ as vector spaces over $\dbR$ and compute their dimensions over $\dbR$.

\item[(b)] Let $R$ be a commutative integral domain and $F$ its field of fractions. Prove that
$F\otimes_{R} F\cong F\otimes_{F} F\cong F$ as $F$-modules.

\skv

{\bf Note:} As we will explain in Lecture~3, if $T$ and $S$ are rings,
$M$ is a left $T$-module and $N$ is an $(S,T)$-bimodule, then $N\otimes_T M$ is a left $S$-module where the action of $S$
on simple tensors is given by $s(n\otimes m)=(sn)\otimes m$ for all $s\in S, m\in M, n\in N$. The $F$-module structures on $F\otimes_{R} F$ and
$F\otimes_{F} F$ are given by this construction (in both cases $S=M=N=F$ and $T=R$ in the first case and $T=F$ in the second case).
\end{itemize}
\skv
\bf{Problem 8. }\rm Problem 25 on p.377 of DF.
\end{document}
