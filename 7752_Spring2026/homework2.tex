\documentclass[11pt]{amsart}

\usepackage{amsmath}
\usepackage{amssymb}
\usepackage{amsthm}
%\usepackage{psfig}

\begin{document}
\baselineskip=15pt
\textheight=8.4in
\parindent=0pt
\def\sk {\hskip .5cm}
\def\skv {\vskip .12cm}
\def\cos {\mbox{cos}}
\def\sin {\mbox{sin}}
\def\tan {\mbox{tan}}
\def\intl{\int\limits}
\def\lm{\lim\limits}
\newcommand{\frc}{\displaystyle\frac}
\def\xbf{{\mathbf x}}
\def\fbf{{\mathbf f}}
\def\gbf{{\mathbf g}}

\def\Ker{{\rm Ker\,}}
\def\phi{\varphi}

\def\dbA{{\mathbb A}}
\def\dbB{{\mathbb B}}
\def\dbC{{\mathbb C}}
\def\dbD{{\mathbb D}}
\def\dbE{{\mathbb E}}
\def\dbF{{\mathbb F}}
\def\dbG{{\mathbb G}}
\def\dbH{{\mathbb H}}
\def\dbI{{\mathbb I}}
\def\dbJ{{\mathbb J}}
\def\dbK{{\mathbb K}}
\def\dbL{{\mathbb L}}
\def\dbM{{\mathbb M}}
\def\dbN{{\mathbb N}}
\def\dbO{{\mathbb O}}
\def\dbP{{\mathbb P}}
\def\dbQ{{\mathbb Q}}
\def\dbR{{\mathbb R}}
\def\dbS{{\mathbb S}}
\def\dbT{{\mathbb T}}
\def\dbU{{\mathbb U}}
\def\dbV{{\mathbb V}}
\def\dbW{{\mathbb W}}
\def\dbX{{\mathbb X}}
\def\dbY{{\mathbb Y}}
\def\dbZ{{\mathbb Z}}

\def\Aut{{\rm Aut}}
\def\GL{{\rm GL}}
\def\SL{{\rm SL}}
\def\rk{{\rm rk}}
\def\lam{{\lambda}}

\def\la{{\langle}}
\def\ra{{\rangle}}

\bf\centerline{Homework \# 2, to be submitted on Canvas by 11:59pm on Fri, Jan 30th.}\rm
\vskip .2cm
{\bf Plan for the next 2 classes (Jan 27 and 29):}  Modules over PIDs, continued (12.1 in DF; Lectures 3-5 from Spring 21 and Lecture 7-9 from Spring 10). Possibly start talking about the rational canonical form (12.2 in DF; Lecture 6-7 from Spring 21 and Lectures 10-11 from Spring 10).
\vskip .1cm

{\bf Note on hints:} Some hints are given at the end of the assignment, each on a separate page.
Problems (or parts of problems) for which a hint is available at the end are marked with *.

\vskip .1cm


{\bf Problem 1.} Let $R$ be a Euclidean domain and $n\geq 2$ an integer.
\begin{itemize}
\item[(a)] Use the proof of the Smith Normal Form theorem from class to show that every matrix $A\in \GL_n(R)$ can be written as a product
of elementary matrices $E_{ij}(\lambda)$, flip matrices $F_{ij}$ and a diagonal matrix $D$.
\item[(b)] Now show that the flip matrices can be eliminated from the product in (a), and one can assume that $D=diag(d,1,\ldots, 1)$,
that is, all diagonal entries of $D$ except possibly the $(1,1)$-entry are equal to $1$.
\item[(c)] Deduce from (b) that $\SL_n(R)$ (the subgroup of matrices of determinant $1$) is generated by the elementary matrices
$E_{ij}(\lambda)$.
\end{itemize} 
\skv
{\bf Problem 2.} Let $R$ be a Euclidean domain, let $k,n\in\dbN$ and $i\leq \min\{k,n\}$. Given a matrix $A\in Mat_{k\times n}(R)$,
define $d_i(A)$ to be the gcd of all $i\times i$ minors of $A$. Prove that 
$d_i(PAQ)=d_i(A)$ for all $P\in \GL_k(R)$ and $Q\in \GL_n(R)$. Recall that this was a key fact in the proof
of uniqueness of the Smith Normal Form. {\bf Hint:} Use Problem~1.
\skv
{\bf Problem 3:} Let $R$ be a commutative ring (with 1).
\begin{itemize}
\item[(a)] Let $C$ be an $R$-algebra and let $A$ and $B$ be $R$-subalgebras of $C$
which commute with each other, that is, $ab=ba$ for any $a\in A, b\in B$
(note that $A$ and $B$ themselves do not have to be commutative).
Prove that there is an \underline{$R$-algebra homomorphism} $\phi:A\otimes_R B\to C$
such that $\phi(a\otimes b)=ab$ for each $a\in A$ and $b\in B$.

\item[(b)] Prove that $\dbR\otimes_{\dbZ} \dbZ[i]\cong \dbC$ as rings (as usual $\dbR$
is real numbers and $\dbC$ are complex numbers).

\item[(c)] Now assume that $R$ is a field, and let $A$ be a finite-dimensional
$R$-algebra. Prove that the algebra $A\otimes_R A$ cannot be a field
unless $\dim_R A=1$. {\bf Hint:} use (a).
\end{itemize}

\skv
{\bf Problem 4.} 
\begin{itemize}
\item[(a)] Prove that $A=\dbC\otimes_{\dbR}\dbC$ and $B=\dbC \times \dbC$ are isomorphic as $\dbC$-algebras. Here $\dbC$ acts on 
$\dbC \times \dbC$ in the usual way, that is, $\lam(a,b)=(\lam a,\lam b)$ and on $\dbC \otimes \dbC$ by $\lam(a\otimes b)=(\lam a\otimes b)$
(so we apply the extension of scalars construction where the second copy of $\dbC$ is viewed as the original $\dbR$-algebra
and the first copy of $\dbC$ is used for the scalar extension).

{\bf Hint:} You may use without proof the following generalization of HW\#1.8: if $R$ is a subring of a commutative ring $S$
with $1_R=1_S$, then for any polynomial $f(x)\in R[x]$ we have $$S\times (R[x]/(f(x))\cong S[x]/(f(x))$$ as $S$-algebras.

\item[(b)] Explain why $\{1\otimes 1, 1\otimes i\}$ is a basis for $A$ over $\dbC$. Now compute $\phi(1\otimes 1)$ and
 $\phi(1\otimes i)$ where $\phi:A\to B$ is your isomorphism from (a) (note that $\phi$ is completely determined by its values on a $\dbC$-basis of $A$).
 \item[(c)*] Prove that there exist precisely $2$ $\dbC$-algebra isomorphisms from $A$ to $B$. {\bf Hint:} First prove $\geq 2$ and then $\leq 2$.
\end{itemize}
\skv
{\bf Problem 5. } Let $V$ and $W$ be finite dimensional vector spaces
over a field $F$, let $\{v_1,\ldots, v_n\}$ be a basis of $V$
and $\{w_1,\ldots, w_m\}$ a basis of $W$.

Let  $\phi: V\otimes_F W\to Mat_{n\times m}(F)$ be the
$F$-linear transformation such that $\phi(v_i\otimes w_j)=e_{ij}$
where $e_{ij}$ is the matrix whose $(i,j)$-entry is equal to $1$ and all
other entries are equal to $0$ (note that such transformation
exists and is unique because $\{v_i\otimes w_j : 1\leq i\leq n, 1\leq j\leq m\}$
is a basis for $V\otimes_F W$; furthermore, $\phi$ is an isomorphism
since matrices $\{e_{ij}\}$ form a basis of $Mat_{n\times m}(F)$).

\begin{itemize}
\item[(a)] Prove that for a matrix $A\in Mat_{n\times m}(F)$ the following are equivalent:
\begin{itemize}
\item[(i)] $A=\phi(v\otimes w)$ for some $v\in V, w\in W$ (note: $v$ and $w$
need not be elements of the above bases)
\item[(ii)] $rk(A)\leq 1$.
\end{itemize}
\item[(b)] Let $A\in Mat_{n\times m}(F)$. Prove that $\rk(A)$ is the smallest $d$
such that $\phi^{-1}(A)$ can be written as a sum of $d$ simple tensors.
\end{itemize}
\skv
{\bf Problem 6*.} Let $R$ be a ring (with 1), let $M$ be a left $R$-module
and $N$ its submodule. Prove that $M$ is Noetherian $\iff$
$N$ and $M/N$ are both Noetherian. {\bf Note:} we will discuss Noetherian modules at the beginning of Lecture~5.
\skv
\newpage

{\bf Hint for 4:} \begin{itemize}
\item To prove that there are at least $2$ $\dbC$-algebra isomorphisms from $A$ to $B$ show that either $A$ or $B$ has a non-trivial $\dbC$-algebra automorphism (the assertion is true for both $A$ and $B$, but you only need to prove it for one of them). 

\item
To prove $\leq 2$ find enough restrictions on $\psi(1\otimes 1)$ and 
$\psi(1\otimes i)$ where $\psi:A\to B$ is a $\dbC$-algebra isomorphism to deduce that there are at most $2$ choices
for the pair $(\psi(1\otimes 1), \psi(1\otimes i))$.
\end{itemize}


\newpage

{\bf Hint for 6:} The forward direction is easy. For the backwards direction,
observe that if $\{P_i\}$ is an ascending chain of submodules of $M$,
then $\{P_i\cap N\}$ is an ascending chain of submodules of $N$
and $\{(P_i+N)/N\}$ is an ascending chain of submodules of $M/N$.





\end{document}
