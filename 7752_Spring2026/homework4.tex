\documentclass[11pt]{amsart}

\usepackage{amsmath}
\usepackage{amssymb}
\usepackage{amsthm}
%\usepackage{psfig}

\begin{document}
\baselineskip=15pt
\textheight=8.4in
\parindent=0pt
\def\sk {\hskip .5cm}
\def\skv {\vskip .12cm}
\def\cos {\mbox{cos}}
\def\sin {\mbox{sin}}
\def\tan {\mbox{tan}}
\def\intl{\int\limits}
\def\lm{\lim\limits}
\newcommand{\frc}{\displaystyle\frac}
\def\xbf{{\mathbf x}}
\def\fbf{{\mathbf f}}
\def\gbf{{\mathbf g}}

\def\Ker{{\rm Ker\,}}
\def\phi{\varphi}

\def\dbA{{\mathbb A}}
\def\dbB{{\mathbb B}}
\def\dbC{{\mathbb C}}
\def\dbD{{\mathbb D}}
\def\dbE{{\mathbb E}}
\def\dbF{{\mathbb F}}
\def\dbG{{\mathbb G}}
\def\dbH{{\mathbb H}}
\def\dbI{{\mathbb I}}
\def\dbJ{{\mathbb J}}
\def\dbK{{\mathbb K}}
\def\dbL{{\mathbb L}}
\def\dbM{{\mathbb M}}
\def\dbN{{\mathbb N}}
\def\dbO{{\mathbb O}}
\def\dbP{{\mathbb P}}
\def\dbQ{{\mathbb Q}}
\def\dbR{{\mathbb R}}
\def\dbS{{\mathbb S}}
\def\dbT{{\mathbb T}}
\def\dbU{{\mathbb U}}
\def\dbV{{\mathbb V}}
\def\dbW{{\mathbb W}}
\def\dbX{{\mathbb X}}
\def\dbY{{\mathbb Y}}
\def\dbZ{{\mathbb Z}}

\def\Aut{{\rm Aut}}
\def\Hom{{\rm Hom}}
\def\GL{{\rm GL}}
\def\SL{{\rm SL}}
\def\SNF{{\rm SNF}}
\def\rk{{\rm rk}}
\def\lam{{\lambda}}

\def\la{{\langle}}
\def\ra{{\rangle}}

\bf\centerline{Homework \# 4, to be submitted on Canvas by 6pm on Sat, Feb 14th.}\rm
\vskip .2cm
{\bf Plan for the next 2 classes (Feb 10 and 12):}  Jordan Canonical Form (12.3 in DF; Lecture 8 from Spring 21 and Lecture 12+ from Spring 10). Start talking about Representation Theory of Groups.
\vskip .1cm

%{\bf Note on hints:} Some hints are given at the end of the assignment, each on a separate page.
%Problems (or parts of problems) for which a hint is available at the end are marked with *.

\skv
{\bf Problem 1:} Let $F$ be a field, $a(x)=x^n+\sum\limits_{k=0}^{n-1}a_k x^k\in F[x]$ a non-constant monic polynomial,
and let $A=C_{a(x)}$ be its companion matrix. Prove by direct computation that 
$\SNF(xI-A)=diag(\underbrace{1,\ldots,1}_{n-1\mbox{ times }},a(x))$.
\skv

{\bf Problem 2:} DF, Problem 6, page 488. 
\skv

{\bf Problem 3:} Determine the number of possible RCFs of $8\times 8$ matrices $A$ over $\dbQ$ with 
$\chi_A(x)=x^8-x^4$. Explain your argument in detail.
\skv
{\bf Problem 4: } \rm Prove that two $3\times 3$ matrices
over some field $F$ are similar if and only if they have the same minimal and characteristic 
polynomials. Give an example showing that this does not hold for $4\times 4$ matrices.
\skv
{\bf Problem 5: } Find the number of distinct conjugacy classes in the group $\GL_3(\dbF_2)$ (where $\dbF_2$ is the field with
$2$ elements) and specify one element in each conjugacy class.
\skv
{\bf Problem 6: } \rm
Prove that there is no matrix $A\in Mat_{10}(\dbQ)$ satisfying $A^4=-I$ (where $I$ is the identity matrix).
\skv
{\bf Problem 7: }\rm DF, Problem~15 on page 500. {\bf Hint:} This problem does not require long and tedious computations.
\skv
{\bf Problem 8: }\rm DF, Problem~20 on page 501. Also find an explicit $P\in \GL_n(F)$ such that $A=PBP^{-1}$ where $A$ and
$B$ are the two matrices in the form (you may choose which one
is $A$ and which one is $B$).
\skv
{\bf Problem 9:} Let $V$ be an $n$-dimensional vector space over some field $F$,
and let $T:V\to V$ be a {\bf nilpotent} $F$-linear map. Prove that $T^n=0$ in two different ways:
\begin{itemize}
\item[(a)] using JCF
\item[(b)] without using JCF or RCF, but instead looking at the sequence of kernels $\{\Ker(T^k)\}_{k=1}^{\infty}$. 
\end{itemize}
\end{document}
