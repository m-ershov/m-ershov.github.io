\documentclass[12pt]{amsart}

\usepackage{amsmath}
\usepackage{amssymb}
\usepackage{amsthm}
%\usepackage{psfig}

\begin{document}
\baselineskip=16pt
\textheight=8.5in
%\parindent=0pt 
\def\sk {\hskip .5cm}
\def\skv {\vskip .08cm}
\def\cos {\mbox{cos}}
\def\sin {\mbox{sin}}
\def\tan {\mbox{tan}}
\def\intl{\int\limits}
\def\lm{\lim\limits}
\newcommand{\frc}{\displaystyle\frac}
\def\xbf{{\mathbf x}}
\def\fbf{{\mathbf f}}
\def\gbf{{\mathbf g}}

\def\dbA{{\mathbb A}}
\def\dbB{{\mathbb B}}
\def\dbC{{\mathbb C}}
\def\dbD{{\mathbb D}}
\def\dbE{{\mathbb E}}
\def\dbF{{\mathbb F}}
\def\dbG{{\mathbb G}}
\def\dbH{{\mathbb H}}
\def\dbI{{\mathbb I}}
\def\dbJ{{\mathbb J}}
\def\dbK{{\mathbb K}}
\def\dbL{{\mathbb L}}
\def\dbM{{\mathbb M}}
\def\dbN{{\mathbb N}}
\def\dbO{{\mathbb O}}
\def\dbP{{\mathbb P}}
\def\dbQ{{\mathbb Q}}
\def\dbR{{\mathbb R}}
\def\dbS{{\mathbb S}}
\def\dbT{{\mathbb T}}
\def\dbU{{\mathbb U}}
\def\dbV{{\mathbb V}}
\def\dbW{{\mathbb W}}
\def\dbX{{\mathbb X}}
\def\dbY{{\mathbb Y}}
\def\dbZ{{\mathbb Z}}

\def\la{{\langle}}
\def\ra{{\rangle}}
\def\summ{{\sum\limits}}

\bf\centerline{Homework \#5. Due Saturday, February 26th, by 11:59pm in filedrop}\rm
\vskip .1cm
All reading assignments and references to exercises, definitions etc. are from our main book `Coding Theory: A First Course' by Ling and Xing 
\vskip .1cm


\bf\centerline{Reading and plan for the next week: }\rm
\skv
1. For this homework assignment read 4.8
\skv
\skv
2. Next week we will start talking about bounds for codes (Chapter 5). I am not sure about the order, but we will almost definitely discuss the sphere-covering bound (5.2) and sphere-packing bound (5.3) next week.
\skv

\skv
\bf\centerline{Problems: }\rm
\skv

{\bf 1.} Problem 4.27. {\bf Hint:} This problem can be solved using the same idea as 4.20(i), (ii), but there is a more conceptual solution involving cosets.
\skv

{\bf 2.} Problem 4.42. 
\skv

{\bf 3.} Problem 4.47. For the decoding part use the ``modified'' syndrome decoding as described below. If $w$ is the received word, compute its syndrome $S(w)$ and then find a leader of the coset $w+C$ using the result of Problem 2 (do not compute the entire coset $w+C$).
\skv

{\bf 4.} Problem 4.44.
\skv

{\bf 5.} Let $r\geq 2$, let $N=2^r-1$ and let $Ham(r,2)$ be the binary Hamming code of length $N$ (see 5.3.1). Recall that we proved in Lecture~9 that $Ham(r,2)$ has distance 3. 
\begin{itemize}
\item[(a)] Consider the following $N+1=2^r$ elements: $0,e_1,\ldots, e_{N}$ (where $e_i$ is the $i^{\rm th}$ element of the standard basis). Prove that every coset of $Ham(r,2)$ in $\dbF_2^N$ contains exactly one of these elements.
\item[(b)] Deduce from (a) that for every $w\in \dbF_2^N$ there exists $c\in Ham(r,2)$ such that $d(w,c)\leq 1$. 
\item[(c)] Based on (b), what can you say about the optimized syndrome decoding (as stated at the beginning of Lecture~9
on Thu, February 17)?
\end{itemize}
\skv
\end{document}


