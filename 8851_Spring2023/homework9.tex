\documentclass[12pt]{amsart}

\usepackage{amsmath}
\usepackage{amssymb}
\usepackage{amsthm}
\usepackage{hyperref}
\usepackage{url}

%\usepackage{psfig}

\newtheorem* {Theorem}    {Theorem}
\newtheorem* {Lemma}    {Lemma}
\newtheorem* {Remark}	{\bf{Remark}}

\begin{document}
 \pagenumbering{gobble}
\baselineskip=16pt
\textheight=8.5in
\textwidth=6.5in
%\parindent=0pt 
\def\sk {\hskip .5cm}
\def\skv {\vskip .08cm}
\def\cos {\mbox{cos}}
\def\sin {\mbox{sin}}
\def\tan {\mbox{tan}}
\def\intl{\int\limits}
\def\lm{\lim\limits}
\newcommand{\frc}{\displaystyle\frac}
\def\xbf{{\mathbf x}}
\def\fbf{{\mathbf f}}
\def\gbf{{\mathbf g}}

\def\dbA{{\mathbb A}}
\def\dbB{{\mathbb B}}
\def\dbC{{\mathbb C}}
\def\dbD{{\mathbb D}}
\def\dbE{{\mathbb E}}
\def\dbF{{\mathbb F}}
\def\dbG{{\mathbb G}}
\def\dbH{{\mathbb H}}
\def\dbI{{\mathbb I}}
\def\dbJ{{\mathbb J}}
\def\dbK{{\mathbb K}}
\def\dbL{{\mathbb L}}
\def\dbM{{\mathbb M}}
\def\dbN{{\mathbb N}}
\def\dbO{{\mathbb O}}
\def\dbP{{\mathbb P}}
\def\dbQ{{\mathbb Q}}
\def\dbR{{\mathbb R}}
\def\dbS{{\mathbb S}}
\def\dbT{{\mathbb T}}
\def\dbU{{\mathbb U}}
\def\dbV{{\mathbb V}}
\def\dbW{{\mathbb W}}
\def\dbX{{\mathbb X}}
\def\dbY{{\mathbb Y}}
\def\dbZ{{\mathbb Z}}

\def\la{{\langle}}
\def\ra{{\rangle}}
\def\Ker{{\rm Ker\,}}
\def\Aut{{\rm Aut}}
\def\Out{{\rm Out}}
\def\Inn{{\rm Inn}}
\def\IA{{\rm IA}}
\def\GL{{\rm GL}}
\def\rk{{\rm rk}}
\def\summ{{\sum\limits}}
\def\phi{{\varphi}}

\bf\centerline{Math 8851. Homework \#9. To be completed by Thu, Apr 13}\rm
\vskip .3cm
{\bf 1.} Fix a prime $p$. Recall that given a group presentation $(X,R)$ where $X$ is finite and $R$ is possibly infinite
we define the $p$-deficiency $def_p(X,R)\in \dbR\cup\{-\infty\}$ by
$$def_p(X,R)=|X|-1-\sum_{r\in R}\frac{1}{p^{\deg_{F(X)}(r)}}.$$
Recall that $\deg_{F(X)}(r)$ is the largest $k\in\dbZ_{\geq 0}$ such that $r=u^{p^k}$ for some $u\in F(X)$.
\skv

If $G$ is a finitely generated group, we define $def_p(G)=\sup\{def_p(X,R)\}$ where $(X,R)$ ranges over all presentations
of $G$ by generators and relators with $X$ finite.

Now let $G^p$ be the subgroup generated by all $p^{\rm th}$ powers $\{g^p: g\in G\}$ and
consider (again assuming $G$ is finitely generated) the quotient $G/[G,G]G^p$ where  -- it is 
a finite abelian group of exponent $p$ and thus can be considered as a vector space over $\mathbb F_p$. Denote
by $d_p(G)$ the dimension of this space (equivalently, $d_p(G)=\log_p[G:[G,G]G^p]$).
\begin{itemize}
\item[(a)] Prove that $d_p(G)\geq def_p(G)+1$.
\item[(b)] Deduce that if $def_p(G)>-1$, then $G$ has a normal subgroup of index $p$. 
\item[(c)] Assume that $H$ is a normal subgroup of $G$ of $p$-power index. Prove that $def_p(H)\geq [G:H] def_p(G)$
(use Proposition~19.3 from class which asserts that this is true when $H$ is normal of index $p$). 
\item[(d)] Now assume that $def_p(G)>0$ and $G$ is finitely presented. Use (a), (c)
and Theorem~1.12 in the following paper of M. Lackenby 
\skv
\centerline{\url{http://arxiv.org/abs/math/0702571}}
\skv
to prove that $G$ has a finite index subgroup which homomorphically maps
onto a non-abelian free group.
\end{itemize}
\skv
{\bf 2.} Let $G$ be a group and $\{G_n\}_{n=1}^{\infty}$ a central series of $G$, that is, a descending chain of normal subgroup of $G$ where $G_1=G$ and $[G_i,G_j]\subseteq G_{i+j}$ for all $i$ and $j$. Recall the definition of the associate Lie ring $L(G)$.

As a set $L(G)=\bigoplus\limits_{n=1}^{\infty}L_n(G)$ where $L_n(G)=G_n/G_{n+1}$. Elements of $L$ which lie
in $L_n(G)$ for some $n$ are called homogeneous.

The addition on each $L_n(G)$ is simply the quotient group operation (note that $G_n/G_{n+1}$ is abelian since 
$[G_n,G_n]\subseteq G_{2n}\subseteq G_{n+1}$). 

The Lie bracket is defined as follows. First given homogeneous elements $u\in L_n(G)$ and $v\in L_m(G)$ we choose $g\in G_n$ and $h\in G_m$ such that $u=gG_{n+1}$ and $v=hG_{m+1}$ and set $[u,v]=[g,h]G_{n+m+1}$
where $[g,h]=g^{-1}h^{-1}gh$ is the group commutator of $g$ and $h$.

Given arbitrary elements $u,v\in L$, we write them as sums of homogeneous components $u=\sum u_i$ and
$v=\sum v_j$ and set $[u,v]=\sum\limits_{i,j}[u_i,v_j]$.

\begin{itemize} 
\item[(a)] Prove that the Lie bracket is well defined, that is, in the definition of $[u,v]$ in the homogeneous case the value is independent of the choice of $g$ and $h$.
\item[(b)] Prove that $L(G)$ with the above operations is a Lie ring, that is, satisfies the following axioms:
\begin{itemize}
\item[(1)] $(L(G),+)$ is an abelian group.
\item[(2)] $[x,y+z]=[x,y+z]$ and $[x+y,z]=[x,z]+[y,z]$ for all $x,y,z\in L(G)$.
\item[(3)] $[x,x]=0$ for all $x\in L(G)$.
\item[(4)] $[[x,y],z]+[[y,z],x]+[[z,y],x]=0$ for all $x,y,z\in L(G)$.
\end{itemize} 
{\bf Hint:} To prove (4) use the following group-theoretic identity called the Hall-Witt identity:
$[[a,c],c^a][[c,a],b^c][[b,c],a^b] = 1$ where $a,b,c$ are elements of some group $G$. You will also need to  use of HW\#8.2 along with the observation that $a^b=a[a,b]$ for all $a,b\in G$ to prove (2),(3) and (4).
\skv
{\bf Warning:} You need to prove (2), (3) and (4) for all elements of $L(G)$, not just homogeneous ones. The reduction to the homogeneous case is straighforward for (2) and (4), but not for (3).

\end{itemize}
\skv
{\bf 3.} Given $n\geq 2$ and an ordered $n$-tuple $g_1,\ldots, g_n$ of elements of a group $G$, the {\it left-normed commutator} $[g_1,\ldots, g_n]$ is defined as follows. If $n=2$, this is the usual commutator,
and for $n\geq 3$ we define $[g_1,\ldots, g_n]$ inductively by $[g_1,\ldots, g_n]=[[g_1,\ldots, g_{n-1}],g_n]$.
\begin{itemize}
\item[(a)] Let $G$ be a group. Prove that $\gamma_n G$ (the $n^{\rm th}$ term of the lower central series of $G$) is generated by all left-normed commutators of length $n$.
\item[(b)] Again let $G$ be a group, and let $L(G)$ be the Lie ring associated to the lower central series of $G$ (that is, we set $G_n=\gamma_n G$ in the definition of $L(G)$). Use (a) to prove that $L(G)$ is generated
as a Lie ring by $L_1(G)=G_1/G_2$.
\end{itemize}
\skv

{\bf 4.} Recall from class that a group $G$ is {\it residually finite} if it satisfies the following equivalent conditions:
\begin{itemize}
\item[(1)] For every $1\neq g\in G$ there exists a finite index subgroup $N$ of $G$ such that $g\not\in N$.
\item[(2)] For every $1\neq g\in G$ there exists a finite index normal subgroup $N$ of $G$ such that $g\not\in N$.
\item[(3)] For every $1\neq g\in G$ there exists a finite group $P$ and a homomorphism $\phi:G\to P$ such that
$\phi(g)\neq 1$.
\item[(3')] For every two distinct elements $x,y\in G$ there exists a finite group $P$ and a homomorphism $\phi:G\to P$ such that $\phi(x)\neq \phi(y)$.
\item[(4)] $G$ is a subgroup of a direct product of some family of finite groups.
\end{itemize}

Now the actual problem.

\begin{itemize}
\item[(a)] Prove that the above conditions are indeed equivalent.
\item[(b)] Suppose we are given a short exact sequence of groups $1\to K\to G\to Q\to 1$
where $K$ is residually finite and $Q$ is finite. Prove that $G$ is residually finite.
\item[(c)] Again assume that $1\to K\to G\to Q\to 1$ is exact, but now $K$ is finite and $Q$ is residually finite. Is $G$ residually finite? Prove or give a counterexample.
\end{itemize}
\end{document}
