\documentclass[12pt]{amsart}

\usepackage{amsmath}
\usepackage{amssymb}
\usepackage{amsthm}
\usepackage{hyperref}
\usepackage{url}

%\usepackage{psfig}

\newtheorem* {Theorem}    {Theorem}
\newtheorem* {Lemma}    {Lemma}
\newtheorem* {Remark}	{\bf{Remark}}

\begin{document}
 \pagenumbering{gobble}
\baselineskip=16pt
\textheight=8.5in
\textwidth=6.5in
%\parindent=0pt 
\def\sk {\hskip .5cm}
\def\skv {\vskip .08cm}
\def\cos {\mbox{cos}}
\def\sin {\mbox{sin}}
\def\tan {\mbox{tan}}
\def\intl{\int\limits}
\def\lm{\lim\limits}
\newcommand{\frc}{\displaystyle\frac}
\def\xbf{{\mathbf x}}
\def\fbf{{\mathbf f}}
\def\gbf{{\mathbf g}}

\def\dbA{{\mathbb A}}
\def\dbB{{\mathbb B}}
\def\dbC{{\mathbb C}}
\def\dbD{{\mathbb D}}
\def\dbE{{\mathbb E}}
\def\dbF{{\mathbb F}}
\def\dbG{{\mathbb G}}
\def\dbH{{\mathbb H}}
\def\dbI{{\mathbb I}}
\def\dbJ{{\mathbb J}}
\def\dbK{{\mathbb K}}
\def\dbL{{\mathbb L}}
\def\dbM{{\mathbb M}}
\def\dbN{{\mathbb N}}
\def\dbO{{\mathbb O}}
\def\dbP{{\mathbb P}}
\def\dbQ{{\mathbb Q}}
\def\dbR{{\mathbb R}}
\def\dbS{{\mathbb S}}
\def\dbT{{\mathbb T}}
\def\dbU{{\mathbb U}}
\def\dbV{{\mathbb V}}
\def\dbW{{\mathbb W}}
\def\dbX{{\mathbb X}}
\def\dbY{{\mathbb Y}}
\def\dbZ{{\mathbb Z}}

\def\la{{\langle}}
\def\ra{{\rangle}}
\def\Ker{{\rm Ker\,}}
\def\Aut{{\rm Aut}} 
\def\Out{{\rm Out}}
\def\Inn{{\rm Inn}}
\def\IA{{\rm IA}}
\def\GL{{\rm GL}}
\def\rk{{\rm rk}}
\def\summ{{\sum\limits}}
\def\phi{{\varphi}}

\bf\centerline{Math 8851. Homework \#10. To be completed by Thu, Apr 20}\rm
\vskip .3cm
{\bf 1.} Let $p$ be a prime. As stated in class, if $G$ is any group such that $g^p=1$ for all $g\in G$ and $L(G)$ is the graded Lie ring associated to the lower central series of $G$, then $L(G)$ is a Lie algebra over $\mathbb F_p$ satisfying the Engel identity $(E_{p-1})$:
$$ [y,\underbrace{x,\ldots, x}_{p-1\mbox{ times}}]\mbox{ for all }x,y\in L(G)\eqno{(E_{p-1})}$$
where the bracket is left-normed.

The goal of this problem is prove an analogous result for associative algebras. Let $A$ be an associative algebra (without 1) satisfying
the identity $a^p=0$ for all $a\in A$. Prove that considered as a Lie algebra (with the bracket $[x,y]=xy-yx$), $A$ satisfies $(E_{p-1})$.
{\bf Hint:} Consider the identity $(x+\lambda y)^p=0$ with $x,y\in A$ and $\lambda\in\dbF_p$ and compute the coefficient of $\lambda$.
\skv
{\bf 2.} Let $L$ be a Lie algebra over $\dbF_3$ satisfying the Engel identity $(E_2)$, that is, $[y,x,x]=0$ for all $x,y\in L$. Prove that
$\gamma_4 L=0$ (that is, $L$ is nilpotent of class at most $3$). Moreover, show that if $X$ generates $L$ as a Lie ring and we choose a total
order on $X$, then $L$ is spanned by left-normed commutators of length $\leq 3$ where all the entries are elements of $X$ and appear in decreasing order (that is, $X$ itself, $[x,y]$ with $x>y$ and $[x,y,z]$ with $x>y>z$). 

Deduce (using the results from class and HW\#9) that if $G$ is any group generated by $d$ elements and satisfying $g^3=1$ for all $g\in G$, then
$|G|\leq 3^{d+{d \choose 2}+{d\choose 3}}$.


You may use without proof that if an arbitrary Lie ring 
$L$ is generated by a set $X$ as a Lie ring, then left-normed commutators of elements of $X$ span $L$ (as an additive group).
Here is a proposed sketch of the proof (all brackets below are left-normed):
\begin{itemize}
\item[(a)] Replacing $x$ by $x_1+x_2$ in $(E_2)$ and using the Jacobi identity, show that for any $x_1,x_2,x_3\in L $
and any $\sigma\in S_3$ (the symmetric group on $\{1,2,3\}$) we have the identity 
$$[x_{\sigma(1)},x_{\sigma(2)},x_{\sigma(3)}]=(-1)^{\sigma}[x_1,x_2,x_3].$$
\item[(b)] Now use (a) (and $(E_2)$ and Jacobi identity again) to show that $[x,y,z,w]=0$ whenever there is a repetition in the sequence $x,y,z,w$.
\item[(c)] Now starting with the result of (b) and applying the same trick as in (a), show that 
$$[x_{\sigma(1)},x_{\sigma(2)},x_{\sigma(3)}, x_{\sigma(4)}]=(-1)^{\sigma}[x_1,x_2,x_3,x_4]$$ for all $x_1,x_2,x_3,x_4\in L$ and $\sigma\in S_4$.
\item[(d)] Finally, use (c) and the Lie algebra axioms again to show that all left-normed commutators of length $4$ are equal to $0$.
The moreover part of the assertion of the problem should follow from what you already established in (a).
\end{itemize}
\skv
{\bf 3}. Fix a prime $p$ and an integer $d$, and define the sequence $\{f_k(d,p)\}_{k=1}^{\infty}$ by $f_1(d,p)=d$ and
$f_{k}(d,p)=1+(f_{k-1}(d,p)-1)p^{f_{k-1}(d,p)}$ for $k\geq 2$. Prove that for all $k\in\dbN$ there exists a $d$-generated group $G_{k}(d,p)$ such that $g^{p^k}=1$ for all $g\in G_k(d,p)$ and $|G_k(d,p)|=f_k(d,p)$ (this provides a lower bound for the order of the restricted Burnside group $R(d,p^k)$). {\bf Hint:} Let $F_d$ be a free group of rank $d$. Construct a suitable sequence of normal subgroups $F_d=N_0\supseteq N_1\supseteq N_2\supseteq\cdots \supseteq N_k$ such that $N_i/N_{i+1}$ has exponent $p$ and define
$G_k(d,p)=F_d/N_k$.
\skv
{\bf Note:} The sequence $f_{k}(d,p)$ grows very fast. In particular, $f_k(d,p)\geq p^{p^{p^{\ldots^{p^d}}}}$ where the total number of exponentiations is $k$. There is a known upper bound for $R(d,p^k)$ of the same form (iterated exponentials), but the number of exponentiations is considerably larger.
  
\end{document}
