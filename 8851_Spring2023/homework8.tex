\documentclass[12pt]{amsart}

\usepackage{amsmath}
\usepackage{amssymb}
\usepackage{amsthm}
\usepackage{hyperref}
\usepackage{url}

%\usepackage{psfig}

\newtheorem* {Theorem}    {Theorem}
\newtheorem* {Lemma}    {Lemma}
\newtheorem* {Remark}	{\bf{Remark}}

\begin{document}
 \pagenumbering{gobble}
\baselineskip=16pt
\textheight=8.5in
\textwidth=6.5in
%\parindent=0pt 
\def\sk {\hskip .5cm}
\def\skv {\vskip .08cm}
\def\cos {\mbox{cos}}
\def\sin {\mbox{sin}}
\def\tan {\mbox{tan}}
\def\intl{\int\limits}
\def\lm{\lim\limits}
\newcommand{\frc}{\displaystyle\frac}
\def\xbf{{\mathbf x}}
\def\fbf{{\mathbf f}}
\def\gbf{{\mathbf g}}

\def\dbA{{\mathbb A}}
\def\dbB{{\mathbb B}}
\def\dbC{{\mathbb C}}
\def\dbD{{\mathbb D}}
\def\dbE{{\mathbb E}}
\def\dbF{{\mathbb F}}
\def\dbG{{\mathbb G}}
\def\dbH{{\mathbb H}}
\def\dbI{{\mathbb I}}
\def\dbJ{{\mathbb J}}
\def\dbK{{\mathbb K}}
\def\dbL{{\mathbb L}}
\def\dbM{{\mathbb M}}
\def\dbN{{\mathbb N}}
\def\dbO{{\mathbb O}}
\def\dbP{{\mathbb P}}
\def\dbQ{{\mathbb Q}}
\def\dbR{{\mathbb R}}
\def\dbS{{\mathbb S}}
\def\dbT{{\mathbb T}}
\def\dbU{{\mathbb U}}
\def\dbV{{\mathbb V}}
\def\dbW{{\mathbb W}}
\def\dbX{{\mathbb X}}
\def\dbY{{\mathbb Y}}
\def\dbZ{{\mathbb Z}}

\def\la{{\langle}}
\def\ra{{\rangle}}
\def\Ker{{\rm Ker\,}}
\def\Aut{{\rm Aut}}
\def\Out{{\rm Out}}
\def\Inn{{\rm Inn}}
\def\IA{{\rm IA}}
\def\GL{{\rm GL}}
\def\rk{{\rm rk}}
\def\summ{{\sum\limits}}
\def\phi{{\varphi}}

\bf\centerline{Math 8851. Homework \#8. To be completed by Thu, Apr 6}\rm
\vskip .3cm
{\bf 1.} Recall the statement of Gromov's polynomial growth theorem.

\begin{Theorem}[Gromov] Let $G$ be a finitely generated group of polynomial growth (that is, $b_G(n)\preceq n^d$ for some $d\in\dbN$).
Then $G$ is virtually nilpotent.
\end{Theorem}

Deduce Gromov's theorem from the following result:

\begin{Theorem}[Gromov's virtual fibering theorem] Let $G$
be a finitely generated group of polynomial growth. Then $G$ has a finite index subgroup $G_1$ which admits an epimorphism onto $\dbZ$.
\end{Theorem}

{\bf Hint:} Let $\pi:G_1\to\dbZ$ be the epimorphism from the virtual fibering theorem and let $H=\Ker(\pi)$. First show that $H$ is finitely generated (this follows immediately from one of the theorems from class).
Then use a direct counting argument to show that $b_G(n)\succeq b_H(n)\cdot n$. Finally use this inequality and the Milnor-Wolf theorem
to prove Gromov's theorem by induction on $d$. 
\skv



\skv
{\bf 2.} Let $G$ be a group.
\begin{itemize}
\item[(a)] Prove the following commutator identities 
(here $x,y,z$ are arbitrary elements of $G$):
\begin{itemize}
\item[(i)] $[xy,z] = [x,z]^y [y,z]=[x,z][x,z,y][y,z]$.
\item[(ii)] $[x,yz] = [x,z] [x,y]^z=[x,z][x,y][x,y,z]$.
\end{itemize}
\item[(b)] Use (a) to show that if $N$ and $M$ are normal
subgroups of $G$, $S$ is a subset of $N$ which generates
$N$ as a normal subgroup and $T$ is a subset of $M$ which generates
$M$ as a normal subgroup, then $[N,M]$
is normally generated by the set $\{[s,t]: s\in S, t\in T\}$.
\end{itemize}
\skv

{\bf 3.} The goal of this problem is to prove Lemma~20.2 from class
(the statement is recalled below). Let $G=\dbZ\ltimes_A \dbZ^n$
where $A\in \GL_n(\dbZ)$. Recall that by definition
$\dbZ\ltimes_A \dbZ^n$ is the group $\dbZ\ltimes_{\phi} \dbZ^n$
where $\phi:\dbZ\to\Aut(\dbZ^n)\cong\GL_n(\dbZ)$ is the unique homomorphism such that $\phi(1)=A$.

As in class, we will write $G$ as an internal semidirect product
$G=\la x\ra\ltimes V$ where $V\cong\dbZ^n$ and $x^{-1}vx=Av$
for all $v\in V$. Also recall that we are switching to additive
notation inside $V$.

Prove Lemma 20.2 which asserts that for all $k\geq 2$ $$\gamma_k G=(A-1)^{k-1}V.$$
{\bf Hint:} Induction on $k$. First use Problem~2 to prove that
$(A-1)^{k-1}V$ generates $\gamma_k G$ as a normal subgroup. Then
show that $(A-1)^{k-1}V$ is already a normal subgroup of $G$,
thus finishing the proof.
\skv

{\bf 4.} Consider the $p$-lamplighter group $G_p=\dbZ wr (\dbZ/p\dbZ)$.
\begin{itemize}
\item[(a)] Use the proof of Proposition~19.2 from class to exhibit specific elements $a,b\in G_p$ which generate a free subsemigroup.
\item[(b)] An HNN extension 
$G=\la H,t \mid t^{-1}at=\phi(a) \mbox{ for all }a\in A \ra$
is called {\it ascending} if $\phi(A)\subseteq A$ and 
{\it propertly ascending} if $\phi(A)$ is a proper subgroup of $A$.
Prove that $G_p$ can be realized as a propertly ascending HNN extension.
{\bf Hint:} Use the presentation from HW~2.4.
\skv
{\bf Note:} One can prove that every ascending HNN extension contains a free subsemigroup
arguing very similarly to the corresponding part of the proof of Proposition~19.2. One can also
show that if $\pi:G\to\dbZ$ is an epimorphism such that $G$ is finitely generated and
$H=\Ker\pi$ is not finitely generated, then $H$ can be represented as an ascending HNN extension.
This provides a more conceptual interpretation of the proof of Proposition~19.2 given in class.
\end{itemize}
\end{document}
