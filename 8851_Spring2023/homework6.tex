\documentclass[12pt]{amsart}

\usepackage{amsmath}
\usepackage{amssymb}
\usepackage{amsthm}
\usepackage{hyperref}
\usepackage{url}

%\usepackage{psfig}

\newtheorem* {Theorem}    {Theorem}
\newtheorem* {Lemma}    {Lemma}
\newtheorem* {Remark}	{\bf{Remark}}

\begin{document}
 \pagenumbering{gobble}
\baselineskip=16pt
\textheight=8.5in
\textwidth=6.5in
%\parindent=0pt 
\def\sk {\hskip .5cm}
\def\skv {\vskip .08cm}
\def\cos {\mbox{cos}}
\def\sin {\mbox{sin}}
\def\tan {\mbox{tan}}
\def\intl{\int\limits}
\def\lm{\lim\limits}
\newcommand{\frc}{\displaystyle\frac}
\def\xbf{{\mathbf x}}
\def\fbf{{\mathbf f}}
\def\gbf{{\mathbf g}}

\def\dbA{{\mathbb A}}
\def\dbB{{\mathbb B}}
\def\dbC{{\mathbb C}}
\def\dbD{{\mathbb D}}
\def\dbE{{\mathbb E}}
\def\dbF{{\mathbb F}}
\def\dbG{{\mathbb G}}
\def\dbH{{\mathbb H}}
\def\dbI{{\mathbb I}}
\def\dbJ{{\mathbb J}}
\def\dbK{{\mathbb K}}
\def\dbL{{\mathbb L}}
\def\dbM{{\mathbb M}}
\def\dbN{{\mathbb N}}
\def\dbO{{\mathbb O}}
\def\dbP{{\mathbb P}}
\def\dbQ{{\mathbb Q}}
\def\dbR{{\mathbb R}}
\def\dbS{{\mathbb S}}
\def\dbT{{\mathbb T}}
\def\dbU{{\mathbb U}}
\def\dbV{{\mathbb V}}
\def\dbW{{\mathbb W}}
\def\dbX{{\mathbb X}}
\def\dbY{{\mathbb Y}}
\def\dbZ{{\mathbb Z}}

\def\la{{\langle}}
\def\ra{{\rangle}}
\def\Ker{{\rm Ker\,}}
\def\Aut{{\rm Aut}}
\def\Out{{\rm Out}}
\def\Inn{{\rm Inn}}
\def\IA{{\rm IA}}
\def\rk{{\rm rk}}
\def\summ{{\sum\limits}}
\def\phi{{\varphi}}

\bf\centerline{Math 8851. Homework \#6. To be completed by Thu, Mar 16}\rm
\vskip .1cm

\skv
{\bf 1.} Give an example of a group $G$ acting on an {\bf unoriented} tree $T$ with edge inversions such that
\begin{itemize}
\item[(i)] the quotient graph $G\backslash T$ is a loop
\item[(ii)] $G$ cannot be decomposed as an HNN extension.
\end{itemize}
\begin{Remark}\rm Note that a group cannot act on an oriented tree $T$ with edge inversions 
(since in that case $T$ contains edges $(v,w)$ and $(w,v)$ for some vertices $v$ and $w$ and thus
$T$ is not a tree). Usually when one is taking about a group acting on a tree, it is assumed
that the tree is unoriented. In class we implicitly assumed that groups acts on oriented trees; we did not
lose any generality by doing so since we always required that the group acts without edge inversions. Indeed,
if $G$ acts on an unoriented tree $T$ without edge inversions, it is not hard to show that one can always
choose orientation on $T$ which is presereved under the action (pick one edge in every orbit of the action of
$G$ on $E(T)$ and orient those edges arbitrarily; then there is a unique way to orient the remaining edges
so that the orientation is preserved by the $G$-action).
\end{Remark}
\skv
{\bf 2.} Recall the formula for the rank of the Cartesian subgroup of the free product of a finite
family of finite groups from HW\#5.3(b):

If $G_1,\ldots, G_k$ are finite groups and $C$ is the Cartesian subgroup of $\ast G_i$, that is, the
kernel of the natural map $\ast G_i\to \prod G_i$, then
$$\rk(C)=\prod_{i=1}^k |G_i| \left(k-1-\sum_{i=1}^k\frac{1}{|G_i|}\right)+1.$$
Give another proof of this formula, this time using the Bass-Serre theory.

{\bf Hint:} Let $G=\ast G_i$, realize $G$ as $\pi_1(\dbG, Y)$
for some graph of groups $(\dbG,Y)$, as in the proof of the Kurosh Subgroup Theorem (KST), and consider the action of $G$ on the associated
Bass-Serre tree $X$. The proof of KST shows that to compute the rank of
$C$ it suffices to know the numbers of vertices and edges in the quotient
graph $C\backslash X$. To compute these first prove the following lemma.

\begin{Lemma} If $G$ is a group, $K$ is a finite subgroup of $G$ and
$H$ is a finite index normal subgroup of $G$ such that $K\cap H=\{1\}$, then the number of orbits of the left-multiplication action of $H$ on $G/K$ is $\frac{[G:H]}{|K|}$.
\end{Lemma}

\skv
{\bf 3.} Before doing this problem read about residually finite and hopfian groups in online Lecture~6. Prove Proposition~6.3 from online
notes which asserts that a finitely generated residually finite group
must be hopfian. {\bf Hint:} Let $G$ be finitely generated and $\phi:G\to G$ an epimorphism. Then $G/\Ker\phi\cong G$. Deduce that for any $n\in\dbN$ 
there is a bijection between the set of all normal subgroups of index $n$ in $G$ 
and the set of those normal sugbroups of index $n$ in $G$ which contain $\Ker\phi$. 
Then use HW\#5.4 to show that $\Ker\phi$ must lie in the intersection of all
finite index subgroups of $G$.
\skv
{\bf 4.} Given natural numbers $m$ and $n$, the Baumslag-Soliar group
$BS(m,n)$ is defined by
$$BS(m,n)=\la t,s \mid t^{-1}s^mt=s^n\ra.$$
\begin{itemize}
\item[(a)] Show that $BS(m,n)$ is an HNN-extension. 
\item[(b)] Prove that there exists an epimorphism
$\phi:BS(m,n)\to BS(m,n)$ such that $\phi(t)=t$ and $\phi(s)=s^2$.
\item[(c)] Now assume that $m=2$ and $n=3$. Show that
the elements $s$ and $s^t=t^{-1}st$ do not commute (use Britton's Lemma)
while their images under $\phi$ from (b) commute. Deduce that
$BS(2,3)$ is not Hopfian.
\end{itemize}
\end{document}