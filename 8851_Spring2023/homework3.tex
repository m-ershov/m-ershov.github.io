\documentclass[12pt]{amsart}

\usepackage{amsmath}
\usepackage{amssymb}
\usepackage{amsthm}
\usepackage{hyperref}
\usepackage{url}

%\usepackage{psfig}

\newtheorem* {Theorem}    {Theorem}
\newtheorem* {Lemma}    {Lemma}


\begin{document}
 \pagenumbering{gobble}
\baselineskip=16pt
\textheight=8.5in
\textwidth=6.5in
%\parindent=0pt 
\def\sk {\hskip .5cm}
\def\skv {\vskip .08cm}
\def\cos {\mbox{cos}}
\def\sin {\mbox{sin}}
\def\tan {\mbox{tan}}
\def\intl{\int\limits}
\def\lm{\lim\limits}
\newcommand{\frc}{\displaystyle\frac}
\def\xbf{{\mathbf x}}
\def\fbf{{\mathbf f}}
\def\gbf{{\mathbf g}}

\def\dbA{{\mathbb A}}
\def\dbB{{\mathbb B}}
\def\dbC{{\mathbb C}}
\def\dbD{{\mathbb D}}
\def\dbE{{\mathbb E}}
\def\dbF{{\mathbb F}}
\def\dbG{{\mathbb G}}
\def\dbH{{\mathbb H}}
\def\dbI{{\mathbb I}}
\def\dbJ{{\mathbb J}}
\def\dbK{{\mathbb K}}
\def\dbL{{\mathbb L}}
\def\dbM{{\mathbb M}}
\def\dbN{{\mathbb N}}
\def\dbO{{\mathbb O}}
\def\dbP{{\mathbb P}}
\def\dbQ{{\mathbb Q}}
\def\dbR{{\mathbb R}}
\def\dbS{{\mathbb S}}
\def\dbT{{\mathbb T}}
\def\dbU{{\mathbb U}}
\def\dbV{{\mathbb V}}
\def\dbW{{\mathbb W}}
\def\dbX{{\mathbb X}}
\def\dbY{{\mathbb Y}}
\def\dbZ{{\mathbb Z}}

\def\la{{\langle}}
\def\ra{{\rangle}}
\def\Ker{{\rm Ker}}
\def\Aut{{\rm Aut}}
\def\Inn{{\rm Inn}}
\def\IA{{\rm IA}}
\def\rk{{\rm rk}}
\def\summ{{\sum\limits}}
\def\phi{{\varphi}}

\bf\centerline{Math 8851. Homework \#3. To be completed by Thu, Feb 16}\rm
\vskip .1cm

{\bf 1.} Let $G=A\ltimes_{\phi}B$ be the (external) semidirect product of groups
$A$ and $B$ corresponding to some homomorphism $\phi:A\to \Aut(B)$. Suppose
we are given presentations by generators and relations for both $A$ and $B$:
$$A=\la X_1 | R_1\ra \qquad B=\la X_2 | R_2\ra. $$
Find (with proof) a presentation for $G$ in terms of $X_1,X_2,R_1,R_2$ and $\phi$.

{\bf Note:} If you succeeded in proving Hall's Theorem asserting that an extension of finitely presented groups is finitely presented (HW\# 2.2), this problem should be straightforward. If you did not succeed in solving HW\# 2.2, you may want to start with this problem and then come back to HW\# 2.2.

\vskip .1cm

{\bf 2.} Let $p$ be a prime. Prove that the lamplighter group 
$G_p=\dbZ\, wr\, \dbZ/p\dbZ$ is not finitely presented.

{\bf Hint:} As discussed in Lecture~5, if $G$ is a finitely presented group,
then for any generating set $X$ of $G$ there exists a finite presentation
$G=\la X|R\ra$. Moreover, if we already have some presentation $\la X|R_0\ra$
for $G$, then one can assume that $R$ is a finite subset of $R_0$ (do you see why?)

Suppose now that $G_p$ is finitely presented. Then by the previous paragraph
and HW\# 2.4(c) there exists $N\in\dbN$ such that $G_p$ admits the following (finite) presentation:
$$\la x,y \mid y^p=1, [y,y^x]=1, [y, y^{x^2}]=1, [y, y^{x^{N-1}}]=1,\ldots\ra \eqno(***)$$ 
and such that the relation $[y, y^{x^{i}}]=1$ holds in $G_p$ for all $i\in\dbZ_{\geq 0}$.

Now prove that this is impossible as follows. Show that for sufficiently large $M$
(depending on $N$) the symmetric group $S_M$ contains 2 elements $t$ and $s$
such that $t^p=1$ and $[t,t^{s^i}]=1$ for all $1\leq i\leq N-1$, but
$[t,t^{s^N}]\neq 1$. Then apply von Dyck's theorem to get a contradiction.
\skv

{\bf 3.} Recall from class that $\Aut^+(F_n)$ is the preimage of $SL_n(\dbZ)$
under the natural ``abelianization'' map $\pi: \Aut(F_n)\to GL_n(\dbZ)$
(which is surjective, as proved in Lecture~8) and thus 
$[\Aut(F_n):\Aut^+(F_n)]=2$. The goal of this problem is to prove that $\Aut^+(F_n)$ is generated by the elements $R_{ij}$ and $L_{ij}$
(recall that $R_{ij}$ sends $x_i$ to $x_i x_j$ and fixes all other $x_k$
and $L_{ij}$ sends $x_i$ to $x_j x_i$ and fixes all other $x_k$).

Define $H=\la R_{ij},L_{ij}\ra$. Then $H\subseteq \Aut^+(F_n)$, and to prove the equality it suffices to show that $[\Aut(F_n):H]=2$.
\begin{itemize}
\item[(a)] Recall that $\Aut(F_n)$ is generated by the elements $R_{ij}$, $L_{ij}$,
$I_i$ and $P_{\sigma}$, with $\sigma\in S_n$, where $I_i$ inverts $x_i$ and fixes all other $x_k$ and $P_{\sigma}$ sends $x_k$ to $x_{\sigma(k)}$ for all $k$.
Use this fact to prove that $H$ is normal in $\Aut(F_n)$.

\item[(b)] For any $1\leq i\neq j\leq n$ let $Q_{ij}$ be the element
of $\Aut(F_n)$ given by $x_i\mapsto x_j$, $x_j\mapsto x_i^{-1}$ and $x_k\to x_k$
for $k\neq i,j$. Prove by direct computation that $Q_{ij}\in \Aut^+(F_n)$.

\item[(c)] Given $g\in \Aut(F_n)$, let $\overline g$ denote the image of $g$
in $\Aut(F_n)/H$. Use (b) to show that $\overline{P_{(ij)}}=\overline{I_i}$
for any $i\neq j$ (here $(ij)$ is the transposition swapping $i$ and $j$). Deduce from this that $|\Aut(F_n)/H|=2$ and thus $[\Aut(F_n):H]=2$.
\end{itemize}


{\bf 4.} Recall from class that $\IA_n$ (also called the Torelli subgroup of $\Aut(F_n)$) is the kernel of $\pi: \Aut(F_n)\to GL_n(\dbZ)$.

\begin{itemize}
\item[(a)] Prove that $\IA_n$ contains $\Inn(F_n)$, the subgroup of inner automorphsisms of $F_n$.
\item[(b)] Magnus (1935) proved that $\IA_n$ is generated by the elements
$K_{ij}$ with $1\leq i\neq j\leq n$ and $K_{ijm}$ with $i,j,m$ distinct
where $K_{ij}$ sends $x_i$ to $x_j^{-1}x_i x_j$ and fixes all other $x_k$
and $K_{ijm}$ sends $x_i$ to $x_i [x_j,x_m]$ and fixes all other $x_k$.
Verify that the elements $K_{ij}$ and $K_{ijm}$ indeed lie in $\IA_n$.
\item[(c)] Use (b) to show that $\IA_2=\Inn(F_2)$. We will discuss a different
proof of this result later in the course.
\end{itemize}

{\bf 5.} The proof of the Nielsen reduction theorem (Theorem~7.5) yields a general algorithm
which, given an $n$-tuple of elements of $F_n$, decides whether these elements
generate $F_n$ or not. In the case $n=2$ one can answer this question almost
immediately using the following commutator test.
\begin{Theorem}[Commutator test] Let $\{x,y\}$ be a free generating set of $F_2$,
and take any $u,v\in F_2$. Then $u$ and $v$ generate $F_2$ if and only if
the commutator $[u,v]=u^{-1}v^{-1}uv$ is conjugate (in $F_2$) to
$[x,y]$ or $[y,x]=[x,y]^{-1}$.
\end{Theorem}
\begin{itemize}
\item[(a)] Prove the `only if' ($\Rightarrow$) part of the commutator test.
{\bf Hint:} Use Nielsen moves.
\item[(b)] Now think of how you would prove the `if' part. I do not know
of a nice short algebraic argument. One possible proof is outlined in the following
paper of Shpilrain (see Proposition~2.4):

\url{https://shpilrain.ccny.cuny.edu/test1.pdf}

\end{itemize} 



\end{document}



