\documentclass[12pt]{article}

\usepackage{amsmath}
\usepackage{amssymb}
\usepackage{amsthm}
%\usepackage{psfig}

\begin{document}
\baselineskip=15pt
\textheight=9in
\parindent=0pt
\def\sk {\hskip .5cm}
\def\skv {\vskip .07cm}
\def\cos {\mbox{cos}}
\def\sin {\mbox{sin}}
\def\tan {\mbox{tan}}
\def\intl{\int\limits}
\def\lm{\lim\limits}
\newcommand{\frc}{\displaystyle\frac}
\def\xbf{{\mathbf x}}
\def\fbf{{\mathbf f}}
\def\gbf{{\mathbf g}}

\def\Ker{{\rm Ker\,}}
\def\Gal{{\rm Gal\,}}
\def\phi{\varphi}

\def\dbA{{\mathbb A}}
\def\dbB{{\mathbb B}}
\def\dbC{{\mathbb C}}
\def\dbD{{\mathbb D}}
\def\dbE{{\mathbb E}}
\def\dbF{{\mathbb F}}
\def\dbG{{\mathbb G}}
\def\dbH{{\mathbb H}}
\def\dbI{{\mathbb I}}
\def\dbJ{{\mathbb J}}
\def\dbK{{\mathbb K}}
\def\dbL{{\mathbb L}}
\def\dbM{{\mathbb M}}
\def\dbN{{\mathbb N}}
\def\dbO{{\mathbb O}}
\def\dbP{{\mathbb P}}
\def\dbQ{{\mathbb Q}}
\def\dbR{{\mathbb R}}
\def\dbS{{\mathbb S}}
\def\dbT{{\mathbb T}}
\def\dbU{{\mathbb U}}
\def\dbV{{\mathbb V}}
\def\dbW{{\mathbb W}}
\def\dbX{{\mathbb X}}
\def\dbY{{\mathbb Y}}
\def\dbZ{{\mathbb Z}}

\def\Aut{{\rm Aut}}
\def\deg{{\rm deg}}

\def\la{{\langle}}
\def\ra{{\rangle}}

\bf\centerline{Homework Assignment \# 11. }\rm
\skv
\skv
{\bf Plan for the next week:} Group cohomology (Chapter 17)
\skv
\skv
\bf\centerline{Problems, to be submitted by Thu, April 25th. }\rm
\skv
\skv
\skv
{\bf 1.} This is a continuation of Problem~1 from Midterm\#2.
Let $p$ be a prime, with $p\equiv 3\mod 4$, $\omega=e^{2\pi i/p}$, $K=\dbQ(\omega)$ and $L$ the unique subfield of $K$ with $[L:\dbQ]=2$. 
Let $S$ be the set of elements of $(\dbZ/p\dbZ)^{\times}$ 
representable as squares and $T$ the set of elements of 
$(\dbZ/p\dbZ)^{\times}$ not representable as squares.
\begin{itemize}
\item[(a)] Prove that any $\alpha\in K$ can be uniquely represented
as $\alpha=\sum_{s\in S}b_s \omega^s+\sum_{t\in T}c_t \omega^t$,
with $b_s,c_t\in\dbQ$.
\item[(b)] Let $\alpha\in K$. Prove that $\alpha\in L$ if and only
if in the above decomposition of $\alpha$ all $b_s$ are the same
and all $c_t$ are the same.
\item[(c)] Let $\zeta=\sum_{s\in S}\omega^s$, $\eta=\zeta\overline\zeta$,
and write $\eta=\sum_{s\in S}b_s \omega^s+\sum_{t\in T}c_t \omega^t$ as in (a).
Prove that 
\begin{itemize}
\item[(i)] there exists $d\in \dbQ$ such that $b_s=c_t=d$
for all $s$ and $t$ and
\item[(ii)] $\sum_{s\in S}b_s+\sum_{t\in T}c_t=(p-1)^2/4-p\cdot (p-1)/2=-(p-1)(p+1)/4$
\end{itemize}
\item[(d)] Use (c) to prove that $\eta=(p+1)/4$ and deduce that 
$L=\dbQ(\sqrt{-p})$.
\end{itemize}
\skv
{\bf Problem 2:} 
\begin{itemize}
\item[(a)] DF, Problem~3 on page 403
\item[(b)] Let $R$ be a PID. Prove that every finitely generated projective module over $R$ is free (the assertion is true for infinitely generated projective modules as well, but the general case probably cannot be solved using just the results we have discussed)
\end{itemize}
\skv
{\bf Problem 3:} Let $F$ be a field, $n\geq 2$ an integer and $R=Mat_n(F)$. Let $M=F^n$, and consider $M$ as a left $R$-module (with respect to left multiplication). Prove that $M$ is projective, but not free. 
\skv 
{\bf Problem 4:} Let $R=\dbZ[\sqrt{-5}]$ be a field and $I=(2,1+\sqrt{-5})$, and define $f:R^2\to I$
by $f(x,y)=2x+(1+\sqrt{-5})y$ (note that $f$ is surjective).
\begin{itemize}
\item[(a)] Show that the exact sequence $0\to\Ker f\to R^2\to I\to 0$ splits by explicitly constructing a splitting. Deduce that $I$ is a projective $R$-module. {\bf Hint:} there is a splitting of the form $g(x)=(ax,bx)$ for suitable
$a,b\in \mathrm{Frac}(R)=\dbQ[\sqrt{-5}]$.
\item[(b)] Prove that $I$ is not a principal ideal and deduce that $I$ is not a free $R$-module.
\end{itemize}
\skv
{\bf Problem 5:} DF, Problem~4 on page 403
\skv
{\bf Problem 6:} Let $R$ be a commutative ring with $1$. An $R$-module $M$ is called {\it torsion-free} if for any $r\in R$
and $m\in M$, the equality $rm=0$ implies that $m=0$ or $r=0$ or $r$ is a zero divisor. Prove that any flat $R$-module is torsion-free. {\bf Hint:} argue by contrapositive.
\skv

\end{document} 