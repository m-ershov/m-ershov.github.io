\documentclass[12pt]{article}

\usepackage{amsmath}
\usepackage{amssymb}
\usepackage{amsthm}
%\usepackage{psfig}

\begin{document}
\baselineskip=15pt
\textheight=8.4in
\parindent=0pt
\def\sk {\hskip .5cm}
\def\skv {\vskip .12cm}
\def\cos {\mbox{cos}}
\def\sin {\mbox{sin}}
\def\tan {\mbox{tan}}
\def\intl{\int\limits}
\def\lm{\lim\limits}
\newcommand{\frc}{\displaystyle\frac}
\def\xbf{{\mathbf x}}
\def\fbf{{\mathbf f}}
\def\gbf{{\mathbf g}}

\def\Ker{{\rm Ker\,}}
\def\phi{\varphi}

\def\dbA{{\mathbb A}}
\def\dbB{{\mathbb B}}
\def\dbC{{\mathbb C}}
\def\dbD{{\mathbb D}}
\def\dbE{{\mathbb E}}
\def\dbF{{\mathbb F}}
\def\dbG{{\mathbb G}}
\def\dbH{{\mathbb H}}
\def\dbI{{\mathbb I}}
\def\dbJ{{\mathbb J}}
\def\dbK{{\mathbb K}}
\def\dbL{{\mathbb L}}
\def\dbM{{\mathbb M}}
\def\dbN{{\mathbb N}}
\def\dbO{{\mathbb O}}
\def\dbP{{\mathbb P}}
\def\dbQ{{\mathbb Q}}
\def\dbR{{\mathbb R}}
\def\dbS{{\mathbb S}}
\def\dbT{{\mathbb T}}
\def\dbU{{\mathbb U}}
\def\dbV{{\mathbb V}}
\def\dbW{{\mathbb W}}
\def\dbX{{\mathbb X}}
\def\dbY{{\mathbb Y}}
\def\dbZ{{\mathbb Z}}

\def\Aut{{\rm Aut}}

\def\la{{\langle}}
\def\ra{{\rangle}}

\bf\centerline{Homework Assignment \# 1. }\rm
\vskip .2cm
{\bf Plan for next week:} Tensor products of modules and algebras.
Reading: 10.4 in DF and Lectures 3, 4, 5 from the webpage.
\vskip .1cm

\bf\centerline{Problems, to be submitted by Fri, January 25th. }\rm
\vskip .1cm

{\bf Convention:} All rings below are assumed to have $1$, and all modules are left modules.
\skv
\vskip .1cm
\bf{Problem 1: }\rm Let $M$ be an $R$-module for some ring $R$ (not necessarily commutative).
\begin{itemize}
\item[(a)] For a subset $N$ of $M$ the {\it annihilator of $N$ in $R$} is defined to be the set
$Ann_R(N):=\{r\in R : rn=0 \mbox{ for any }n\in N\}$. Prove that $Ann_R(N)$ is a left
ideal of $R$.
\item[(b)] Prove that if $N$ is a submodule of $M$, then $Ann_R(N)$ is an ideal of $R$ (that is, a two-sided ideal).
\item[(c)] For a subset $I$ of $R$ the {\it annihilator of $I$ in $M$} is defined to be the
set $Ann_M(I):=\{m\in M : xm=0 \mbox{ for any }x\in I\}$. Find a natural condition on $I$
which guarantees that $Ann_M(I)$ is a submodule of $M$.
\end{itemize}
\bf{Problem 2: }\rm Let $R$ be a ring and let $M$ be an $R$-module.
\begin{itemize}
\item[(a)] Prove that for any $m\in M$, the map $x\mapsto xm$ from $R$ to $M$ is a homomorphism of $R$-modules
(recall that $R$ is an $R$-module with the left multiplication action).
\item[(b)] Assume that $R$ is commutative, and let $M$ be an $R$-module. Prove that $Hom_R(R,M)\cong M$ as $R$-modules. {\bf Note:} For the definition and justification
of the $R$-module structure on the set $Hom_R(M,N)$ (where $R$ is commutative and $M$ and $N$ are $R$-modules) see Proposition~2 on page 346 in DF. {\bf Hint:}
An element of $Hom_R(R,M)$ is uniquely determined by where it maps $1$.
\end{itemize}
\bf{Problem 3: }\rm In Lecture 1 (see also online Lecture 1) we obtained a simple characterization of $R$-modules for $R=\dbZ$ and $R=F[x]$, with $F$ a field. 
\begin{itemize}
\item[(a)] Find a similar characterization of $R$-modules for $R=\dbZ/n\dbZ$; 
\item[(b)] Let $R=S[x]$ for some commutative ring $S$ with $1$. Prove that there is a natural correspondence
between $R$-modules and pairs $(M,T)$ where $M$ is an $S$-module and $T:M\to M$ is an $S$-linear map.
\item[(c)] Let $R=F[x,y]$, with $F$ a field. Use (b) to construct a natural correspondence between
$R$-modules and triples $(V,A,B)$ where $V$ is an $F$-vector space and $A,B:V\to V$ are commuting linear transformations.
{\bf Note:} This can be done without (b), but (b) yields a complete proof which does not involve long
and tedious verifications.
\end{itemize}
\skv
\bf{Problem 4 (practice): }\rm Let $G$ be a group and $\dbZ[G]$ its integral group ring
(see online Lecture 1 or DF, \S~7.2 for definition). Let $M$ be an abelian group. Show that there is a natural
bijection between $\dbZ[G]$-module structures on $M$ and actions of $G$
on $M$ by group automorphisms (that is, actions of $G$ on $M$ such that for any
$g\in G$ the map $m\mapsto gm$ is an automorphism of the abelian group $M$).
\skv

\bf{Problem 5: }\rm An $R$-module $M$ is called {\it simple (or irreducible)} if
$M$ has no submodules besides $\{0\}$ and $M$. An $R$-module $M$ is called
{\it indecomposable} if $M$ is not isomorphic to $N\oplus P$ for nonzero
$R$-modules $N$ and $P$.
\begin{itemize}
\item[(a)] Prove that every simple module is indecomposable
\item[(b)] Describe all simple $\dbZ$-modules and all finitely generated
indecomposable $\dbZ$-modules. Deduce that an indecomposable module need not
be simple.
\end{itemize}

\bf{Problem 6: }\rm  An $R$-module $M$ is called {\it cyclic} if $M$ is generated
(as an $R$-module) by one element.
\begin{itemize}
\item[(a)] Prove that cyclic $R$-modules are precisely the ones which are isomorphic to $R/I$
for some left ideal $I$ of $R$.
\item[(b)] Prove that every simple module is cyclic. Then show that simple
$R$-modules are precisely the ones which are isomorphic to $R/I$
for some maximal left ideal $I$ of $R$.
\end{itemize}

{\bf Problem 7.} Let $R$ be a commutative domain, and let $I$ be a non-principal ideal of $R$. 
Prove that $I$, considered as an $R$-module (with left-multiplication action) is indecomposable but not cyclic. 
{\bf Hint:} One way to prove that $I$ is indecomposable is to show that any two elements of $I$ 
are linearly dependent over $R$. {\bf Note:} As we will prove in a couple of weeks, if $R$ is a principal
ideal domain, every finitely generated indecomposable module is cyclic.
\vskip .12cm
{\bf Problem 8.} Prove Schur's lemma [DF, problem 11, p.356].
\end{document}
