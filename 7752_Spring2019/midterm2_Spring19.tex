\documentclass[12pt]{article}

\usepackage{amsmath}
\usepackage{amssymb}
\usepackage{amsthm}
%\usepackage{psfig}

\begin{document}
\baselineskip=16pt
\textheight=8.6in
\parindent=0pt
\def\sk {\hskip .5cm}
\def\skv {\vskip .12cm}
\def\cos {\mbox{cos}}
\def\sin {\mbox{sin}}
\def\tan {\mbox{tan}}
\def\intl{\int\limits}
\def\lm{\lim\limits}
\newcommand{\frc}{\displaystyle\frac}
\def\xbf{{\mathbf x}}
\def\fbf{{\mathbf f}}
\def\gbf{{\mathbf g}}

\def\dbA{{\mathbb A}}
\def\dbB{{\mathbb B}}
\def\dbC{{\mathbb C}}
\def\dbD{{\mathbb D}}
\def\dbE{{\mathbb E}}
\def\dbF{{\mathbb F}}
\def\dbG{{\mathbb G}}
\def\dbH{{\mathbb H}}
\def\dbI{{\mathbb I}}
\def\dbJ{{\mathbb J}}
\def\dbK{{\mathbb K}}
\def\dbL{{\mathbb L}}
\def\dbM{{\mathbb M}}
\def\dbN{{\mathbb N}}
\def\dbO{{\mathbb O}}
\def\dbP{{\mathbb P}}
\def\dbQ{{\mathbb Q}}
\def\dbR{{\mathbb R}}
\def\dbS{{\mathbb S}}
\def\dbT{{\mathbb T}}
\def\dbU{{\mathbb U}}
\def\dbV{{\mathbb V}}
\def\dbW{{\mathbb W}}
\def\dbX{{\mathbb X}}
\def\dbY{{\mathbb Y}}
\def\dbZ{{\mathbb Z}}

\def\la{{\langle}}
\def\ra{{\rangle}}
\def\phi{{\varphi}}

\def\Aut{{\rm Aut}}
\def\Gal{{\rm Gal}}
\def\End{{\rm End}}
\def\Inn{{\rm Inn}}

\bf
\begin{center}Algebra-II, Spring 2019. Midterm \#2. \\Due Friday, April 12th, by 1pm
\end{center}
\rm
\vskip .1cm


{\bf Directions: } Each problem is worth 10 points, and all 6 problems
will be counted. Provide complete arguments
(do not skip steps). State clearly any result you are referring to. Partial credit for
incorrect solutions, containing steps in the right direction, may be given.
\vskip .1cm

{\bf Rules: } You are not allowed to discuss midterm problems with each other.
You may ask me any questions about the problems (e.g. if the formulation is unclear),
but as a rule I will only provide minor hints. You may freely use your class notes,
previous homework assignments and the book by Dummit and Foote. The use of other books
is allowed, but not encouraged. If you happen to run across a problem
very similar or identical to one on the midterm which is solved in another book,
do not consult that solution. The use of any online resources (except the class webpage)
is absolutely prohibited.


\skv
{\bf Problem 1:} Let $F$ be an algebraically closed field, let $A\in Mat_n (F)$
for some $n\in\dbN$, and let $C$ be the centralizer of $A$ in $Mat_n (F)$.
Prove that $$dim_F(C)\geq n.$$ {\bf Hint:} First assume that $A$ is in Jordan
canonical form and has just one Jordan block; then consider the case when
$A$ is an arbitrary matrix in Jordan canonical form, and finally prove
the statement for general $A$.
\skv
{\bf Problem 2:} DF, Problem~6 on page 582
\skv
{\bf Problem 3:} Let $K/F$ be a finite Galois extension and $G=\Gal(K/F)$.
\begin{itemize}
\item[(a)] Assume  that $G$ is a simple group, let $\alpha\in K\setminus F$
and $\mu_{\alpha,F}(x)$ the minimal polynomial of $\alpha$ over $F$.
Prove that $K$ is a splitting field for $\mu_{\alpha,F}(x)$.

\item[(b)] Let $n=[K:F]$, and fix integers $m$ and $l$ with $ml=n$.
Find a condition on the \underline{subfield lattice of $K/F$} which is equivalent
to the following: $G$ can be written as a semidirect product $G=A\rtimes B$
for some subgroups $A$ and $B$ where $|A|=m$ and $|B|=l$.
\end{itemize}
\newpage
{\bf Problem 4:} Let $F$ be a field.
\begin{itemize}
\item[(a)] Let $f(x)\in F[x]$ be a nonzero polynomial and $K/F$ a field extension.
Prove that $$F[x]/(f(x))\otimes_F K\cong K[x]/(f(x))$$ as $F$-algebras.
\item[(b)] Let $L/F$ be a finite separable extension. Prove that
there exists a finite extension $K/F$ such that 
$L\otimes_F K\cong \underbrace{K\times\ldots\times K}_{n \mbox{ times }}$
for some $n$.
\end{itemize}
\skv
{\bf Problem 5:} \rm Let $S=\{n_1,\ldots, n_k\}$ be a finite set of positive integers $\geq 2$
and let $K=\dbQ(\sqrt{n_1},\ldots, \sqrt{n_k})$. You are NOT allowed to refer to HW\#6.3.
\begin{itemize}
\item[(a)] Prove that $K/\dbQ$ is a Galois extension and $Gal(K/\dbQ)\cong \dbZ_2^m$ for some $m\leq k$
\item[(b)] Now assume that $n_1,\ldots, n_k$ are pairwise coprime. Prove that $K$ contains at least $2^k$
distinct subfields $L$ with $[L:\dbQ]=2$. 
\item[(c)] Keep the extra assumption from (b). Use (b) to prove that $[K:\dbQ]=2^k$.  
\end{itemize}
\skv
{\bf Problem 6:} Let $p$ be an odd prime, $\omega=e^{2\pi i/p}$, $K=\dbQ(\omega)$
and $L$ the unique subfield of $K$ with $[L:\dbQ]=2$. As we proved in class, if we let
$m$ be a generator of $(\dbZ/p\dbZ)^{\times}$ and $\zeta=\sum_{i=0}^{(p-3)/2} \omega^{m^{2i}}$,
then $L=\dbQ(\zeta)$.
\begin{itemize}
\item[(a)] Prove by direct computation that if $p=5$, then $L=\dbQ(\sqrt{5})$.
\item[(b)] Let $S$ be the set of all elements of $(\dbZ/p\dbZ)^{\times}$ representable 
as squares. Prove that $\zeta=\sum_{s\in S}\omega^s$.
\item[(c)] Prove that $-1\in S$ if and only if $p\equiv 1\mod 4$
\item[(d)] Prove that $\overline\zeta=\zeta$ if $p\equiv 1\mod 4$ and $\overline\zeta=-1-\zeta$
if $p\equiv 3\mod 4$. Deduce that $L\subset \dbR$ if and only if $p\equiv 1\mod 4$.
\item[(e)] Let $M$ be the unique subfield of $K$ with $[K:M]=2$. As proved in class,
$M\subset \dbR$. Now prove that $L\subset \dbR$ if and only if $p\equiv 1\mod 4$ just 
by using this fact and Galois correspondence (do not use an explicit description of $L$).
{\bf Hint:} What is the relationship between the subgroups of $\Gal(K/\dbQ)$ corresponding
to $L$ and $M$ depending on $p$ mod $4$?
\end{itemize} 

\end{document}
