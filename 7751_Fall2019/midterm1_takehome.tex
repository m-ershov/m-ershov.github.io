\documentclass[12pt]{amsart}

\usepackage{amsmath}
\usepackage{amssymb}
\usepackage{amsthm}
%\usepackage{psfig}

\begin{document}
\baselineskip=16pt
\textheight=9.6in
\parindent=0pt
\def\sk {\hskip .5cm}
\def\skv {\vskip .12cm}
\def\cos {\mbox{cos}}
\def\sin {\mbox{sin}}
\def\tan {\mbox{tan}}
\def\intl{\int\limits}
\def\lm{\lim\limits}
\newcommand{\frc}{\displaystyle\frac}
\def\xbf{{\mathbf x}}
\def\fbf{{\mathbf f}}
\def\gbf{{\mathbf g}}

\def\dbA{{\mathbb A}}
\def\dbB{{\mathbb B}}
\def\dbC{{\mathbb C}}
\def\dbD{{\mathbb D}}
\def\dbE{{\mathbb E}}
\def\dbF{{\mathbb F}}
\def\dbG{{\mathbb G}}
\def\dbH{{\mathbb H}}
\def\dbI{{\mathbb I}}
\def\dbJ{{\mathbb J}}
\def\dbK{{\mathbb K}}
\def\dbL{{\mathbb L}}
\def\dbM{{\mathbb M}}
\def\dbN{{\mathbb N}}
\def\dbO{{\mathbb O}}
\def\dbP{{\mathbb P}}
\def\dbQ{{\mathbb Q}}
\def\dbR{{\mathbb R}}
\def\dbS{{\mathbb S}}
\def\dbT{{\mathbb T}}
\def\dbU{{\mathbb U}}
\def\dbV{{\mathbb V}}
\def\dbW{{\mathbb W}}
\def\dbX{{\mathbb X}}
\def\dbY{{\mathbb Y}}
\def\dbZ{{\mathbb Z}}

\def\la{{\langle}}
\def\ra{{\rangle}}

\def\Aut{{\rm Aut}}
\def\End{{\rm End}}
\def\Inn{{\rm Inn}}
\def\Im{{\rm Im\,}}
\def\phi{{\varphi}}

\bf\centerline{Algebra-I, Fall 2019. Midterm \#1, take-home part}\rm
\skv
\bf\centerline{due Thursday Oct 3rd, in class}\rm
\vskip .3cm
{\bf Directions: } Provide complete arguments
(do not skip steps). State clearly any result you are referring to. Partial credit for
incorrect solutions, containing steps in the right direction, may be given.
\vskip .1cm

{\bf Rules: } You are not allowed to discuss midterm problems with each other.
You may ask me any questions about the problems (e.g. if the formulation is unclear),
but as a rule I will only provide minor hints. You may freely use your class notes,
previous homework assignments and the book by Dummit and Foote. You may also use materials posted on any of my course pages (including the Fall 18 page for Algebra-I). The use of other books or other online resources is prohibited.

\newpage

\skv
{\bf 1.} The main goal of this problem is to prove the isomorphism $$PSL_2(\dbZ_5)\cong A_5$$
\skv
First some notations and terminology. Let $V$ be a finite-dimensional space over a field $F$. The projective space $\dbP(V)$ is defined to be the set of all $1$-dimensional subspaces of $V$; equivalently $\dbP(V)=(V\setminus\{0\})/\sim$ where $v\sim w$ for nonzero vectors $v$ and $w$
$\iff$ $w=\lambda v$ for some $\lambda\in F\setminus\{0\}$. For each $n\in\dbN$ we set $\dbP^n(F)=\dbP(F^{n+1})$ (where $F^{m}$ is the standard $m$-dimensional vector space over $F$). Also recall that $PSL_n(F)=SL_n(F)/Z_n(F)$ where $Z_n(F)$ is the subgroup of scalar matrices in $SL_n(F)$.
\skv
\begin{itemize}
\item[(a)] Prove that for any field $F$ and any $n\in\dbN$ there is a natural faithful action of $PSL_n(F)$ on $\dbP^{n-1}(F)$. Do not use any results about simplicity of $PSL_n(F)$. Deduce that there is a natural injective homomorphism $\phi:PSL_2(\dbZ_5)\to S_6$.
\item[(b)] Use (a) and the idea from HW\#4.3 to prove that $\Im\phi\cong A_5$.
\end{itemize}
{\bf Hint:} Use may use the fact that $PSL_n(\dbZ_p)$ is always generated by elementary matrices $E_{ij}$, $i\neq j$ (where $E_{ij}$ is the matrix which has $1$'s on the diagonal and in the position $(i,j)$ and $0$'s everywhere else). It may not be immediately clear how this result helps for this problem. 
\skv
{\bf 2.} A subgroup $H$ of a group $G$ is called {\it characteristic} if $H$ is invariant under all automorphisms of $G$, that is,
$\phi(H)=H$ for all $\phi\in \Aut(G)$. Prove that for a finite group $G$ the following conditions are equivalent:
\begin{itemize}
\item[(i)] $G\cong U^n$ for some $n\in\dbN$ and simple group $U$
\item[(ii)] $G$ does not have any proper non-trivial characteristic subgroups.
\end{itemize}
{\bf Hint:} When proving the implication ``(i)$\Rightarrow$(ii)'' consider separately the cases $U$ abelian and $U$ non-abelian.
\end{document}
