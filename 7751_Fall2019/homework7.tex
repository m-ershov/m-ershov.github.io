\documentclass[12pt]{amsart}

\usepackage{amsmath}
\usepackage{amssymb}
\usepackage{amsthm}
\usepackage{amscd}
\usepackage[all]{xy}
%\usepackage{psfig}

\begin{document}
\baselineskip=16pt
\textheight=8.5in
\textwidth=6in
\parindent=0pt
\def\sk {\hskip .5cm}
\def\skv {\vskip .12cm}
\def\cos {\mbox{cos}}
\def\sin {\mbox{sin}}
\def\tan {\mbox{tan}}
\def\intl{\int\limits}
\def\lm{\lim\limits}
\newcommand{\frc}{\displaystyle\frac}
\def\xbf{{\mathbf x}}
\def\fbf{{\mathbf f}}
\def\gbf{{\mathbf g}}

\def\dbA{{\mathbb A}}
\def\dbB{{\mathbb B}}
\def\dbC{{\mathbb C}}
\def\dbD{{\mathbb D}}
\def\dbE{{\mathbb E}}
\def\dbF{{\mathbb F}}
\def\dbG{{\mathbb G}}
\def\dbH{{\mathbb H}}
\def\dbI{{\mathbb I}}
\def\dbJ{{\mathbb J}}
\def\dbK{{\mathbb K}}
\def\dbL{{\mathbb L}}
\def\dbM{{\mathbb M}}
\def\dbN{{\mathbb N}}
\def\dbO{{\mathbb O}}
\def\dbP{{\mathbb P}}
\def\dbQ{{\mathbb Q}}
\def\dbR{{\mathbb R}}
\def\dbS{{\mathbb S}}
\def\dbT{{\mathbb T}}
\def\dbU{{\mathbb U}}
\def\dbV{{\mathbb V}}
\def\dbW{{\mathbb W}}
\def\dbX{{\mathbb X}}
\def\dbY{{\mathbb Y}}
\def\dbZ{{\mathbb Z}}

\def\la{{\langle}}
\def\ra{{\rangle}}

\def\Ker{{\rm Ker\,}}
\def\Aut{{\rm Aut}}
\def\Inn{{\rm Inn}}
\def\SL{{\rm SL}}
\def\eps{{\varepsilon}}

\bf\centerline{Homework \#7. }\rm
\vskip .1cm
{\bf Plan for next week:} We will start ring theory. I plan to give a brief survey of Chapter 7, with emphasis on the following topics: maximal ideals (7.4), rings of fractions and localizations (7.5) and Chinese Remainder Theorem (7.6).

\vskip .1cm
\centerline{\bf Problems, to be submitted by Thursday, October 24th}
\skv
\skv
{\bf 1.} Let $G=\SL_2(\dbZ)$, and let $X=\begin{pmatrix} 1&2\\ 0&1 \end{pmatrix}$ and
$Y=\begin{pmatrix} 1&0\\ 2&1 \end{pmatrix}$, and consider the natural action of $G$ on $\mathbb Z^2$.
Find subsets $A$ and $B$ of $\dbZ^2$ satisfying the hypotheses of the Ping-Pong Lemma (with respect to this action).
This implies that $\la X,Y\ra$ is free of rank $2$.
\skv
{\bf 2.} Let $G$ be a group, $S$ a subset of $G$ and $H=\la S\ra$ the subgroup generated by $S$. One says that $H$ is {\it freely generated by $S$} if the evaluation homomorphism $\eps:F(S)\to H$ is an isomorphism.

\sk Now let $G=F(a,b)$, the free group with $2$ generators $a$ and $b$. For each $n\in\dbN$ let $s_n=a^n b a^{-n}\in G$, and let
$S=\{s_n\}_{n=1}^{\infty}$. Prove that $\la S\ra$ is free of countable rank (this gives another example of an infinitely generated subgroup of a finitely generated group). Do not use the theorem that subgroups of free groups are free. Give a detailed argument (what I have in mind is a purely combinatorial argument).
\skv
{\bf 3.} Given a group $G$, the quotient $G/[G,G]$ is called the {\it abelianization} of $G$ and denoted by $G^{ab}$ (basic properties of the commutator subgroup imply that $G^{ab}$ is the largest abelian quotient of $G$).
\begin{itemize}
\item[(a)] Let $n\in\dbN$. Prove that $F_n^{ab}\cong \dbZ^n$.
\item[(b)] Use (a) to prove that if $X$ and $Y$ are finite sets such that $F(X)\cong F(Y)$, then $|X|=|Y|$ (the corresponding result for arbitrary $X$ and $Y$ can be deduced similarly from a suitable generalization of (a)).
\end{itemize}

{\bf 4.} Let $X$ be a set and $F(X)$ the free group on $X$.
Given $f\in F(X)$, the {\it length of $f$}, denoted by $l(f)$ is defined to be
the length of the unique reduced word in $X\cup X^{-1}$ representing $f$. Equivalently,
$l(f)$ is the smallest $n$ such that $f=x_1^{\eps_1}\ldots x_n^{\eps_n}$
for some $x_i\in X$ and $\eps_i\in\{\pm 1\}$.

\begin{itemize}
\item[(a)] Prove that for any $f\in F(X)$ there exist integers $a,b\in\dbZ_{\geq 0}$
(depending on $f$) such that $l(f^n)=na+2b$ for any $n\in\dbN$.
Describe explicitly (i.e. give an algorithm) how to compute $a$ and $b$
for a given $f$.

\item[(b)] Use (a) to show that free groups are torsion-free.
\end{itemize}
{\bf 5.} 
\begin{itemize}
\item[(a)] Explain why for any $n\in\dbN$ there are only finitely many isomorphism
classes of groups of order $n$.

\item[(b)] Let $G$ be a finitely generated group and $H$ a finite group. Prove that
there are only finitely many homomorphisms from $G$ to $H$. 

\item[(c)] Let $G$ be a finitely generated group. Prove that for any $n\in\dbN$
there are only finitely many normal subgroups of index $n$ in $G$. Then deduce that
$G$ has only finitely many subgroups of index $n$ (use the small index lemma).
\end{itemize}
{\bf 6.} Let $p$ and $q$ be primes with $p<q$ and $q\equiv 1\mod p$, and let $G$ be a non-abelian group or order $pq$. Recall that such $G$ is unique up to isomorphism. Prove that $G$ has a presentation $\la x,y\mid x^p=1, y^q=1, xyx^{-1}=y^a \ra$ where $a$ is coprime to $q$ and the order of $\bar a$ in $\dbZ_q$ is equal to $p$.
\skv
{\bf Hint:} Let $\widehat G=\la x,y\mid x^p=1, y^q=1, xyx^{-1}=y^a \ra$. By definition $\widehat G$ is the quotient of $F(x,y)$ by the normal closure of the set $\{x^p, y^q, xyx^{-1}y^{-a}\}$. To prove the statement first show that $G$ has elements $X$ and $Y$ satisfying the above 3 relations; then show that there is a surjective homomorphism $\phi:\widehat G\to G$ such that $\phi(x)=X$
and $\phi(y)=Y$. Finally, prove that $|\widehat G|\leq pq$ and deduce that $\phi$ is an isomorphism.



\end{document}
