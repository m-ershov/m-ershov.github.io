\documentclass[12pt]{amsart}

\usepackage{amsmath}
\usepackage{amssymb}
\usepackage{amsthm}
%\usepackage{psfig}

\begin{document}
\baselineskip=16pt
%\textheight=9.6in
%\parindent=0pt
\def\sk {\hskip .5cm}
\def\skv {\vskip .12cm}
\def\cos {\mbox{cos}}
\def\sin {\mbox{sin}}
\def\tan {\mbox{tan}}
\def\intl{\int\limits}
\def\lm{\lim\limits}
\newcommand{\frc}{\displaystyle\frac}
\def\xbf{{\mathbf x}}
\def\fbf{{\mathbf f}}
\def\gbf{{\mathbf g}}

\def\dbA{{\mathbb A}}
\def\dbB{{\mathbb B}}
\def\dbC{{\mathbb C}}
\def\dbD{{\mathbb D}}
\def\dbE{{\mathbb E}}
\def\dbF{{\mathbb F}}
\def\dbG{{\mathbb G}}
\def\dbH{{\mathbb H}}
\def\dbI{{\mathbb I}}
\def\dbJ{{\mathbb J}}
\def\dbK{{\mathbb K}}
\def\dbL{{\mathbb L}}
\def\dbM{{\mathbb M}}
\def\dbN{{\mathbb N}}
\def\dbO{{\mathbb O}}
\def\dbP{{\mathbb P}}
\def\dbQ{{\mathbb Q}}
\def\dbR{{\mathbb R}}
\def\dbS{{\mathbb S}}
\def\dbT{{\mathbb T}}
\def\dbU{{\mathbb U}}
\def\dbV{{\mathbb V}}
\def\dbW{{\mathbb W}}
\def\dbX{{\mathbb X}}
\def\dbY{{\mathbb Y}}
\def\dbZ{{\mathbb Z}}

\def\la{{\langle}}
\def\ra{{\rangle}}

\def\Aut{{\rm Aut}}
\def\End{{\rm End}}
\def\Inn{{\rm Inn}}
\def\Ker{{\rm Ker}}
\def\Im{{\rm Im\,}}
\def\phi{{\varphi}}

\bf\centerline{Algebra-I, Fall 2019. Midterm \#2}\rm
\skv
\bf\centerline{due Thursday Nov 14th, in class}\rm
\vskip .3cm
{\bf Directions: } Provide complete arguments
(do not skip steps). State clearly any result you are referring to. Partial credit for
incorrect solutions, containing steps in the right direction, may be given.
\vskip .1cm

{\bf Rules: } You are not allowed to discuss midterm problems with each other.
You may ask me any questions about the problems (e.g. if the formulation is unclear),
but as a rule I will only provide minor hints. You may freely use your class notes,
previous homework assignments and the book by Dummit and Foote. You may also use materials posted on any of my course pages (including the Fall 18 page for Algebra-I). The use of other books or other online resources is prohibited.

\skv
{\bf Scoring:} The best 4 out of 5 problems will count, but there will be some sort of bonus for solving all 5 essentially correctly (I will make this more precise later).

\skv
{\bf 1.} Let $G$ be a group. A subgroup $H$ of $G$ will be called {\it essential}
if $H\cap K\neq \{1\}$ for every non-trivial subgroup $K$ of $G$.
\begin{itemize}
\item[(a)] Let $p$ be a prime and $k\geq 2$. Prove that the group $\dbZ/p^k\dbZ$
has a proper essential subgroup.
\item[(b)] Assume that $H_1$ is an essential subgroup of $G_1$ and $H_2$
is an essential subgroup of $G_2$. Prove that $H_1\times H_2$ is an essential subgroup
of $G_1\times G_2$.
\item[(c)] Let $G$ be a finite abelian group. Prove that $G$ does not have a proper
essential subgroup if and only if $G$ is a direct product of groups of prime order.
\end{itemize}
\skv
{\bf 2.} Let $G$ be the group of $3\times 3$ upper unitriangular matrices over $\dbZ$ (called the {\it Heisenberg group} over $\dbZ$). 
Given distinct integers $i\neq j$, let $E_{ij}$ denote the matrix with $1$'s on the diagonal and in the position $(i,j)$ and $0$'s everywhere else. Let $x=E_{12}$, $y=E_{23}$ and $z=E_{13}$ (note that all these elements lie in $G$).
\begin{itemize}
\item[(a)] Prove that every element of $G$ can be uniquely written as $x^a y^b z^c$ with $a,b,c\in\dbZ$
\item[(b)] Prove that $G$ is nilpotent of class two, that  $[G,G]=\la z\ra$ and that $G/[G,G]\cong \dbZ^2$
\item[(c)] Let $A=\Aut(G)$. Use (a) and (b) to prove that there exists a homomorphism $\phi: A\to GL_2(\dbZ)$ with $\Ker(\phi)\cong \dbZ^2$. 
\item[(d)] Now prove that $\phi$ in (c) is surjective. You can use without proof the fact that $SL_n(\dbZ)$ is generated by elementary matrices $E_{ij}$.
\end{itemize}


\skv
{\bf 3.} Let $F$ be a free group of finite rank. As stated in class, subgroups of free groups are always free. Moreover, there is a simple formula for the rank of finite index subgroups, called the {\it Schreier formula}: if $H$ is a subgroup of finite index $n$ in $F$, then $H$ is free and $${\rm rk}(H)= n({\rm rk}(F)-1)+1.$$ The purpose of this problem is to prove Schreier's formula in a special case (without assuming the theorem that subgroups of free groups are free).

Let $d\in\dbN$, let $X=\{x_0,x_1,\ldots, x_d\}$ be a finite set with $d+1$ elements, and let $F=F(X)$ be the free group on $X$. Let $n\in\dbN$, and let $\phi: F\to\dbZ_n$ be the unique homomorphism
such that $\phi(x_0)=1$ and $\phi(x_i)=0$ for $i>0$ (such $\phi$ exists by the universal property
of free groups). Let $K=\Ker\,\phi$.

\begin{itemize}
\item[(a)] Prove that $K$ is a subgroup of $F$ of index $n$.

\item[(b)] Consider the following subset $Y$ of $F=F(X)$:  $$Y=\{x_0^n\}\cup\{x_0^k x_i x_0^{-k} : 1\leq i\leq d, 0\leq k\leq n-1\}.$$
Note that $|Y|=nd+1$. Prove that $Y$ is a generating set for $K$. {\bf Hint: } First prove that any element $w\in F$
can be written as $w=vr$ where $v\in \la Y\ra$ and $r=x_0^k$ for some $0\leq k<n$. 

\item[(c)] Now prove that $Y$ is a {\it free generating set for $K$}. In other words, let $m=|Y|=dn+1$, let $Y'=\{y_1,\ldots, y_m\}$
be another set of cardinality $m$, let $f:Y'\to Y$ be any bijective map and let $\phi:F(Y')\to K$ be the unique homomorphism extending $f$. Prove that $\phi$ is an isomorphism.
\end{itemize}


\skv
{\bf 4.} In all parts of this problem $G$ is a non-abelian group. A {\it maximal abelian} subgroup of $G$ is a maximal element of the set of all abelian subgroups of $G$ ordered by inclusion.
\begin{itemize}
\item[(a)] Prove that the union of all maximal abelian subgroups of $G$ is equal to $G$ and the intersection of all maximal abelian subgroups of $G$ is $Z(G)$.
\item[(b)] Prove that $G$ has at least 3 maximal abelian subgroups. 
\item[(c)] Give an example of $G$ which has exactly $3$ maximal abelian subgroups.
\end{itemize}
\skv
{\bf 5.} Let $\dbR$ denote the real numbers. The purpose of this problem is to show that the ring 
$A=\dbR[x,y]/(x^2+y^2-1)$ is not a UFD. For an element $f\in \dbR[x,y]$ 
we denote its image in $A$ by $[f]$.

\begin{itemize}
\item[(a)] Show that every element of $A$ can be uniquely represented in the form $[f(x)+g(x)y]$
where $f(x),g(x)\in F[x]$. 
\item[(b)] Show that $A$ has an automorphism $\phi$ of order $2$ such that
$\phi([f(x)])=[f(x)]$ for each $f(x)\in F[x]$ and $\phi([y])= -[y]$.
\item[(c)] Use (a) and (b) to construct a function $N:A\to F[x]$ such that $N(uv)=N(u)N(v)$ for all $u,v\in A.$   
\item[(d)] Use the function $N$ from (c) to show that $[x]$ is an irreducible element of $A$
and that the only invertible elements of $A$ are (images of) nonzero constant polynomials.
{\bf Hint:} It is essential that you are working over $\dbR$, not over $\dbC$.
\item[(e)] Now show that $A$ is not a UFD.
\end{itemize}
\skv


\end{document}
