\documentclass[12pt]{amsart}

\usepackage{amsmath}
\usepackage{amssymb}
\usepackage{amsthm}
%\usepackage{psfig}

\begin{document}
\baselineskip=16pt
\textheight=9in
\parindent=0pt 
\def\sk {\hskip .5cm}
\def\skv {\vskip .12cm}
\def\cos {\mbox{cos}}
\def\sin {\mbox{sin}}
\def\tan {\mbox{tan}}
\def\intl{\int\limits}
\def\lm{\lim\limits}
\newcommand{\frc}{\displaystyle\frac}
\def\xbf{{\mathbf x}}
\def\fbf{{\mathbf f}}
\def\gbf{{\mathbf g}}

\def\dbA{{\mathbb A}}
\def\dbB{{\mathbb B}}
\def\dbC{{\mathbb C}}
\def\dbD{{\mathbb D}}
\def\dbE{{\mathbb E}}
\def\dbF{{\mathbb F}}
\def\dbG{{\mathbb G}}
\def\dbH{{\mathbb H}}
\def\dbI{{\mathbb I}}
\def\dbJ{{\mathbb J}}
\def\dbK{{\mathbb K}}
\def\dbL{{\mathbb L}}
\def\dbM{{\mathbb M}}
\def\dbN{{\mathbb N}}
\def\dbO{{\mathbb O}}
\def\dbP{{\mathbb P}}
\def\dbQ{{\mathbb Q}}
\def\dbR{{\mathbb R}}
\def\dbS{{\mathbb S}}
\def\dbT{{\mathbb T}}
\def\dbU{{\mathbb U}}
\def\dbV{{\mathbb V}}
\def\dbW{{\mathbb W}}
\def\dbX{{\mathbb X}}
\def\dbY{{\mathbb Y}}
\def\dbZ{{\mathbb Z}}

\def\la{{\langle}}
\def\ra{{\rangle}}

\bf\centerline{Homework \#1}\rm
\vskip .1cm
{\bf Plan for next week:} Group actions (\S~4.1-4.3).
\vskip .1cm
\centerline{\bf Problems, to be submitted by Thursday, September 5th}
\vskip .1cm

{\bf 1.} Let $G$ be a group.

\begin{itemize}
\item[(a)] Define $\phi: G\to G$ by $\phi(g)=g^2$. Prove that $\phi$ is a homomorphism
if and only if $G$ is abelian.

\item[(b)] Assume that $x^2=1$ for any $x\in G$. Prove that $G$ is abelian.
\end{itemize}


{\bf 2.} 
\begin{itemize}

\item[(a)] Let $G$ be a cyclic group of order $n<\infty$. Prove that if $k\in\dbZ$,
then the mapping $\phi:G\to G$ defined by $\phi(x)=x^k$ is bijective if and
only if $k$ is relatively prime to $n$.

\item[(b)] Let $G$ be an arbitrary finite group of order $n<\infty$. Prove that if $k\in\dbZ$
is relatively prime to $n$, then the mapping $\phi:G\to G$ defined by $\phi(x)=x^k$ is bijective. 
{\bf Hint: } Use one of the corollaries of Lagrange theorem.
\end{itemize}


{\bf 3.} Find the minimal $n$ for which the symmetric group $S_n$ contains an element of order $15$
(and explain why your $n$ is indeed minimal). {\it Note: }All you need to know about $S_n$ for this
problem is stated in Section 1.3 of DF (pp.29-32).
\skv

{\bf 4.} Prove that an element $\bar a\in\dbZ_n$ is invertible if and only if $gcd(a,n)=1$
where $gcd$ is the greatest common divisor. You may use any standard theorem about integers
(e.g. unique factorization), but do not use any theorems about $\dbZ_n$.

{\bf Hint: } The forward direction is easy. For the opposite direction
either use the theorem about representation of $gcd(a,n)$ as an integral linear 
combination of $a$ and $n$ or, alternatively, show that the mapping $\phi_n:\dbZ_n\to\dbZ_n$
given by  $\phi_n(\bar x)=\bar x\bar a$ is injective whenever $gcd(a,n)=1$.
\skv
{\bf 5.} A group $G$ is called {\it finitely generated} if there exists a finite subset $S$ of $G$ such that
$\la S\ra=G$.
\begin{itemize}
\item[(a)] Prove that every finite group is finitely generated. 

\item[(b)] Let $\dbQ$ be the group of rational numbers with addition. Prove that $\dbQ$
is not finitely generated. 

\item[(c)] Prove that any finitely generated subgroup of $\dbQ$ is cyclic.
\end{itemize}

{\bf 6.} Let $G=D_8$, the dihedral group of order $8$ (that is, the group of isometries of a square). Prove that $|[G,G]|=2$ and describe $[G,G]$
explicitly without computing every single commutator. 


\skv
{\bf Index of a subgroup.} If $G$ is a group and $H$ is a subgroup of $G$, the index of $H$ in $G$, denoted by $[G:H]$, is
defined to be the cardinality of $G/H$, that is, the number of left cosets of $H$ in $G$.
It is not hard to show that the sets $G/H$ (the set of left cosets of $H$) and
$H\setminus G$ (the set of right cosets of $H$) always have the same cardinality, so there is no need to introduce ``left index'' and ``right index''. 

The full statement of Lagrange theorem asserts that if $G$ is a finite group and $H$ is a subgroup of $G$, then $[G:H]=\frac{|G|}{|H|}$ (typically one applies not the full statement but its most useful consequence, namely, that the order of $H$ divides the order of $G$).
\skv




{\bf 7.} Let $G$ be a group and let $H$ and $K$ be subgroups of $G$ of finite index
(note that $G$ is not assumed to be finite).

\begin{itemize}

\item[(a)] Assume that $H\subseteq K$. Prove that $[G:H]=[G:K][K:H]$ (recall that $[A:B]$
denotes the index of a subgroup $B$ in a group $A$).

\item[(b)] Let $m=[G:H]$ and $n=[G:K]$. Prove that
${\rm LCM}(m,n)\leq [G:H\cap K]\leq  mn$ (where ${\rm LCM}$ is the least common multiple).
\end{itemize}


{\bf Hint for (a):} If $A$ is a group and $B$ a subgroup of $A$, a
subset $S$ of $A$ is called a left transversal of $B$ in $A$
if $S$ contains precisely one element from each left coset of $B$
(an alternative name for a transversal is a system of left coset representatives).
Let $\{g_1,\ldots, g_r\}$ be a left transversal of $K$ in $G$
and $\{k_1,\ldots, k_s\}$ a left transversal of $H$ in $K$. Prove
that $\{g_i k_j\}_{1\leq i\leq r, 1\leq j\leq s}$ is a left transversal for $H$ in $G$.
Recall that if $B$ is a subgroup of a group $G$, then $xB=yB$ $\iff$ $x^{-1}y\in B$
for $x,y\in G$.
\skv

{\bf 8.} Let $G$ be a group and $H$ a subgroup of $G$.
\begin{itemize}
\item[(a)] Prove directly from the definitions that the following two statements are equivalent:
\begin{itemize}
\item[(i)] $gHg^{-1}= H$ for all $g\in G$
\item[(ii)] $gHg^{-1}\subseteq H$ for all $g\in G$
\end{itemize}
\item[(b)] Give an example of a group $G$, a subgroup $H$ of $G$ and an element $g\in G$ such that
$gHg^{-1}$ is a proper subgroup of $H$.
\end{itemize}
 
\end{document}
