\documentclass[12pt]{amsart}

\usepackage{amsmath}
\usepackage{amssymb}
\usepackage{amsthm}
\usepackage{amscd}
\usepackage[all]{xy}
%\usepackage{psfig}

\begin{document}
\baselineskip=16pt
\textheight=8.5in
\textwidth=6in
\parindent=0pt
\def\sk {\hskip .5cm}
\def\skv {\vskip .12cm}
\def\cos {\mbox{cos}}
\def\sin {\mbox{sin}}
\def\tan {\mbox{tan}}
\def\intl{\int\limits}
\def\lm{\lim\limits}
\newcommand{\frc}{\displaystyle\frac}
\def\xbf{{\mathbf x}}
\def\fbf{{\mathbf f}}
\def\gbf{{\mathbf g}}

\def\dbA{{\mathbb A}}
\def\dbB{{\mathbb B}}
\def\dbC{{\mathbb C}}
\def\dbD{{\mathbb D}}
\def\dbE{{\mathbb E}}
\def\dbF{{\mathbb F}}
\def\dbG{{\mathbb G}}
\def\dbH{{\mathbb H}}
\def\dbI{{\mathbb I}}
\def\dbJ{{\mathbb J}}
\def\dbK{{\mathbb K}}
\def\dbL{{\mathbb L}}
\def\dbM{{\mathbb M}}
\def\dbN{{\mathbb N}}
\def\dbO{{\mathbb O}}
\def\dbP{{\mathbb P}}
\def\dbQ{{\mathbb Q}}
\def\dbR{{\mathbb R}}
\def\dbS{{\mathbb S}}
\def\dbT{{\mathbb T}}
\def\dbU{{\mathbb U}}
\def\dbV{{\mathbb V}}
\def\dbW{{\mathbb W}}
\def\dbX{{\mathbb X}}
\def\dbY{{\mathbb Y}}
\def\dbZ{{\mathbb Z}}

\def\la{{\langle}}
\def\ra{{\rangle}}

\def\Ker{{\rm Ker\,}}
\def\Aut{{\rm Aut}}
\def\Inn{{\rm Inn}}

\bf\centerline{Homework \#9. }\rm
\vskip .1cm
{\bf Plan for next week:} Unique Factorization Domains (8.3, 9.3) and Irreducibility criteria in polynomial rings (9.4). Note that we already discussed some of the main results from 8.3.

\centerline{\bf Problems, to be submitted by Thursday, November 7th}
\skv
{\bf 1.} Let $\dbZ[i]=\{a+bi: a,b\in\dbZ\}$ be the ring of Gaussian integers.
\begin{itemize}
\item[(a)] Prove that $\dbZ[i]$ is a Euclidean domain.
\item[(b)] Prove that $\dbZ[i]\cong \dbZ[x]/(x^2+1)$
\item[(c)] Now use (a) and (b) to find all maximal ideals of $\dbZ[x]$ (the ring of polynomials over $\dbZ$ in one variable) which contain 
$x^2+1$ and $15$. If you have not studied Gaussian integers before, carefully read the corresponding section in DF (pp. 289-292).
\end{itemize}
\skv
{\bf 2.} Let $D$ be a positive integer such that $D\equiv 3\mod 4$, and let
$R=\dbZ[\frac{1+\sqrt{-D}}{2}]$, that is, $R$ is the minimal subring of $\dbC$ containing $\dbZ$ and $\frac{1+\sqrt{-D}}{2}$.

\begin{itemize}
\item[(a)] Prove that $R=\{a+ b \frac{1+\sqrt{-D}}{2} : a,b\in\dbZ \}$. You may skip details, but it should
be clear from your argument where the assumption $D\equiv 3\mod 4$ is used (otherwise the result is simply not true).

\item[(b)] Assume that $D=3, 7$ or $11$. Prove that $R$ is a Euclidean domain.
\end{itemize}

\skv

{\bf 3.} Let $R=\dbZ[\sqrt{5}]=\{a+b\sqrt{5} : a,b\in\dbZ\}$. Find an element of $R$
which is irreducible but not prime and deduce that $R$ is not a unique factorization domain (UFD).
\skv
 {\bf Hint:} Consider the equality $2\cdot 2=(\sqrt{5}+1)(\sqrt{5}-1)$. In order to check
whether some element of $R$ is irreducible it is convenient to use the standard norm function
$N:R\to\dbZ_{\geq 0}$ given by $N(a+b\sqrt{5})=|a^2-5b^2|$ (note that $N(uv)=N(u)N(v)$).
\skv\skv 
{\bf 4.} Let $R=\dbZ+x\dbQ[x]$, the subring of $\dbQ[x]$ consisting of
polynomials whose constant term is an integer.
\begin{itemize}
\item[(a)] Show that the element $\alpha x$, with $\alpha\in\dbQ$ is NOT irreducible
in $R$. Then show that $x$ cannot be written as a product of irreducibles in $R$.
Note that by Lecture~18, this implies that $R$ is not Noetherian.

\item[(b)] Now prove directly that $R$ is not Noetherian by showing that
$I=x\dbQ[x]$ is an ideal of $R$ which is not finitely generated.

\item[(c)] Give an example of a non-Noetherian domain which is a UFD.
\end{itemize}
\skv
{\bf 5.} Let $R$ be a commutative ring with $1$. 
\begin{itemize}
\item[(a)]   Let $M$ be an ideal of $R$. Prove that the following conditions are equivalent:
\begin{itemize}
\item[(i)] $M$ is the unique maximal ideal of $R$ 
\item[(ii)] every element of $R\setminus M$ is invertible. 
\end{itemize}
Rings satisfying these equivalent conditions are called local.
\item[(b)] Let $F$ be a field and $F[[x]]$ the ring of power series over $F$. Prove that $F[[x]]$ is local.
\item[(c)] Now let $R$ be arbitrary, let $P$ be a prime ideal of $R$ and $S=R\setminus P$. Prove that the localization
$S^{-1}R$ is local and explicitly describe its unique maximal ideal.
\end{itemize}
\end{document}
