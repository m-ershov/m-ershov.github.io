\documentclass[12pt]{article}

\usepackage{amsmath}
\usepackage{amssymb}
\usepackage{amsthm}
\usepackage{url}
\usepackage{hyperref}
%\usepackage{psfig}

\begin{document}
\baselineskip=16pt
\textheight=8.5in
\textwidth=6in
\parindent=0pt
\def\sk {\hskip .5cm}
\def\skv {\vskip .07cm}
\def\cos {\mbox{cos}}
\def\sin {\mbox{sin}}
\def\tan {\mbox{tan}}
\def\intl{\int\limits}
\def\lm{\lim\limits}
\newcommand{\frc}{\displaystyle\frac}
\def\xbf{{\mathbf x}}
\def\fbf{{\mathbf f}}
\def\gbf{{\mathbf g}}

\def\Ker{{\rm Ker\,}}
\def\Gal{{\rm Gal\,}}
\def\phi{\varphi}

\def\dbA{{\mathbb A}}
\def\dbB{{\mathbb B}}
\def\dbC{{\mathbb C}}
\def\dbD{{\mathbb D}}
\def\dbE{{\mathbb E}}
\def\dbF{{\mathbb F}}
\def\dbG{{\mathbb G}}
\def\dbH{{\mathbb H}}
\def\dbI{{\mathbb I}}
\def\dbJ{{\mathbb J}}
\def\dbK{{\mathbb K}}
\def\dbL{{\mathbb L}}
\def\dbM{{\mathbb M}}
\def\dbN{{\mathbb N}}
\def\dbO{{\mathbb O}}
\def\dbP{{\mathbb P}}
\def\dbQ{{\mathbb Q}}
\def\dbR{{\mathbb R}}
\def\dbS{{\mathbb S}}
\def\dbT{{\mathbb T}}
\def\dbU{{\mathbb U}}
\def\dbV{{\mathbb V}}
\def\dbW{{\mathbb W}}
\def\dbX{{\mathbb X}}
\def\dbY{{\mathbb Y}}
\def\dbZ{{\mathbb Z}}

\def\Aut{{\rm Aut}}
\def\deg{{\rm deg}}

\def\la{{\langle}}
\def\ra{{\rangle}}

\bf\centerline{Homework \# 11. }\rm

{\bf Plan for the remaining classes:} Dimension theory of affine varieties, localization of affine varieties
and prime spectrum of a ring (parts of 15.2, 15.4 and 15.5). Good references on
commutative algebra and algebraic geometry freely available online are notes by 
J. Milne

\centerline{\url{http://www.jmilne.org/math/xnotes/CA.pdf}} 

and 

\centerline{\url{http://www.jmilne.org/math/CourseNotes/AG.pdf}}

\skv
\skv
\bf\centerline{Problems, to be submitted by Thu, December 5th. }\rm
\skv
{\bf Problem 1:} DF, Problem~19 on p. 332. Make sure to read about the Buchberger's algorithm in 9.6 prior to solving this problem.
\skv
{\bf Problem 2:} Let $I$ be an ideal of $\dbZ[x]$, and suppose that $I$ contains a monic polynomial $f(x)$ of degree $n$. Prove that
$I$ can be generated (as an ideal) by at most $n+1$ elements. 
\skv
{\bf Problem 3:} Let $k$ be a field.
An algebraic set $V\subseteq k^n$ is called \underline{irreducible} 
if $V\neq\emptyset$ and $V$ cannot be written as the union $V=V_1\cup V_2$ where
$V_1$ and $V_2$ are both algebraic, with $V_1\neq V$ and $V_2\neq V$.
\begin{itemize}
\item[(a)] (practice) Prove that $V$ is irreducible if and only if its vanishing ideal
$I(V)$ is prime.  
\item[(b)] We will prove that any algebraic set $V$
can be uniquely written as a union of finitely many algebraic subsets 
$V=\cup_{i=1}^k V_i$ where $V_i$'s are irreducible and do not contain
each other. Such $V_i$'s are called \underline{irreducible components of $V$}.
Assume that $k$ is infinite, and 
let $$V=Z(xy-y, x^2z-z)\subset k^3,$$ the set of common zeroes of $xy-y$
and $x^2z-z$. Find irreducible components of $V$ and their vanishing ideals.
The answer will depend on $char(k)$.
\end{itemize}
\skv
{\bf Problem 4:} Let $k$ be an algebraically closed field. Prove that any
nonzero prime ideal  of $k[x,y]$ is equal to $(f)$ for some irreducible $f\in k[x,y]$
or $(x-a,y-b)$ for some $a,b\in k$. You may use the fact that $k[x,y]$ has Krull dimension 2.
\skv
\skv 
 
In Problems 5 and 6 we identify the set $Mat_n(k)$ of $n\times n$ matrices over a field $k$
with $k^{n^2}$ and thus can talk about Zariski topology on $Mat_n(k)$.
\skv
{\bf Problem 5:} Let $k$ be an algebraically closed field.
\begin{itemize}
\item[(a)] Prove that $SL_n(k)=\{A\in Mat_n(k): \det(A)=1\}$ is Zariski closed
(that is, closed in Zariski topology) and find its dimension.

\item[(b)] Fix $1\leq d\leq n$, and let $R_d(n,k)$  be the set of all matrices in $Mat_n(k)$
which have rank $\leq d$. Prove that $R_d(n,k)$ is Zariski closed, guess its dimension and
give a heuristic argument.
\end{itemize}

\skv
{\bf Problem 6:} Let $k$ be an arbitrary field. If $Y$ is a subset of $k^n$, we will
denote by $\overline Y$ the {\it Zariski closure} of $Y$, that is, the closure of $Y$ in the 
Zariski topology. 

\sk Now let $A$ be a commutative subset of $Mat_n(k)$, that is, $ab=ba$ for all $a,b\in A$.
Prove that $\overline A$ is also commutative. {\bf Hint:} First show that for any $a\in Mat_n(k)$,
the centralizer of $a$ in $Mat_n(k)$ is Zariski closed. Then show that $ab=ba$ for all $a\in A$
and $b\in \overline A$ and finally deduce the assertion of the problem.
\skv
{\bf Problem 7:} Again let $k$ be an algebraically closed field.
Let $Y$ be a subset of $k^n$. Let 
$k[Y]=k[x_1,\ldots, x_n]/I(Y)$. As we will discuss in class on Tue, Nov 26, $k[Y]$ can be naturally identified with the ring of polynomial functions from $Y$ to $k$ (with pointwise addition and multiplication). Let $O(Y)$
be the set of all everywhere defined rational functions on $Y$, that is,
all functions $f:Y\to k$ for which there exist polynomials $p,q\in k[x_1,\ldots, x_n]$
s.t. $q$ does not vanish at any point of $Y$ and $f=p/q$ as a function on $Y$. Clearly,
$k[Y]\subseteq O(Y)$.
\begin{itemize}
\item[(a)] Prove that if $Y$ is an algebraic set, then $O(Y)=k[Y]$. {\bf Hint:} Use the weak
Nullstellensatz.
\item[(b)] Let $Y=k^1\setminus\{0\}$, the affine line with $0$ removed. Prove that
$k[Y]=k[x]$ (polynomials in one variable) while $O(Y)=k[x,1/x]$.
\item[(c)] Find an algebraic subset $Z$ of $k^2$ such that $k[Z]\cong k[x,1/x]$.
How is $Z$ related to $Y$ from part (b)? (No formal answer is expected).
\item[(d)] Find a non-algebraic subset $W$ of $k^2$ for which $O(W)=k[W]\cong k[x_1,x_2]$.
\end{itemize}
\end{document} 