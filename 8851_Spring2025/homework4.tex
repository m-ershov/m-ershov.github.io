\documentclass[12pt]{amsart}

\usepackage{amsmath}
\usepackage{amssymb}
\usepackage{amsthm}
%\usepackage{psfig}

\newtheorem* {Theorem}    {Theorem}
\newtheorem* {Lemma}    {Lemma}
\newtheorem* {Definition}    {Definition}


\begin{document}
 \pagenumbering{gobble}
\baselineskip=16pt
\textheight=8.7in
\textwidth=6.5in
%\parindent=0pt 
\def\sk {\hskip .5cm}
\def\skv {\vskip .08cm}
\def\cos {\mbox{cos}}
\def\sin {\mbox{sin}}
\def\tan {\mbox{tan}}
\def\intl{\int\limits}
\def\lm{\lim\limits}
\newcommand{\frc}{\displaystyle\frac}
\def\xbf{{\mathbf x}}
\def\fbf{{\mathbf f}}
\def\gbf{{\mathbf g}}

\def\dbA{{\mathbb A}}
\def\dbB{{\mathbb B}}
\def\dbC{{\mathbb C}}
\def\dbD{{\mathbb D}}
\def\dbE{{\mathbb E}}
\def\dbF{{\mathbb F}}
\def\dbG{{\mathbb G}}
\def\dbH{{\mathbb H}}
\def\dbI{{\mathbb I}}
\def\dbJ{{\mathbb J}}
\def\dbK{{\mathbb K}}
\def\dbL{{\mathbb L}}
\def\dbM{{\mathbb M}}
\def\dbN{{\mathbb N}}
\def\dbO{{\mathbb O}}
\def\dbP{{\mathbb P}}
\def\dbQ{{\mathbb Q}}
\def\dbR{{\mathbb R}}
\def\dbS{{\mathbb S}}
\def\dbT{{\mathbb T}}
\def\dbU{{\mathbb U}}
\def\dbV{{\mathbb V}}
\def\dbW{{\mathbb W}}
\def\dbX{{\mathbb X}}
\def\dbY{{\mathbb Y}}
\def\dbZ{{\mathbb Z}}

\def\la{{\langle}}
\def\ra{{\rangle}}
\def\eps{{\varepsilon}}
\def\lam{{\lambda}}
\def\Ker{{\rm Ker}}
\def\rk{{\rm rk}}
\def\summ{{\sum\limits}}

\centerline{\bf Math 8851. Homework \#4. To be completed by 6pm on Thu, Mar 27} 
\skv
\skv

{\bf 1.} Let $G=\la S\ra$ be a group and $H=\la T\ra$ a subgroup of $G$, with $S,T$ finite. The {\it distortion} function
of $H$ in $G$ is $D_H:\dbN\to\dbN$ defined as follows:
$D_H(n)$ is the maximal word length $\|h\|_T$ where $h$ ranges over all elements of $H$ with $\|h\|_S\leq n$.
\skv
While $D_H$ depends on the choice of generating sets $S$ and $T$, its equivalence class (with respect to standard notion
of equivalence of functions) does not. In particular, one can talk about $D_H$ being polynomial of fixed degree or exponential.

\begin{itemize}
\item[(a)] Prove that if $H$ is infinite, there exists $C>0$ such that $D_H(n)\geq Cn$ for all sufficiently large $n$.
\item[(b)] Now prove that $H$ is undistorted (as defined in class) if and only if there exist $K>0$
such that $D_H(n)\leq Kn$ for all $n$.
 
\item[(c)] Let $G=BS(1,2)=\la a,b | bab^{-1}=a^2\ra$ and $H=\la a\ra$. Prove that the distortion function is exponential. By
definition this means there exist constants $\lam,\mu>1$ such that $\lam^n\leq  D_H(n)\leq \mu^n$ for all sufficiently large $n$.
{\bf Note:} Unlike word growth, $D_H(n)$ can grow super-exponentially.
\item[(d)] Let $G$ be the Heisenberg group over $\dbZ$. It can be defined as the group of $3\times 3$ upper unitriangular matrices
over $\dbZ$ (upper unitriangular means that the entries below the diagonal are $0$ and the entries on the diagonal are $1$)
and has the presentation $G=\la x,y,z \mid [x,y]=z, [x,z]=[y,z]=1\ra$ where $x=E_{12}(1), y=E_{23}(1)$ and $z=E_{13}(1)$ (by
$E_{ij}(\lambda)$ we denote the matrix which has $1$'s on the diagonal, $\lambda$ in the $(i,j)$-position and $0$'s everywhere else).
Prove that the subgroups $\la x\ra$ and $\la y\ra$ are undistorted while for $H=\la z\ra$ the distortion is quadratic.
\end{itemize}

{\bf 2.} Let $G=\dbZ^2$. Prove that for any non-trivial subgroup $H$ of $G$ there exist generating sets $S$ and $T$ of $G$ such that
$H$ is quasi-convex with respect to $S$, but not quasi-convex with respect to $T$. {\bf Hint:} First prove this
for $H$ generated by $(n,0)$ for some $n\in\dbN$ and then use a suitable theorem to deduce the general case.
\skv
{\bf 3.} Prove that an undistorted (equivalently, quasi-convex) subgroup of a hyperbolic group is hyperbolic (Corollary~19.3 from class).

\skv
{\bf 4.} Let $G$ be a group, $S$ a finite generating set for $G$ and $H$ a subgroup of $G$ which is quasi-convex with respect to $S$.
Use the  \u Svar\u c-Milnor lemma to prove that $H$ is finitely generated and undistorted in $G$. {\bf Hint:} Apply the 
 \u Svar\u c-Milnor lemma to the left-multiplication action of $H$ on a suitable subspace $X$ of $Cay(G,S)$. Note that you cannot
 take $X=Cay(G,S)$ since in that case the action need not be cobounded (in fact, will not be cobounded unless $H$ has finite index in $G$).
 Also keep in mind that $X$ needs to be geodesic.

\skv
{\bf 5.} This problem deals with quasi-convex subsets (not necessary subgroups) in hyperbolic groups. Let $G$ be a hyperbolic group
and $S$ a finite generating set for $G$.  
\begin{itemize}
\item[(a)] Let $X$ be an arbitrary subset of $G$. Prove that the following equivalent:
\begin{itemize}
\item[(i)] $X$ is quasi-convex when considered as a subset of $Cay(G,S)$
\item[(ii)] There exists $\tau$ such that for any $x\in X$ and any geodesic $[1,x]$ in $Cay(G,S)$ we have $[1,x]\subseteq X^{+\tau}$.
{\bf Note:} We are not assuming that $1\in X$. Here $1$ could be replaced by any fixed element of $Cay(G,S)$.
\end{itemize}
\item[(b)] Use (a) to show that if $X$ and $Y$ are quasi-convex subsets of $X$, then the sets $X\cup Y$ and $XY=\{xy: x\in X, y\in Y\}$
are also quasi-convex.
\end{itemize}
{\bf Hint:} Directly apply the slim triangle condition to relate (i) and (ii) in (a).



\skv

{\bf 6.} In 1921 Nielsen proved that finitely generated subgroups of free groups are free. This problem relates Nielsen's proof to the fact
that finitely generated subgroups of free groups are undistorted.

Let $X$ be a finite set, $F=F(X)$, the free group on $X$ and $U=\{u_1,\ldots, u_n\}$ a finite  tuple of elements of $F$
(the order here does not really play a role; we are talking about tuples rather than sets to allow repetitions).
Then $U$
is called {\it Nielsen reduced} if
\begin{itemize}
\item[(i)] $u_i\neq 1$ for all $i$, all $u_i$ are distinct and $U\cap U^{-1}=\emptyset$
\item[(ii)] for all $v,w\in U\,\cup\, U^{-1}$ with $w\neq v^{-1}$ we have $\|vw\|\geq \max(\|v\|,\|w\|)$.
\item[(iii)] for all $u,v,w\in U\cup U^{-1}$ with $v\neq u^{-1},w^{-1}$ we have
$$\|uvw\|>\|u\|-\|v\|+\|w\|.$$
\end{itemize}
Nielsen proved that if we are given any $U=\{u_1,\ldots, u_n\}$ such that $u_i\neq 1$ for some $i$, one can obtain a Nielsen-reduced set from
$U$ using the following operations: 
\begin{itemize}
\item[(a)] replace $u_i$ by $u_i^{-1}$ for some $i$;
\item[(b)] replace $u_i$ by $u_i u_j^{\pm 1}$ or by $u_j^{\pm 1} u_i$ for some $i\neq j$;
\item[(c)] if $u_i=1$ for some $i$, remove $u_i$.
\end{itemize}
Since none of operations (a), (b) and (c) changes the subgroup generated by the tuple, it follows that any non-trivial finitely generated
subgroup of $F$ has a Nielsen-reduced generating set.

Suppose now that $H\neq \{1\}$ is a finitely generated subgroup and $U=\{u_1,\ldots, u_n\}$ a Nielsen reduced generating set for $H$.
Consider another alphabet with $n$ elements $Y=\{y_1,\ldots, y_n\}$, and let $\phi:F(Y)\to F(X)$ be the unique homomorphism
such that $\phi(y_i)=u_i$ for all $i$. In other words, $\phi$ takes a reduced word in $Y\cup Y^{-1}$ and replaces each $y_i$ by $u_i\in F(X)$.
Note that the image of $\phi$ is precisely $H$.

Now we arrive at the actual statement of the problem:

\begin{itemize}
\item[(1)] Prove that for any $1\neq w\in F(X)$ we have $\|\phi(w)\|_X\geq \|w\|_Y$. {\bf Hint:} Show that for any subword
of length $3$ in $w$, say $y_{1} y_2 y_3$ (with $y_i\in Y^{\pm 1}$), the middle word $\phi(y_2)$ of the corresponding
product $\phi(y_{1})\phi( y_2)\phi( y_3)$ cannot completely cancel in $F(X)$.
\item[(2)] Property (1) immediately implies that $\phi$ has trivial kernel which means that $H$ is free and $U$ is a free generating set
for $H$. Show that (1) also implies that $H$ is undistorted in $F(X)$. 
\end{itemize}

\skv


{\bf 7.} In 1926 Schreier proved that arbitrary subgroups of free groups are free. This problem relates Schreier's proof to the fact
that finitely generated subgroups of free groups are quasi-convex.

We start with Schreier's method for computing a generating set for a subgroup of an arbitrary group for which a generating set is given
(part (a) below). So let $G$ be any group, $X$ a generating set for $G$, $H$ a subgroup of $G$ and $T$ a right transversal for $H$ (a subset of $G$
containing exactly one element from each right coset $Hg$). As in HW\#3.6, for each $g\in G$ we denote by $\overline g$ the unique
element of $T$ such that $H\overline{g}=Hg$. Let us also assume that $1\in T$ (this condition is not really necessary, but
slightly simplifies the argument). Let
$Y$ be the set of all elements of the form $tx(\overline{tx})^{-1}$ with $t\in T$ and $x\in X$.

\begin{itemize}
\item[(a)] Prove that $Y$ is a generating set for $H$. Moreover prove that for any $h\in H$ we have $\|h\|_{Y}\leq \|h\|_{X}$
and describe an algorithm with starts with a word of length $n$ in $X$ representing some $h\in H$ and produces
a word of length $n$ in $Y$ representing the same $h\in H$.
\item[(b)] Now assume that $G$ is free, $H$ is finitely generated and $T$ is a Schreier transversal as defined in HW\#3.6.
As stated in HW\#3.6, by Schreier's theorem non-identity element of $Y$ are all distinct and freely generate $Y$. Since
$H$ is finitely generated, it follows that $Y$ contains only finitely many non-identity elements. Use this fact to show
that $H$ is quasi-convex in $G$ (with respect to $X$).
\end{itemize}
{\bf Hint for (a):} It is probably more convenient to prove the following more general statement: any $g\in G$ with
$\|g\|_X=n$ can be written as $g=y_1\ldots y_n t$ where $y_i\in Y^{\pm 1}$ and $t\in T$ (note that if $g\in H$,
we are forced to have $t=1$ since $y_i\in H$ for each $i$). To prove this statement by induction on $n$ it suffices to show that
for any $t_0\in Y$ and $x$ one can write each of the words $t_0 x$ and $t_0x^{-1}$ in the form $y t$ for some $y\in Y^{\pm 1}$
and $t\in T$. The first case is quite straightforward. For the second case, first show that if $t'=\overline{t_0x^{-1}}$,
then $\overline{t'x}=t_0$.
\end{document}