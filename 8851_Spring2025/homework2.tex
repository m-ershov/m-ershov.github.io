\documentclass[12pt]{amsart}

\usepackage{amsmath}
\usepackage{amssymb}
\usepackage{amsthm}
%\usepackage{psfig}

\newtheorem* {Theorem}    {Theorem}
\newtheorem* {Lemma}    {Lemma}


\begin{document}
 \pagenumbering{gobble}
\baselineskip=16pt
\textheight=8.7in
\textwidth=6.5in
%\parindent=0pt 
\def\sk {\hskip .5cm}
\def\skv {\vskip .08cm}
\def\cos {\mbox{cos}}
\def\sin {\mbox{sin}}
\def\tan {\mbox{tan}}
\def\intl{\int\limits}
\def\lm{\lim\limits}
\newcommand{\frc}{\displaystyle\frac}
\def\xbf{{\mathbf x}}
\def\fbf{{\mathbf f}}
\def\gbf{{\mathbf g}}

\def\dbA{{\mathbb A}}
\def\dbB{{\mathbb B}}
\def\dbC{{\mathbb C}}
\def\dbD{{\mathbb D}}
\def\dbE{{\mathbb E}}
\def\dbF{{\mathbb F}}
\def\dbG{{\mathbb G}}
\def\dbH{{\mathbb H}}
\def\dbI{{\mathbb I}}
\def\dbJ{{\mathbb J}}
\def\dbK{{\mathbb K}}
\def\dbL{{\mathbb L}}
\def\dbM{{\mathbb M}}
\def\dbN{{\mathbb N}}
\def\dbO{{\mathbb O}}
\def\dbP{{\mathbb P}}
\def\dbQ{{\mathbb Q}}
\def\dbR{{\mathbb R}}
\def\dbS{{\mathbb S}}
\def\dbT{{\mathbb T}}
\def\dbU{{\mathbb U}}
\def\dbV{{\mathbb V}}
\def\dbW{{\mathbb W}}
\def\dbX{{\mathbb X}}
\def\dbY{{\mathbb Y}}
\def\dbZ{{\mathbb Z}}

\def\la{{\langle}}
\def\ra{{\rangle}}
\def\eps{{\varepsilon}}
\def\lam{{\lambda}}
\def\Ker{{\rm Ker}}
\def\rk{{\rm rk}}
\def\summ{{\sum\limits}}

\bf\centerline{Math 8851. Homework \#2. Preliminary version} 
\skv
\bf\centerline{To be completed by 11:59pm on Mon, Feb 10}\rm
\vskip .1cm
1 (extended version of HW\#1.1). Let $G$ be a group and $S$ a generating set of $G$. 
\begin{itemize}
\item[(a)] Prove that the following are equivalent:
\begin{itemize}
\item[(i)] $G$ is free and $S$ is a free generating set of $G$. By definition this means that every element of $G$
can be uniquely written as a reduced word $\prod\limits_{i=1}^n s_i^{\eps_i}$ with $s_i\in S$ and $\eps_i=\pm 1$
(reduced means that $s_i\neq s_{i+1}$ whenever $\eps_{i+1}=-\eps_i$).
\item[(ii)] The Cayley graph $Cay(G,S)$ is a tree and $S$ has no elements of order $2$.
\end{itemize}
\item[(b)] Describe all groups $G$ with the property that $Cay(G,S)$ is a tree for some generating set $S$ of $G$.
\end{itemize}
\skv
2. Let $(X,R)$ be a group presentation, $G=\la X|R \ra$, and let $\mathcal D$ be a van Kampen diagram over $(X,R)$. Prove that one can label the vertices of $\mathcal D$ by elements of $G$ such that whenever $e$ is an oriented edge from a vertex $v$ to a vertex $w$ we have $\ell(w)=\ell(v)\ell(e)$ (where $\ell(\cdot)$ denotes the label of a vertex or an edge). Moreover, show that if we fix a base vertex $v_0$,
then $\ell(v_0)$ can be chosen to be any element of $G$, and once
$\ell(v_0)$ is chosen, all other vertex labels are uniquely determined.
{\bf Hint:} Use van Kampen's lemma.
\skv
3. Let $X=\{a,b\}$, $R=\{aba^{-1}b^{-1}\}$ and $G=\la X | R \ra\cong\dbZ^2$.
\begin{itemize}
\item[(a)] Let $w=a^2b^2a^{-2}b^{-2}$, and let $\mathcal D$ be the disk van Kampen diagram 
of area $4$ from the example in Lecture~9 with $\ell(\partial D)=w$. Use the proof of van Kampen lemma
to explicitly write $w$ in the form $\prod_{i=1}^4 u_i r_i^{\pm 1}u_i^{-1}$ with
$u_i\in F(X)$ and $r_i=aba^{-1}b^{-1}$ (as the only element of $R$ in this case).
\item[(b)] Now reverse the process from (a): start with the factorization found in (a),
construct the corresponding `lollipop' diagram, call it $\mathcal D'$, and show that
after edge cancellations in $\partial \mathcal D'$ (as defined below), one obtains the original diagram $\mathcal D$ from (a).
\end{itemize}

Here is what we formally mean by an edge cancellation. Suppose that $e_1$ and $e_2$
are consecutive edges of $\partial \mathcal D'$ (as we traverse $\partial \mathcal D'$ in some direction) which have the same label $x\in X$ and point in opposite directions.
As we traverse $e_1$, we move from some vertex $u$ to some vertex $v$, and then
as we traverse $e_2$, we move from $v$ to some vertex $w$ (which may concide with $u$).

\begin{itemize}
\item[(i)] If $w\neq u$, we start by gluing the edges $e_1$ and $e_2$, identifying 
$u$ and $w$. If after this process the vertex $u=w$ becomes a leaf, we also remove the entire edge $e_1=e_2$.
\item[(ii)] If $w=u$, we remove the edges $e_1$ and $e_2$ possibly together with any cells of $\mathcal D'$ enclosed between $e_1$ and $e_2$.
\end{itemize}
You should convince yourself that each of the operations (i) and (ii) results in a valid van Kampen diagram whose boundary label is obtained from $\ell(\partial \mathcal D')$
by the cancellation of the subword $x x^{-1}$ or $x^{-1} x$ corresponding to the edges
$e_1$ and $e_2$.
\end{document}



