\documentclass[12pt]{article}

\usepackage{amsmath}
\usepackage{amssymb}
\usepackage{amsthm}
%\usepackage{psfig}

\begin{document}
\baselineskip=16pt
\parindent=0pt
\def\sk {\hskip .4cm}
\def\skv {\vskip .19cm}
\def\cos {\mbox{cos}}
\def\sin {\mbox{sin}}
\def\tan {\mbox{tan}}
\def\summ{\sum\limits}
\def\intl{\int\limits}
\def\lm{\lim\limits}
\newcommand{\frc}{\displaystyle\frac}

\def\dbA{{\mathbb A}}
\def\dbB{{\mathbb B}}
\def\dbC{{\mathbb C}}
\def\dbD{{\mathbb D}}
\def\dbE{{\mathbb E}}
\def\dbF{{\mathbb F}}
\def\dbG{{\mathbb G}}
\def\dbH{{\mathbb H}}
\def\dbI{{\mathbb I}}
\def\dbJ{{\mathbb J}}
\def\dbK{{\mathbb K}}
\def\dbL{{\mathbb L}}
\def\dbM{{\mathbb M}}
\def\dbN{{\mathbb N}}
\def\dbO{{\mathbb O}}
\def\dbP{{\mathbb P}}
\def\dbQ{{\mathbb Q}}
\def\dbR{{\mathbb R}}
\def\dbS{{\mathbb S}}
\def\dbT{{\mathbb T}}
\def\dbU{{\mathbb U}}
\def\dbV{{\mathbb V}}
\def\dbW{{\mathbb W}}
\def\dbX{{\mathbb X}}
\def\dbY{{\mathbb Y}}
\def\dbZ{{\mathbb Z}}

\def\Ker{{\rm Ker\,}}
\def\phi{{\varphi}}
\def\la{{\langle}}
\def\ra{{\rangle}}

\bf\centerline{ Solutions to the Final Exam from Fall 2007.}
\skv 
\bf{1. }\rm Prove that for any integer $n\geq 2$
$$\frac{1}{1\cdot 2}+\frac{1}{2\cdot 3}+\ldots +\frac{1}{n\cdot (n+1)}=\frac{n}{n+1} \eqno \mbox{ statement }P_n$$
\skv
\bf{Solution: }\rm We prove that $P_n$ is true for any $n\geq 2$ by induction on $n$. 

{\it Base case ($n=2$)}: $\mbox{LHS of }P_2=\frac{1}{1\cdot 2}+\frac{1}{2\cdot 3}=\frac{4}{6}=\frac{2}{3}$;
$\mbox{RHS of }P_2=\frac{2}{2+1}=\frac{2}{3}$. Thus, $P_2$ is true.

{\it Induction step $P_{n-1}\Rightarrow P_n$ (where $n\geq 3$)}. Assume that $P_{n-1}$ is true, that is,
$\frac{1}{1\cdot 2}+\frac{1}{2\cdot 3}+\ldots +\frac{1}{(n-1)\cdot n}=\frac{n-1}{n}$. 
Then 
\begin{multline*}
\frac{1}{1\cdot 2}+\frac{1}{2\cdot 3}+\ldots +\frac{1}{n\cdot (n+1)}=
\left(\frac{1}{1\cdot 2}+\frac{1}{2\cdot 3}+\ldots +\frac{1}{(n-1)\cdot n}\right)+\frac{1}{n\cdot (n+1)}=\\
\frac{n-1}{n}+\frac{1}{n(n+1)}=\frac{(n-1)(n+1)+1}{n(n+1)}=\frac{n^2}{n(n+1)}=\frac{n}{n+1}
\end{multline*}
Thus, $P_n$ is true as well.
\skv

\skv
\bf{2. }\rm In all parts of this problem NO proofs are required.
\vskip .1cm
(a) Let $\phi: G\to G'$ be a group homomorphism. Give the definition of 
the \it{kernel }\rm of $\phi$ and the \it{range }\rm of $\phi$. 
\skv
{\bf Answer: } $\Ker\phi=\{x\in G : \phi(x)=e'\}$ where $e'$ is the identity of $G'$.

The range of $\phi$ is the set $\{y\in G' : y=\phi(x) \mbox{ for some } x\in G\}$.
\skv
(b)  State the \it{fundamental theorem of homomorphisms }\rm (FTH):

{\bf Answer: } If $G$ and $G'$ are groups and $\phi: G\to G'$ is a homomorphism, then
$G/\Ker\phi\cong \phi(G)$.

{\bf Note: } The problem asked to complete the sentence: `Let $\phi: G\to G'$ be a {\bf surjective } homomorphism' 
This is because in Fall 07 I reserved the term 
{\it Fundamental theorem of homomorphisms } for the special case dealing with surjective homomorphisms.
The above answer corresponds to the general form of FTH, as defined this semester.

\skv
(c) Let $\dbR^*$ be the group of nonzero real numbers (with respect to multiplication).
Find the kernel and the range for each of the following homomorphisms:

\sk $\psi: \dbR^*\to\dbR^*$ given by $\psi(x)=x^{-1}$

\sk $\theta: \dbR^*\to\dbR^*$ given by $\theta(x)=x^{2}$

\bf{Answer: }\rm $\Ker\psi=\{1\}$; $\quad\psi(\dbR^*)=\dbR^*$;
$\quad\Ker\theta=\{\pm 1\}$;  $\quad\theta(\dbR^*)=\dbR_{>0}$ (positive reals).
\skv

\bf{3. }\rm (a) See Problem\#2(a) in Homework\#6.
\skv
(b) Let $G$ be an \bf{abelian }\rm group. Let $H$ and $K$ be subgroups
of $G$, and let $HK$ be the product of $H$ and $K$ as subsets of $G$, that is,
$$HK=\{x\in G: x=hk \mbox{ for some }h\in H\mbox{ and }k\in K\}.$$
Prove that $HK$ is a subgroup of $G$.
\skv
{\bf Solution: } (i) Does $HK$ contain identity element?

Since $H$ and $K$ are subgroups, $e\in H$ and $e\in K$.
Therefore, $e=e\cdot e\in HK$.
\skv

(ii) Is $HK$ closed under group operation?

Let $x,y\in HK$. Then $x=h_1k_1$ and $y=h_2 k_2$ for some $h_1,h_2\in H$ and $k_1,k_2\in K$.
Then $xy=h_1 k_1 h_2 k_2=(h_1 h_2) (k_1 k_2)$ since $G$ is abelian.

Since $H$ and $K$ are closed under group operation, $h_1 h_2\in H$ and $k_1 k_2\in K$, and thus
$xy=(h_1 h_2) (k_1 k_2)\in HK$.
\skv

(iii) Is $HK$ closed under inversion?
\skv 
Let $x\in HK$, so $x=hk$ for some $h\in H$ and $k\in K$. By the reverse order formula,
$x^{-1}=k^{-1}h^{-1}$, and since $G$ is abelian, $k^{-1}h^{-1}=h^{-1}k^{-1}$.
Since $H$ and $K$ are closed under inversion, $h^{-1}\in H$ and $k^{-1}\in K$,
and therefore $x=h^{-1}k^{-1}\in HK$.
\skv

\bf{4. }\rm (a) Let $G$ be a finite group and $g\in G$.
Prove that $o(g)$ divides $|G|$.
\skv
{\bf Solution: } By one of the definitions of the order, $o(g)$ is equal to $|\la g\ra|$
(where $\la g\ra$ is the cyclic subgroup generated by $g$). By Lagrange
theorem $|\la g\ra|$ divides $|G|$, so $o(g)$ divides $|G|$.
\skv
(b) Let $G$ be a finite group, and let $n=|G|$. Prove that
$g^n=e$ for any $g\in G$.
\skv
{\bf Solution: } Let $m=o(g)$. Then $g^m=e$ by the other definition of the order.
On the other hand, by part (a) $m$ divides $n=|G|$, so $n=mk$ 
for some $k\in\dbZ$. Therefore,
$g^m=g^{nk}=(g^n)^k=e^k=e.$
\skv
(c) See Problem\#8 on Homework\#9.
\skv
\bf{5. }\rm In this problem $S_n$ denotes the permutation group
of the set $\{1,2,\ldots, n\}$.
\skv
(a) (3 pts) Let $f\in S_n$ be a cycle of length $k$. Show
by explicit formula that $f$ is a product of $k-1$ transpositions
(no explanations needed):
\skv
{\bf Answer: } $(i_1, i_2,\ldots, i_k) = (i_1, i_k) (i_1, i_{k-1})\ldots (i_1,i_2)$.
\skv
(b) (1 pt) Let $f\in S_n$, and suppose $f=f_1 f_2\ldots f_t$
where $f_1, f_2,\ldots, f_t$ are disjoint cycles and $f_i$ has length $k_i$
for each $i$. State the formula for the \bf{order }\rm of $f$ in terms of
$k_1,\ldots, k_t$ (no proof required).
\bf{Answer: }\rm $o(f)=LCM(k_1,\ldots, k_t)$.
\skv
(c) Let $f\in S_n$ be an element of \it{odd order. }\rm Prove that
$f$ is an \it{even permutation. }\rm
\skv
{\bf Solution: } Write $f$ as a product of disjoint cycles: $f=f_1 f_2\ldots f_t$.
Let $k_i$ be the length of $f_i$ for $1\leq i\leq t$.
Then $o(f)=LCM(k_1,\ldots, k_t)$.
\skv
We are given that $o(f)$ is odd, so $LCM(k_1,\ldots, k_t)$ is odd,
which means that each $k_i$ is odd (if at least one of the $k_i$'s
was even, the least common multiple would have been even as well).

Thus each $f_i$ has odd length $k_i$, so $f_i$ is a product of
an even number of transpositions (namely $k_i-1$). Thus, each $f_i$ 
is an even permutation. Since product of even permutations is even,
we get that $f=f_1\ldots f_t$ is also an even permutation.
\skv
\bf{6. }\rm (a) Give the definition of a normal subgroup (given in this course):
\skv
{\bf Answer: } If $G$ is a group, a subgroup $H$ of $G$ is called normal if 
$gH=Hg$ for any $g\in G$.
\skv
(b) Let $G$ be a group, and let $H$ be a normal subgroup of $G$.
Prove directly from the definition of a normal subgroup that for any $g\in G$
and $h\in H$ we have $ghg^{-1}\in H.$
\skv
{\bf Proof: } Let $g\in G$ and $h\in H$. Then $gh\in gH=Hg$ since $H$ is normal.
Therefore, there exists $h'\in H$ such that $gh=h'g$. Multiplying by $g^{-1}$
on the right, we get $ghg^{-1}=h'\in H$.
\skv
(c) (4 pts) Let $G$ be a group, and let $H$ be a normal subgroup of $G$
such that $H$ has \it{order $2$ }\rm. Prove that
$H$ is a subset of $Z(G)$, where $Z(G)$ is the center of $G$. Recall that
$Z(G)=\{g\in G: gx=xg \mbox{ for any }x\in G\}.$
\skv
{\bf Solution: } Since $|H|=2$, we have $H=\{e,h\}$ for some non-identity element $h$.
We know that $e\in Z(G)$ since $ex=xe=x$ for any $x\in G$ by axiom (G2),
so we only need to show that $h\in Z(G)$. We will prove the latter
using normality criterion (alternatively, we could use the definition of normal subgroup).

Take any $g\in G$. By normality criterion (in fact by part (b))
$ghg^{-1}\in H$. Thus, either $ghg^{-1}=e$ or $ghg^{-1}=h$ (since $H$ only has
two elements: $h$ and $e$). But if $ghg^{-1}=e$, then 
$h=g^{-1}eg=e$, contrary to the choice of $h$. Therefore,
we must have $ghg^{-1}=h$, so $gh=hg$.
\vskip .05cm
So, we showed that $h$ commutes with every element of $G$, which means that
$h\in Z(G)$.
\skv

\bf{7. }\rm (a) Let $G=\dbZ_{16}$ (with addition), and let
$H=\langle[4]\rangle$, the cyclic subgroup generated by $[4]$. Describe
left cosets with respect to $H$: state the number
of distinct cosets and list elements of each coset.

{\bf Answer: } There are four cosets: 
$H=[0]+H=\{[0],[4],[8],[12]\}$, 

$[1]+H=\{[1],[5],[9],[13]\}$, \sk $[2]+H=\{[2],[6],[10],[14]\}$ and 

$[3]+H=\{[3],[7],[11],[15]\}$.
\skv
(b) (2 pts) Let $G$ be a group, and $H$ a normal subgroup of $G$.
Define the quotient group $G/H$ by answering the following questions:

(i) what are the elements of $G/H$?

{\bf Answer: } Cosets of the form $gH$ (without repetitions)
\skv
(ii) how is the group operation on $G/H$ defined?

{\bf Answer: } $gH\cdot kH=gk H$. Equivalently,
$gH\cdot kH$ is the product of $gH$ and $kH$ as subsets of $G$.
\skv
(iii) which element of $G/H$ is the identity element?

{\bf Answer: } The coset $H=eH$.
\skv

(c) Let $G=\dbR$ be the group of real numbers (with addition), 
let $H=2\dbZ$ (even integers), and consider the quotient group 
$G/H=\dbR/2\dbZ.$ Find all elements of $G/H$ which have order $12$.
\skv
{\bf Answer: } Four elements: $\frac{1}{6}+2\dbZ$, $\frac{5}{6}+2\dbZ$, $\frac{7}{6}+2\dbZ$ and $\frac{11}{6}+2\dbZ$.
Solution: virtually identical to that of Problem\#4(b) on Homework\#11 
(multiply everything by $2$ in that solution).
\skv
\bf{8. }\rm (a) This is a special case of Problem\#7(b) on Homework\#2 with $d=1$.
\skv
(b) Proved in Lecture~7 (right before the Chinese Remainder Theorem).
\skv
\bf{9. }\rm (a) Give the definition of an ideal. 
\skv
{\bf Answer: } If $R$ is a commutative ring, a subset $I$ of $R$ is called an ideal if

\sk (i) $I$ is a subgroup of $(R,+)$

\sk (ii) For any $x\in I$ and $r\in R$ we have $xr\in I$.
\skv
%\vskip 3cm
(b) See Problem\#1(a) on Homework\#12.
\skv
(c) Let $R=\dbZ[x]$, the ring of polynomials with integer coefficients.
Let $I$ be an ideal of $R$ containing $1+x^3$ and $x^2$. Prove that $I$ contains $1$
(and hence $I=R$ by part (b)).
\skv
{\bf Solution: } Since $I$ is an ideal and $x^2\in I$, we also have $-x^3=x^2\cdot (-x)\in I$
by product absorption. Thus, $1+x^3\in I$ and $-x^3\in I$, and therefore $1=(1+x^3)+(-x^3)\in I$
since $I$ is a subgroup with respect to addition.
\skv
(d) Once again, let $R=\dbZ[x]$, and let $S$ be the minimal subring of $R$ containing 
$1$, $\,\,\,1+x^3$ and $x^2$.

\sk (i) Prove that $x^k\in S$ for any integer $k\geq 2$.

{\bf Solution: } Since $1\in S$ and $1+x^3\in S$, we have
$x^3=(1+x^3)-1\in S$. After that we argue as in Problem\#8(b) on Homework\#11.
\skv
\sk (ii) Prove that $x\not \in S$.

{\bf Solution: } Let $S_1=\{a_0+a_2 x^2+ a_3 x^3+\ldots + a_n x_n : \mbox{ each }a_i\in\dbZ\}$.
Then $S_1$ is a subring (direct computation), and clearly $S_1$ contains $1, x^2$ and $1+x^3$.
Since $S$ is the {\bf minimal } subring of $R$ containing $1,\,\, x^2$ and $1+x^3$, it follows that
$S\subseteq S_1$ (in fact, $S$ will be equal to $S_1$, but we do not need this
for our problem). Since $x$ does not belong to $S_1$ (by definition of $S_1$),
$x$ does not belong to $S$ either.
\skv
{\bf Alternative proof (somewhat informal): } By definition, $S$ is the set of all polynomials
which can be obtained from $1, x^2$ and $1+x^3$ using addition, additive inversion and multiplication.

If $f$ and $g$ are two polynomials, and both $f$ and $g$ have no linear term (that is,
the coefficient in front of $x$ is $0$), then each of the polynomials $f+g, -f$ and $fg$
has no linear term either: for $f+g$ and $-f$ this is clear, and for $fg$ this follows from
the fact that the only way to get $x$ as a product $x^k\cdot x^l$ with $k,l\geq 0$
is $x=x\cdot 1$ or $x=1\cdot x$. Since we start with three polynomials $1, x^2$ and $1+x^3$
with no linear term, the same is going to be true for any polynomial obtained from them
using addition, additive inversion and multiplication. Thus, any element of $S$
will have no linear term, so $x\not\in S$.
\skv
\skv

\bf{10. }\rm BONUS In our discussion of cosets, we occasionally
made use of the following fact: if $G$ is a group and $H$ is a subgroup of $G$,
then the number of left cosets with respect to $H$ is equal to
the number of right cosets. This is clear if $G$ is finite (in which case
the number of left or right cosets is simply $|G|/|H|$), but not in general.
The goal of this problem is to prove equality of the number of left and right
cosets, at least in the case when the number of left cosets is finite.
\skv

{\bf Solution: } This is somewhat sketchy. A few details are left for you
to check.

Consider the mapping $\phi:G\to G$ given by $\phi(g)=g^{-1}$.
Then $\phi$ is bijective (check details !!!).

\sk
If $X$ is a subset of $G$, we define $\phi(X)=\{\phi(x): x\in G\}$. 
We claim that $$\phi(gH)=Hg^{-1} \mbox{ for any }g\in G. \eqno (***)$$
\skv
Indeed, $\phi(gH)=\{(gh)^{-1}: h\in H\}=\{h^{-1} g^{-1}: h\in H\}$.
If $h$ runs over all elements of $H$, then $h^{-1}$ runs over all elements
of $H$ as well since $H$ is a subgroup (check details). Therefore,
$\{h^{-1} g^{-1}: h\in H\}=\{h g^{-1}: h\in H\}=H g^{-1}$, and thus
we proved (***).
\skv
Formula (***) implies that $\phi$ sends every left coset with respect to $H$
to some right coset; furthermore, every right coset is equal to $\phi(gH)$
for some $g\in G$ since $\phi$ is surjective (check details). 
Finally, distinct left cosets map to distinct right cosets since $\phi$ is injective (check details). Thus, $\phi$ establishes a bijection between left cosets and right cosets, and therefore the number
of left cosets equals the number of right cosets.
\end{document}