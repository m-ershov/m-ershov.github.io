\documentclass[12pt]{article}

\usepackage{amsmath}
\usepackage{amssymb}
\usepackage{amsthm}
%\usepackage{psfig}

\begin{document}
\baselineskip=15pt
\textheight=8.4in
\parindent=0pt
\def\sk {\hskip .5cm}
\def\skv {\vskip .12cm}
\def\cos {\mbox{cos}}
\def\sin {\mbox{sin}}
\def\tan {\mbox{tan}}
\def\intl{\int\limits}
\def\lm{\lim\limits}
\newcommand{\frc}{\displaystyle\frac}
\def\xbf{{\mathbf x}}
\def\fbf{{\mathbf f}}
\def\gbf{{\mathbf g}}

\def\Ker{{\rm Ker\,}}
\def\phi{\varphi}

\def\dbA{{\mathbb A}}
\def\dbB{{\mathbb B}}
\def\dbC{{\mathbb C}}
\def\dbD{{\mathbb D}}
\def\dbE{{\mathbb E}}
\def\dbF{{\mathbb F}}
\def\dbG{{\mathbb G}}
\def\dbH{{\mathbb H}}
\def\dbI{{\mathbb I}}
\def\dbJ{{\mathbb J}}
\def\dbK{{\mathbb K}}
\def\dbL{{\mathbb L}}
\def\dbM{{\mathbb M}}
\def\dbN{{\mathbb N}}
\def\dbO{{\mathbb O}}
\def\dbP{{\mathbb P}}
\def\dbQ{{\mathbb Q}}
\def\dbR{{\mathbb R}}
\def\dbS{{\mathbb S}}
\def\dbT{{\mathbb T}}
\def\dbU{{\mathbb U}}
\def\dbV{{\mathbb V}}
\def\dbW{{\mathbb W}}
\def\dbX{{\mathbb X}}
\def\dbY{{\mathbb Y}}
\def\dbZ{{\mathbb Z}}

\def\Aut{{\rm Aut}}

\def\la{{\langle}}
\def\ra{{\rangle}}

\bf\centerline{Math 8851. Linear groups and Expander Graphs. }
\vskip .1cm

\bf\centerline{Suggested problems (through Lecture 11.)}\rm
\vskip .2cm
{\bf Problem 1.1.} Prove that if a finitely generated group $G$ is linear over a field of characteristic zero,
then it is actually linear over $\mathbb C$. (We will probably discuss this problem in class shortly).
\vskip .1cm
{\bf Problem 1.2.} Prove that the symmetric group $S_n$ cannot be embedded in $GL_{n-2}(\mathbb C)$.
\vskip .1cm
{\bf Problem 1.3.} Work out the details of the counterexample to Jordan's theorem in characteristic $p>0$ described in Lecture~1. 
\vskip .1cm
{\bf Problem 4.1.} Let $R$ be a finitely generated domain of characteristic $0$ and $d\geq 2$ an integer.
As we proved in class, the group $GL_d(R)$ (and any of its subgroups) is virtually residually-$p$
for every prime $p$ outside certain finite set $B(R)$. For each of the rings 
$R=\dbZ,\, \dbZ[1/2]$ and $\dbZ[(1+\sqrt{-3})/2]$ do the following:
\begin{itemize}
\item[(a)] Find the minimal possible set of bad primes $B(R)$, that is, the set of all primes $p$
for which $GL_d(R)$ is not virtually residually-$p$.
\item[(b)] For each $p\not\in B(R)$, give an explicit estimate on the index
of a residually-$p$ subgroup of $GL_d(R)$ which is guaranteed to be residually-$p$.
\end{itemize}
\vskip .1cm
{\bf Problem 5.1.} Recall Burnside's irreducibility criterion: if $F$ is an algebraically
closed field, then a subgroup $G\subseteq GL_n(F)$ is irreducible if and only if
$FG=Mat_n(F)$, that is, $G$ spans $Mat_n(F)$ as $F$-vector space.
Deduce Burnside's irreducibility criterion from Schur's Lemma
and the following theorem, called Jacobson's Density Theorem (it is directly related
to the material discussed in Algebra-III, but may not have been stated there explicitly).
\vskip .1cm
{\bf Jacobson's Density Theorem:} \it Let $R$ be a ring with $1$ and $M$ an irreducible 
left $R$-module. Let $S=End_R(M)$, and note that $M$ is naturally a left $S$-module.
Prove that for any $a\in End_S(M)$ and any finite set of elements 
$m_1,\ldots, m_d\in M$ there exists $r\in R$ such that $a m_i=r m_i$ for $1\leq i\leq d$.
\rm
\vskip .1cm
{\bf Problem 5.2.} Let $F$ be a field which is not necessarily algebraically closed.
Prove that Burnside's irreducibility criterion holds over $F$ if and only if
there are no finite-dimensional division algebras over $F$ except $F$ itself.
Do there exist such fields which are not algebraically closed?


\vskip .1cm
{\bf Problem 5.3.} Prove that a finitely generated torsion nilpotent group is finite.
\vskip .1cm
In Problems 6.1-6.3 all groups under consideration are subgroups of $GL_n(F)$ (with $F$ algebraically closed) with Zariski topology.
\vskip .1cm
{\bf Problem 6.1.} Prove that $G$ is connected if and only if $G$ is irreducible 
(as topological space). By definition, a topological space is irreducible if it cannot be written as a union of two proper (not necessarily disjoint) closed subsets. 
This has nothing to do with irreducibility of the action of $G$ on $V$.

{\bf Oultine:} The backward direction is obvious. For the forward direction,
assume that $G$ is connected. Since $G$ is a Noetherian topological space, it is
the union of finitely many irreducible components $G_1,\ldots, G_k$ (by definition
this means that $G_i$'s are closed irreducible subsets of $G$ which do not contain
each other, and it is known that $G_i$'s are unique up to permutation). Show that $G$ naturally acts on the set $\{G_1,\ldots, G_k\}$ by left multiplication and that the stabilizer of $G_1$ is a closed subgroup of finite index in $G$. Then deduce that $G_1=G$.
\vskip .1cm
{\bf Problem 6.2.} Prove that if $G$ and $H$ are connected subgroups, then 
the set $GH=\{gh: g\in G, h\in H\}$ is connected. Deduce that the subgroup
generated by a family of connected subgroups is connected.
\vskip .1cm
{\bf Problem 6.3.} Let $H\subseteq G$ be subgroups of $GL_n(F)$, and let
$\overline H$ be the Zariski closure of $H$ in $GL_n(F)$. Note that $\overline H\cap G$
is the Zariski closure of $H$ in $G$. Prove that
\begin{itemize}
\item[(i)] $\overline H$ is a subgroup;
\item[(ii)] if $H$ is solvable of length $k$, then $\overline H$ is also solvable
of length $k$;
\item[(iii)] if $H$ is connected, then $\overline H\cap G$ is connected;
\item[(iv)] if $H$ is normal in $G$, then  $\overline H\cap G$ is normal in $G$.
\end{itemize}
\vskip .1cm
Note that using Problems 6.1-6.3 one can justify the definition of the solvable radical
of an algebraic group from Lecture~6. Recall that given an algebraic group $G$, we defined $R(G)$
as the subgroup generated by all connected solvable normal subgroups of $G$. We claimed
that $R(G)$ is itself connected, solvable and normal, and moreover that $R(G)$ is algebraic
(that is, Zariski-closed). 

First note $R(G)$ is normal by construction and connected by Problem 6.2. To prove that
it is solvable we can appeal to the following theorem of Zassenhaus:
\vskip .1cm
{\bf Zassenhaus Theorem: }\it If $S$ is solvable subgroup of $GL_n(F)$, then the solvability
length of $S$ is bounded above by a function of $n$.\rm
\vskip .1cm
However, there is a way around it. First, by Problem~6.3(i),(iii) and (iv), we can replace
every subgroup in the generating set for $R(G)$ by its Zariski closure (which is contained
in $G$ since $G$ is algebraic), so $R(G)$ is equal to the subgroup generated by all {\bf algebraic} connected solvable normal subgroups of $G$. Next we can use a standard result
from dimension theory of affine varieties:
\vskip .1cm
{\bf Dimension Theorem: }\it If $X$ and $Y$ are irreducible affine varieties with $X\subseteq Y$
and $X\neq Y$, then $\dim X<\dim Y$.\rm
\vskip .1cm
Algebraic subgroups of $GL_n(F)$ can naturally be considered as affine varieties
of dimension $\leq n^2$. So by Dimension Theorem and Problems 6.1, 6.2 and 6.3(i)(iii), 
if $A$ and $B$ are distinct algebraic connected solvable normal subgroups of $G$, then $\overline{AB}$ is also an algebraic connected solvable normal subgroup of $G$ of dimension strictly larger than $\max\{\dim A, \dim B\}$. Thus, if we 
take $R$ to be an algebraic connected solvable normal subgroups of $G$ of maximal possible
dimension, it must contain any other group with this property and thus must equal $R(G)$.
This also shows that $R(G)$ is algebraic.
\vskip .2cm
Using Zassenhaus Theorem and repeating the above reasoning, one can show that every linear (not necessarily algebraic) group contains the largest normal solvable subgroup $Solv(G)$ (which contains any other subgroup with this property). One can the define $R(G)$ as the connected component of identity in $Solv(G)$. Note that $Solv(G)$ is sometimes also called the solvable
radical of $G$.

\vskip .1cm
{\bf Problem 7.1.} Let $V$ be a finite-dimensional vector space and $G\subseteq GL(V)$.
Prove that if $\chi_1,\ldots,\chi_m$ are distinct characters of $G$ and 
$V_{\chi_1},\ldots, V_{\chi_m}$ the corresponding weight subspaces, then the sum
$V_{\chi_1}+\ldots + V_{\chi_m}$ is direct.
\vskip .1cm
{\bf Problem 7.2.} Prove that if $G\subseteq GL(V)$ is a connected subgroup (with 
respect to Zariski topology), then $[G,G]$ is also connected. {\bf Hint:} First prove that for a fixed $g\in G$,
the map $x\mapsto xgx^{-1}$ from $G$ to $G$ is continuous and deduce that the conjugacy
class of any element of $G$ is connected.
\vskip .1cm
{\bf Problem 7.3.} By a theorem of Maltsev stated at the end of Lecture 7, any solvable
linear group $G\subseteq GL_n(F)$, with $F$ algebraically closed, has a normal triangularizable subgroup of finite index $\leq f(n)$ for some function $f:\mathbb N\to \mathbb N$. Deduce the Zassenhaus theorem stated above from  Maltsev's theorem.
\vskip .1cm
{\bf Problem 7.4.} Prove the following theorem of Platonov: if $F$ is an algebraically
closed field of characteristic zero, then any virtually solvable subgroup $G\subseteq GL_n(F)$ has a normal triangularizable subgroup of finite index bounded by a function
of $n$. {\bf Note:} The proof of this theorem is given, for instance, in Appendix~B
of the following paper: 

http://arxiv.org/pdf/1005.1881v1.pdf

which by the way is directly related to part 3 of our class. The proof given there
uses the fact that a reductive algebraic group is a torus, which is a basic result
in the theory of algebraic groups, but is not so easy to prove from scratch. However,
one can avoid using this fact by adapting our proof of Lie-Kolchin theorem. 
\vskip .1cm
{\bf Problem 8.1.} Use Problem 7.4 to prove that in characteristic zero
Tits alternative holds true for non-finitely generated groups. Also give an example
showing that in positive characteristic zero Tits alternative is false (in general) 
for non-finitely generated groups.
\vskip .1cm
{\bf Problem 8.2.} Prove the (full version of) Claim 8.1 (in class
we made an additional assumption that $g$ is diagonalizable).
\vskip .1cm
{\bf Problem 8.3.} Prove Claim 8.2.
\vskip .1cm
{\bf Problem 8.4.} Let $g=\small \begin{pmatrix} 2 & 0\\ 0& 1/2 \end{pmatrix}$.
Find $f\in SL_2(\mathbb Q)$ and $k\in\dbN$ such that $g^k$ and $fg^kf^{-1}$
generate a free subgroup. Then do the same for 
$g=\small \begin{pmatrix} 2 & 1\\ 1& 1 \end{pmatrix}$ with an additional assumption
that $f\in SL_2(\mathbb Z)$.
\vskip .1cm
{\bf Problem 9.1.} Prove Observation 9.0.
\vskip .1cm
{\bf Problem 9.2.} Let $K$ be a local field, $n$ a positive integer, and let
$Dom_n(K)$ be the set of all $n\times n$ matrices over $K$ with dominant eigenvalue.
Prove that $Dom_n(K)$ is open in the field topology on $Mat_n(K)$ (the field topology
we mean the product topology coming from identification of $Mat_n(K)$ with $K^{n^2}$). 
\vskip .1cm
{\bf Problem 10.1.} Prove that if a linear group $G$ is Zariski-connected, then
$G\times G$ is also Zariski-connected.
\vskip .1cm
{\bf Problem 11.1.} Prove that if $K/E$ is an arbitrary field extension and $n$
a positive integer, then Zariski topology on $E^n$ is induced from the Zariski
topology on $K^n$ under the natural embedding of $E^n$ into $K^n$. 
\vskip .1cm
{\bf Problem 11.2.} Prove Lemma 11.3: If $E$ is a finitely generated field, $n$
a positive integer, $A\in Mat_n(E)$ and $\zeta$ is an eigenvalue of $A$ which
is a root of unity (possibly $\zeta\not\in E$), then $\zeta^N=1$ for some $N$ which depends only on $E$ and $n$.
{\bf Hint:} This can be proved in three steps as follows:
\begin{itemize}
\item[(i)] Reduce the problem to the case when $E$ is purely transcendental over
the prime field $E_0$ (we used a similar trick in one of the previous lectures).
\item[(ii)] Assuming that $E$ is purely transcendental over $E_0$, prove
that $\deg_{E_0}(\zeta)\leq n$.
\item[(iii)] Prove that the (multiplicative) order of $\zeta$
is bounded by a function of its degree over $E_0$. Consider separately
the cases of zero and positive characteristic.
\end{itemize}
\vskip .1cm
The following two problems fill the missing details in the proof of Local Field Lemma (Lemma 11.2).
\vskip .1cm
{\bf Problem 11.3.} Let $\zeta\in \dbC$ be an algebraic integer such that
$|\zeta|=1$ and $\zeta$ is not a root of unity, and let $m(x)$ be the minimal polynomial
of $\zeta$ over $\dbQ$. Prove that $m(x)$ has a root $\eta$ with $|\eta|\neq 1$.  
\vskip .1cm
{\bf Problem 11.4.} Let $E$ be a finitely generated field of characteristic
zero, $\zeta\in E$, $m(x)$ the minimal polynomial of $\zeta$ over $\dbQ$ and
$\eta$ any complex root of $m(x)$. Prove that there exists an embedding $\iota:E\to\dbC$
such that $\iota(\zeta)=\eta$.
\end{document}
