\documentclass[12pt]{article}

\usepackage{amsmath}
\usepackage{amssymb}
\usepackage{amsthm}
%\usepackage{psfig}

\begin{document}
\baselineskip=15pt
\textheight=9in
\parindent=0pt
\def\sk {\hskip .5cm}
\def\skv {\vskip .07cm}
\def\cos {\mbox{cos}}
\def\sin {\mbox{sin}}
\def\tan {\mbox{tan}}
\def\intl{\int\limits}
\def\lm{\lim\limits}
\newcommand{\frc}{\displaystyle\frac}
\def\xbf{{\mathbf x}}
\def\fbf{{\mathbf f}}
\def\gbf{{\mathbf g}}

\def\Ker{{\rm Ker\,}}
\def\Gal{{\rm Gal\,}}
\def\phi{\varphi}

\def\dbA{{\mathbb A}}
\def\dbB{{\mathbb B}}
\def\dbC{{\mathbb C}}
\def\dbD{{\mathbb D}}
\def\dbE{{\mathbb E}}
\def\dbF{{\mathbb F}}
\def\dbG{{\mathbb G}}
\def\dbH{{\mathbb H}}
\def\dbI{{\mathbb I}}
\def\dbJ{{\mathbb J}}
\def\dbK{{\mathbb K}}
\def\dbL{{\mathbb L}}
\def\dbM{{\mathbb M}}
\def\dbN{{\mathbb N}}
\def\dbO{{\mathbb O}}
\def\dbP{{\mathbb P}}
\def\dbQ{{\mathbb Q}}
\def\dbR{{\mathbb R}}
\def\dbS{{\mathbb S}}
\def\dbT{{\mathbb T}}
\def\dbU{{\mathbb U}}
\def\dbV{{\mathbb V}}
\def\dbW{{\mathbb W}}
\def\dbX{{\mathbb X}}
\def\dbY{{\mathbb Y}}
\def\dbZ{{\mathbb Z}}

\def\Aut{{\rm Aut}}
\def\deg{{\rm deg}}

\def\la{{\langle}}
\def\ra{{\rangle}}

\bf\centerline{Homework Assignment \# 10. }\rm

{\bf Plan for the next week:} Tuesday (April 17) -- proof of Hilbert's Nullstellensatz;
Thursday (April 19) -- dimension theory of affine varieites. Good references on
commutative algebra and algebraic geometry freely available online are notes by 
J. Milne

\centerline{http://www.jmilne.org/math/xnotes/CA.pdf} 

and 

\centerline{http://www.jmilne.org/math/CourseNotes/AG.pdf}

\skv
\skv
\bf\centerline{Problems, to be submitted by Thu, April 19th. }\rm
\skv
{\bf Problem 1:} Let $k$ be a field and $R=k[x_1,\ldots, x_n]$ for some $n\geq 1$.
\begin{itemize}
\item[(a)] Prove that if $\{I_{\alpha}\}$ is any collection of ideals of $R$, then
$\cap Z(I_{\alpha})=Z(\cup I_{\alpha})=Z(\sum I_{\alpha})$.
\item[(b)] Prove that if $I$ and $J$ are ideals of $R$, then
$Z(I)\cup Z(J)=Z(I\cap J)=Z(IJ)$.
\end{itemize}
\skv
{\bf Problem 2:} Let $k$ be an algebraically closed field.
An algebraic set $V\subseteq k^n$ is called \underline{irreducible} 
if $V\neq\emptyset$ and $V$ cannot be written as the union $V=V_1\cup V_2$ where
$V_1$ and $V_2$ are both algebraic, with $V_1\neq V$ and $V_2\neq V$.
\begin{itemize}
\item[(a)] Prove that $V$ is irreducible if and only if its vanishing ideal
$I(V)$ is prime.
\item[(b)] We will prove in class next week that any algebraic set $V$
can be uniquely written as a union of finitely many algebraic subsets 
$V=\cup_{i=1}^k V_i$ where $V_i$'s are irreducible and do not contain
each other. Such $V_i$'s are called \underline{irreducible components of $V$}.
Let $$V=Z(xy-y, x^2z-z)\subset k^3,$$ the set of common zeroes of $xy-y$
and $x^2z-z$. Find irreducible components of $V$ and their vanishing ideals.
The answer will depend on $char(k)$.
\skv


{\bf Problem 3:} DF, Problem 17 on pages 582-583 (this time really assigned).
\skv
{\bf Problem 4:} This is a continuation of Problem~7 in HW\#9.
\begin{itemize}
\item[(c)] Suppose $K/L$ and $L/F$ are both $p$-extensions, and
let $M$ be the Galois closure of $K$ over $F$ (note: we do
not know whether $K/F$ is Galois or not). Prove that
$M/F$ is also a $p$-extension. {\bf Hint:} first show that
$M/L$ is Galois.

\item[(d)] Now assume only that $L/F$ is a separable extension
with $[L:F]$ a power of $p$, and let $M$ be the Galois closure
of $L$ over $F$. Prove that $[M:F]$ need not be a power of $p$.
\end{itemize}

\skv
{\bf Problem 5:} Let $f(x)$ and $g(x)$ be irreducible polynomials in $\mathbb F_p[x]$
of the same degree and let $F=\mathbb F_p[x]/(f(x))$. Prove that $g(x)$ splits completely over $F$.
\skv
{\bf Problem 6:} Let $p$ be a prime, $n$ a positive integer and
$\Phi_n(x)=x^{p^n}-x\in \mathbb F_p[x]$. Prove that $\Phi_n(x)$ is equal to the product
of all monic irreducible polynomials in $\mathbb F_p[x]$ whose degrees divide $m$
(where each polynomial occurs with multiplicity one).
\skv
\skv
{\bf Problem 7:} (practice) Prove the following analogue of Kummer's theorem for abelian extensions:
Let $n\in\dbN$ and let $F$ be a field containing primitive $n^{\rm th}$ root of unity.
\begin{itemize}
\item[(a)] Let $K/F$ be a finite Galois extension such that $\Gal(K/F)$ is abelian
of exponent $n$. Then there exists $a_1,\ldots, a_t\in K$ s.t.
$K=F(\sqrt[n]{a_1},\ldots,\sqrt[n]{a_t})$, or more precisely, there exists
$\alpha_1,\ldots, \alpha_t\in K$ s.t.
$K=F(\alpha_1,\ldots,\alpha_t)$ and $\alpha_i^n\in F$ for all $i$.

\item[(b)] Conversely, suppose that $K=F(\sqrt[n]{a_1},\ldots,\sqrt[n]{a_t})$
for some $a_1,\ldots, a_t\in F$. Prove that $K/F$ is Galois, and $\Gal(K/F)$
is abelian of exponent $n$.
\end{itemize}


\end{itemize}
\end{document} 