\documentclass[12pt]{amsart}

\usepackage{amsmath}
\usepackage{amssymb}
\usepackage{amsthm}
\usepackage{hyperref}
\usepackage{url}
%\usepackage{psfig}

\begin{document}
\baselineskip=16pt
\textheight=9in
\parindent=0pt 
\def\sk {\hskip .5cm}
\def\skv {\vskip .12cm}
\def\cos {\mbox{cos}}
\def\sin {\mbox{sin}}
\def\tan {\mbox{tan}}
\def\intl{\int\limits}
\def\lm{\lim\limits}
\newcommand{\frc}{\displaystyle\frac}
\def\xbf{{\mathbf x}}
\def\fbf{{\mathbf f}}
\def\gbf{{\mathbf g}}

\def\dbA{{\mathbb A}}
\def\dbB{{\mathbb B}}
\def\dbC{{\mathbb C}}
\def\dbD{{\mathbb D}}
\def\dbE{{\mathbb E}}
\def\dbF{{\mathbb F}}
\def\dbG{{\mathbb G}}
\def\dbH{{\mathbb H}}
\def\dbI{{\mathbb I}}
\def\dbJ{{\mathbb J}}
\def\dbK{{\mathbb K}}
\def\dbL{{\mathbb L}}
\def\dbM{{\mathbb M}}
\def\dbN{{\mathbb N}}
\def\dbO{{\mathbb O}}
\def\dbP{{\mathbb P}}
\def\dbQ{{\mathbb Q}}
\def\dbR{{\mathbb R}}
\def\dbS{{\mathbb S}}
\def\dbT{{\mathbb T}}
\def\dbU{{\mathbb U}}
\def\dbV{{\mathbb V}}
\def\dbW{{\mathbb W}}
\def\dbX{{\mathbb X}}
\def\dbY{{\mathbb Y}}
\def\dbZ{{\mathbb Z}}

\def\la{{\langle}}
\def\ra{{\rangle}}
\def\Aut{{\rm Aut}}
\def\Tor{{\rm Tor}}
\def\Inn{{\rm Inn\,}}
\def\Ker{{\rm Ker\,}}
\def\Im{{\rm Im\,}}
\def\phi{{\varphi}}

\bf\centerline{Homework \#9}\rm
\vskip .1cm
{\bf Plan for next week:} Basic properties of modules (10.1-10.3). Start tensor products of modules (10.4). See also Lectures 1-3 at
\skv
\url{http://people.virginia.edu/~mve2x/7752_Spring2010/}
\skv


\vskip .1cm
\centerline{\bf Problems, to be submitted by 11:59pm on Sat, November 14th}
\vskip .1cm
\skv

{\bf 1.} 
\begin{itemize}
\item[(a)] Classify all abelian groups of order $360=2^3\cdot 3^2\cdot 5$ up to isomorphism. For each isomorphism type, state the corresponding elementary divisors form and invariant factors form.
\item[(b)] Let $n\in\dbN$, and decompose $n$ as a product of primes: $n=p_1^{\alpha_1}\ldots p_k^{\alpha_k}$. Find (with justification) the number of non-isomorphic abelian groups of order $n$. Express your answer in terms of the partition function $P$ (where $P(n)$ is the number of partitions of $n$).
\end{itemize}
\skv
{\bf 2.} Let $G$ be a finite abelian group. Prove that $G$ is cyclic if and only if $G$ does not contain a subgroup isomorphic to $B\oplus B$ for some non-trivial group $B$.
\skv
{\bf 3.} \skv Let $G$ be an abelian group (not necessarily finitely generated), and
let $\Tor(G)$ be the set of elements of finite order in $G$. 
\begin{itemize}
\item[(a)] Prove that $\Tor(G)$ is a subgroup. It is called the torsion subgroup of $G$.
\item[(b)] Prove that the quotient group $G/\Tor(G)$ is torsion-free, that is, $G/\Tor(G)$ has no elements of finite order
apart from the identity element.  
\item[(b)] For each prime $p$ let $\Tor_p(G)$ be the set of elements of order $p^k$ (with $k\geq 0$)
in $G$. Prove that each $\Tor_p(G)$ is a subgroup of $\Tor(G)$ and that $\Tor(G)=\bigoplus_p \Tor_p(G)$ where $p$ ranges over all primes.
\end{itemize}
{\bf Note:} Recall that the definition of internal direct sum (of a potentially infinite collection of subgroups) was given at the end of Lecture~19. In the case of abelian groups written additively, this definition can be rephrased as follows:

Let $\{A_i\}_{i\in I}$ be a family of subgroups of $A$. Then $A=\bigoplus_{i\in I} A_i$ if
\begin{itemize}
\item[(1)] $A=\la A_i: i\in I\ra$, that is (since $A$ is abelian), every $a\in A$ can be written as a {\bf finite} sum $a=a_1+\ldots+ a_m$
where each $a_k$ lies in $A_{i_k}$ for some $i_k\in I$ 
\item[(2)] for each $i\in I$ the intersection $A_i\cap \la A_j: j\neq i\ra$ is trivial. Since every element of $\la A_j: j\neq i\ra$
if a finite sum of elements of $\bigcup_{j\neq i}A_j$, this is the same as requiring that for any distinct indices $i,j_1,\ldots, j_m\in I$
the intersection $A_i\cap \la A_{j_1},\ldots, A_{j_m}\ra$ is trivial.
\end{itemize}
{\bf 4.} Let $X$ be a set. As in Lecture~21, define $FA(X)$ to be the group of all formal linear combinations $\sum\limits_{x\in X} \lambda_x x$
where each $\lambda_x\in\dbZ$ and only finitely many $\lambda_x$ are nonzero. Let $F(X)^{ab}=F(X)/[F(X),F(X)]$ be the abelianization of
$F(X)$, the free group on $X$.
\skv
In Lecture~22 we proved that $F(X)^{ab}\cong FA(X)$ by showing that both $F(X)^{ab}$ and  $FA(X)$ are free objects on $X$ in the category
of abelian groups (and using the uniqueness of a free object up to isomorphism). Give another proof of the isomorphism $F(X)^{ab}\cong FA(X)$
by constructing homomorphisms in both directions and showing that they are mutually inverse.
\skv
{\bf 5.} 
Let $p$ and $q$ be primes with $p<q$ and $q\equiv 1\mod p$, and let $G$ be a non-abelian group or order $pq$. Recall that such $G$ is unique up to isomorphism. Prove that $G$ has a presentation $$\la x,y\mid x^p=1, y^q=1, xyx^{-1}=y^a \ra$$ where $a$ is coprime to $q$ and the order of 
$[a]_q$ in $\dbZ_q^{\times}$ is equal to $p$.
\end{document}
