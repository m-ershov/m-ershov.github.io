\documentclass[12pt]{amsart}

\usepackage{amsmath}
\usepackage{amssymb}
\usepackage{amsthm}
%\usepackage{psfig}

\begin{document}
\baselineskip=16pt
\textheight=9in
\parindent=0pt 
\def\sk {\hskip .5cm}
\def\skv {\vskip .12cm}
\def\cos {\mbox{cos}}
\def\sin {\mbox{sin}}
\def\tan {\mbox{tan}}
\def\intl{\int\limits}
\def\lm{\lim\limits}
\newcommand{\frc}{\displaystyle\frac}
\def\xbf{{\mathbf x}}
\def\fbf{{\mathbf f}}
\def\gbf{{\mathbf g}}

\def\dbA{{\mathbb A}}
\def\dbB{{\mathbb B}}
\def\dbC{{\mathbb C}}
\def\dbD{{\mathbb D}}
\def\dbE{{\mathbb E}}
\def\dbF{{\mathbb F}}
\def\dbG{{\mathbb G}}
\def\dbH{{\mathbb H}}
\def\dbI{{\mathbb I}}
\def\dbJ{{\mathbb J}}
\def\dbK{{\mathbb K}}
\def\dbL{{\mathbb L}}
\def\dbM{{\mathbb M}}
\def\dbN{{\mathbb N}}
\def\dbO{{\mathbb O}}
\def\dbP{{\mathbb P}}
\def\dbQ{{\mathbb Q}}
\def\dbR{{\mathbb R}}
\def\dbS{{\mathbb S}}
\def\dbT{{\mathbb T}}
\def\dbU{{\mathbb U}}
\def\dbV{{\mathbb V}}
\def\dbW{{\mathbb W}}
\def\dbX{{\mathbb X}}
\def\dbY{{\mathbb Y}}
\def\dbZ{{\mathbb Z}}

\def\la{{\langle}}
\def\ra{{\rangle}}
\def\Aut{{\rm Aut}}
\def\Inn{{\rm Inn\,}}
\def\Ker{{\rm Ker\,}}
\def\Im{{\rm Im\,}}
\def\phi{{\varphi}}

\bf\centerline{Homework \#6}\rm
\vskip .1cm
{\bf Plan for next week:} Direct and semi-direct products and further applications of Sylow theorems (5.4, 5.5). Jordan-H\"older Theorem (3.4).

\vskip .1cm
\centerline{\bf Problems, to be submitted by 11:59pm on Thu, October 15th}
\vskip .1cm
\skv
\skv
{\bf 1.} Let $F$ be a finite field of order $q$.
\begin{itemize}
\item[(a)] Prove that $|GL_n(F)|=\prod_{k=0}^{n-1} (p^n-p^k)$. {\bf Note:} We outlined the argument in Lecture~14, so your main task is to carefully write down the details.
\item[(b)] Prove that $|SL_n(F)|=\frac{|GL_n(F)|}{q-1}$ for any $n\geq 2$.
\end{itemize}
{\bf 2.} 
\begin{itemize}
\item[(a)] Solve Problem 10 on page 117 in Dummit and Foote.
\item[(b)] Use Problem~10 to prove the index formula in Theorem~14.2 from class:
$$[G:K]=\sum_{x\in\Omega}[H:H\cap xKx^{-1}]$$ 
where $\Omega$ is a set of representatives of $(H,K)$-double cosets.
\end{itemize}
\skv
{\bf 3.} An action of a group $G$ on a set $X$ is called {\it transitive} if it has
just one orbit, that is, for any $x,y\in X$ there exists $g\in G$ with $g. x=y$.
\begin{itemize}
\item[(a)] Let $(G,X,.)$ be a group action. Prove that if $x,y\in X$ lie in the same orbit,
then their stabilizers $Stab_G(x)$ and $Stab_G(y)$ are conjugate, that is, there exists $g\in G$
with $g Stab_G(x) g^{-1} = Stab_G(y)$. 

\item[(b)] Suppose that $(G,X,.)$ is a transitive action and fix $x\in X$. Prove that
the kernel of this action is equal to $\bigcap\limits_{g\in G} g Stab_G(x) g^{-1}$

\item[(c)] Now suppose that $G$ and $X$ are both finite, $(G,X,.)$ is a transitive 
faithful action (where `faithful' means the kernel is trivial) and $G$ is abelian. 
Prove that for any $g\in G\setminus \{1\}$ the fixed set $Fix_X(g)$ is empty. 
Deduce that $|X|=|G|$. {\bf Hint:} Use (b).
\end{itemize}
\skv

{\bf 4.} Let $C$ be the cube in $\dbR^3$ whose vertices have coordinates 
$(\pm 1, \pm 1,\pm 1)$. Let $G$ be the group of rotations of $C$, that is rotations in $\dbR^3$ which preserve the cube (you may assume that $G$ is a group without proof). Let $X$ be the set of $4$ main diagonals of $C$ (diagonals
connecting the opposite vertices). Note that $G$ naturally acts on $X$ and therefore we have a homomorphism $\pi:G\to Sym(X)\cong S_4$. Prove that $\pi$ is an isomorphism.

{\bf Hint: } First show that $G$ acts transitively on the 8 vertices of $C$. Then show that the stabilizer of a fixed vertex had order $\geq 3$. This implies that $|G|\geq 24=|S_4|$. Finally, show that $\pi$ is injective (since $|G|\geq |S_4|$, this would force $\pi$ to be an isomorphism).
\skv
{\bf 5.} Let $G$ be a finite group, $P$ a Sylow $p$-subgroup of $G$ for some $p$
and $H$ a subgroup of $G$ such that $N_{G}(P)\subseteq H\subseteq G$.
Prove that $N_H(P)=N_G(P)$ and $[G:H]\equiv 1\mod p$.
\skv
{\bf 6.} Prove that a group of order $132=3\cdot 4\cdot 11$ has a normal Sylow $11$-subgroup. {\bf Hint:} first show that
$G$ has a normal Sylow $p$-subgroup for some $p\in {2,3,11}$.
\skv
{\bf 7.} Explicitly describe a Sylow $p$-subgroup of $S_{p^2}$. See next page for a hint.
\newpage
{\bf Hint for 7:} The largest power of $p$ dividing $(p^2)!$ is $p^{p+1}$. First find a subgroup of order $p^p$
in $S_{p^2}$ (this is quite easy) and then think how to enlarge it to a subgroup of order $p^{p+1}$. Proposition~10 on p. 125 of DF
is very relevant here.
\end{document}
