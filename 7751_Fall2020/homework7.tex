\documentclass[12pt]{amsart}

\usepackage{amsmath}
\usepackage{amssymb}
\usepackage{amsthm}
%\usepackage{psfig}

\begin{document}
\baselineskip=16pt
\textheight=9in
\parindent=0pt 
\def\sk {\hskip .5cm}
\def\skv {\vskip .12cm}
\def\cos {\mbox{cos}}
\def\sin {\mbox{sin}}
\def\tan {\mbox{tan}}
\def\intl{\int\limits}
\def\lm{\lim\limits}
\newcommand{\frc}{\displaystyle\frac}
\def\xbf{{\mathbf x}}
\def\fbf{{\mathbf f}}
\def\gbf{{\mathbf g}}

\def\dbA{{\mathbb A}}
\def\dbB{{\mathbb B}}
\def\dbC{{\mathbb C}}
\def\dbD{{\mathbb D}}
\def\dbE{{\mathbb E}}
\def\dbF{{\mathbb F}}
\def\dbG{{\mathbb G}}
\def\dbH{{\mathbb H}}
\def\dbI{{\mathbb I}}
\def\dbJ{{\mathbb J}}
\def\dbK{{\mathbb K}}
\def\dbL{{\mathbb L}}
\def\dbM{{\mathbb M}}
\def\dbN{{\mathbb N}}
\def\dbO{{\mathbb O}}
\def\dbP{{\mathbb P}}
\def\dbQ{{\mathbb Q}}
\def\dbR{{\mathbb R}}
\def\dbS{{\mathbb S}}
\def\dbT{{\mathbb T}}
\def\dbU{{\mathbb U}}
\def\dbV{{\mathbb V}}
\def\dbW{{\mathbb W}}
\def\dbX{{\mathbb X}}
\def\dbY{{\mathbb Y}}
\def\dbZ{{\mathbb Z}}

\def\la{{\langle}}
\def\ra{{\rangle}}
\def\Aut{{\rm Aut}}
\def\Inn{{\rm Inn\,}}
\def\Ker{{\rm Ker\,}}
\def\Im{{\rm Im\,}}
\def\phi{{\varphi}}

\bf\centerline{Homework \#7}\rm
\vskip .1cm
{\bf Plan for next week:} Simple groups and proof of simplicity of $A_n$ (4.6). Jordan-H\"older Theorem (3.4). Classification of finitely generated abelian groups with applications (5.2)

\vskip .1cm
\centerline{\bf Problems, to be submitted by 11:59pm on Thu, October 22nd}
\vskip .1cm
\skv
\skv
{\bf 1.} DF, Problem 19 on page 131. {\bf Note:} if $G$ is a finite group, $x\in G$ and $\mathcal K(x)$ is the conjugacy class of $x$ in $G$,
then $$|\mathcal K(x)|=[G:C_G(x)]$$ (this is just the orbit-stabilizer formula applied to the conjugation action of $G$ on itself). 
\skv
{\bf 2.} $\empty$
\begin{itemize}
\item[(a)] Prove Lemma 15.7 from class: Let $H,K$ be groups, let $\phi$ and $\psi$ be homomorphisms from $K$ to $\Aut(H)$, and assume that there exists $\theta\in \Aut(K)$ such that $\phi\circ\theta=\psi$. Prove that $H\rtimes_{\phi} K\cong H\rtimes_{\psi} K$.
\item[(b)] DF, Problem 6 on page 184. {\bf Note:} you will need to use a result from one of the first three homeworks.
\end{itemize}
\skv
{\bf 3.} The goal of this problem is to complete classification of groups of order 56 (we discussed the outline in Lecture~16).
\skv
\begin{itemize}
\item[(a)] Let $p$ be a prime, let $P$ be a finite abelian $p$-group, and let $H$ be a finite group with $p\nmid |H|$. Let $\phi_1$ and $\phi_2$ be homomorphisms from $H$ to $\Aut(P)$, and let $G_{i}=P\rtimes_{\phi} H$ for $i=1,2$. Suppose that $G_1\cong G_2$. Prove that
$\Ker(\phi_1)\cong \Ker(\phi_2)$. {\bf Hint:} Compute the centralizer of $P$ in $P\rtimes_{\phi} H$. 
\item[(b)] Let $Q$ be a group of order $8$, let $\Omega_{4}(Q)$ be the set of subgroups of $Q$ isomorphic to $\dbZ_4$, and let
$\Omega_{2,2}(Q)$ be the set of subgroups of $Q$ isomorphic to $\dbZ_2\oplus \dbZ_2$. Prove that $\Aut(Q)$ acts transitively 
on each of the sets $\Omega_{4}(Q)$ and $\Omega_{2,2}(Q)$. {\bf Note:} Most likely you will need to argue separately for each isomorphism class.
\item[(c)] Now justify the entries in the right columns of the table on page 2 of class notes of Lecture~16 (they describe the number of isomorphic classes of groups of order $56$ covered by the given case; you do not need to justify the entries for the number of homomorphisms). 
{\bf Hint:} most likely you will need to use both parts of Problem~2, 3(a), 3(b), and the fact that a homomorphism from a given group $G$
whose image has order $\leq 2$ is completely determined by its kernel.
\end{itemize}
\skv
{\bf 4.} Redo Problem~4 on the first midterm (unless you got full credit). If you did not get a correct answer on the midterm, I encourage you
to start with an easier problem -- find the number of ideals of $\dbZ$ which contain $12$ -- and then generalize.
\skv
{\bf 5}. Let $G$ be a group, let $H_1,\ldots, H_n$ be a finite collection of subgroups of $G$, and let $H_1\times \ldots \times H_n$ denote
their external direct product. Prove that the following three conditions are equivalent (note that (b) and (c)
by themselves consist of several ``subconditions''):
\begin{itemize}
\item[(a)] The map $\phi:H_1\times \ldots \times H_n\to G$ given by $$\phi((h_1,\ldots, h_n))=h_1 h_2\ldots h_n$$ is a group isomorphism (the product on the right-hand side is taken in $G$)  
\item[(b)] The following hold:
\begin{itemize}
\item[(i)] For any $i\neq j$ the subgroups $H_i$ and $H_j$ commute elementwise, 
\item[(ii)] For any $i$ we have $H_i\cap H_i '=\{1\}$ where $H_i ' =\la H_j: j\neq i\ra$, the subgroup generated by all $H_j$'s besides $H_i$ 
\item[(iii)] $G=\la H_j: 1\leq j\leq n\ra$  
\end{itemize}
\item[(c)] The following hold:
\begin{itemize}
\item[(i)] Each $H_i$ is normal in $G$ 
\item[(ii)] For any $i$ we have $H_i\cap H_i '=\{1\}$ where $H_i ' =\la H_j: j\neq i\ra$, the subgroup generated by all $H_j$'s besides $H_i$ 
\item[(iii)] $G=\la H_j: 1\leq j\leq n \ra$  
\end{itemize}
\end{itemize}
\end{document}
