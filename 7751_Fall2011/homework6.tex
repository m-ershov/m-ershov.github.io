\documentclass[12pt]{article}

\usepackage{amsmath}
\usepackage{amssymb}
\usepackage{amsthm}
%\usepackage{psfig}

\begin{document}
\baselineskip=16pt
\textheight=9in
\textwidth=6in
\parindent=0pt
\def\sk {\hskip .5cm}
\def\skv {\vskip .12cm}
\def\cos {\mbox{cos}}
\def\sin {\mbox{sin}}
\def\tan {\mbox{tan}}
\def\intl{\int\limits}
\def\lm{\lim\limits}
\newcommand{\frc}{\displaystyle\frac}
\def\xbf{{\mathbf x}}
\def\fbf{{\mathbf f}}
\def\gbf{{\mathbf g}}

\def\dbA{{\mathbb A}}
\def\dbB{{\mathbb B}}
\def\dbC{{\mathbb C}}
\def\dbD{{\mathbb D}}
\def\dbE{{\mathbb E}}
\def\dbF{{\mathbb F}}
\def\dbG{{\mathbb G}}
\def\dbH{{\mathbb H}}
\def\dbI{{\mathbb I}}
\def\dbJ{{\mathbb J}}
\def\dbK{{\mathbb K}}
\def\dbL{{\mathbb L}}
\def\dbM{{\mathbb M}}
\def\dbN{{\mathbb N}}
\def\dbO{{\mathbb O}}
\def\dbP{{\mathbb P}}
\def\dbQ{{\mathbb Q}}
\def\dbR{{\mathbb R}}
\def\dbS{{\mathbb S}}
\def\dbT{{\mathbb T}}
\def\dbU{{\mathbb U}}
\def\dbV{{\mathbb V}}
\def\dbW{{\mathbb W}}
\def\dbX{{\mathbb X}}
\def\dbY{{\mathbb Y}}
\def\dbZ{{\mathbb Z}}

\def\la{{\langle}}
\def\ra{{\rangle}}

\def\Aut{{\rm Aut}}
\def\Inn{{\rm Inn}}

\bf\centerline{Homework \#6. }\rm
\vskip .1cm
{\bf Plan for next week:} Free groups (6.3)

\vskip .1cm
\centerline{\bf Problems, to be submitted by Thursday, October, 13th}
\skv
{\bf 1.} (a) Classify all abelian groups of order $360=2^3\cdot 3^2\cdot 5$ up to isomorphism. For each isomorphism type, state the corresponding elementary divisors form and invariant factors form.

(b) Let $n\in\dbN$, and decompose $n$ as a product of primes: $n=p_1^{\alpha_1}\ldots p_k^{\alpha_k}$. Find the number of non-isomorphic abelian groups of order $n$. Express your answer in terms of the partition function.
\skv
{\bf 2.} Let $G$ be a finite abelian group. Prove that $G$ is cyclic if and only if $G$ does not contain a subgroup isomorphic to $B\oplus B$ for some non-trivial group $B$.
\skv
{\bf 3.} Let $G$ be an abelian group (not necessarily finitely generated), and
let $Tor(G)$ be the set of elements of finite order in $G$. Recall that
$Tor(G)$ is a subgroup of $G$ (since $G$ is abelian), called the torsion subgroup of $G$.

(a) Prove that the quotient group $G/Tor(G)$ is torsion-free (a group is called torsion-free
if it does not have non-identity elements of finite order).

(b) For each prime $p$ let $Tor_p(G)$ be the set of elements of order $p^k$ (with $k\geq 0$)
in $G$. Prove that each $Tor_p(G)$ is a subgroup of $Tor(G)$ and that $Tor(G)$
is a direct sum of these subgroups where $p$ runs over all primes.

\skv
{\bf 4.} Let $\Omega$ be a countable set (for simplicity you may assume that $\Omega=\dbZ$,
the integers). Let $S(\Omega)$ be the group of all permutations of $\Omega$.
A permutation $\sigma\in S(\Omega)$ is called {\it finitary} if it moves
only a finite number of points, that is, the set $\{i\in\Omega : \sigma(i)\neq i\}$
is finite. It is easy to see that finitary permutations form a subgroup of $S(\Omega)$
which will be denoted by $S_{fin}(\Omega)$. Finally, let $A_{fin}(\Omega)$ be the subgroup
of even permutations in $S_{fin}(\Omega)$ (note that it makes sense to talk about
even permutations in $S_{fin}(\Omega)$, but not in $S(\Omega)$).

(a) Prove that the group $A_{fin}(\Omega)$ is simple and that $A_{fin}(\Omega)$
is a subgroup of index two in $S_{fin}(\Omega)$. {\bf Hint:} To prove the first
assertion solve problem 5 in [DF, page 151].

(b) Prove that $A_{fin}(\Omega)$ and $S_{fin}(\Omega)$ are both normal in $S(\Omega)$.

(c) Prove that neither of the groups $S(\Omega)$ and $S_{fin}(\Omega)$ is finitely generated. {\bf Hint:} The two groups are not finitely generated for completely
different reasons.

(d) Construct a finitely generated subgroup $G$ of $S(\Omega)$ which contains
$S_{fin}(\Omega)$. {\bf Note:} This example shows that a subgroup of a finitely
generated group does not have to be finitely generated.
\end{document}
