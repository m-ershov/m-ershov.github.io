\documentclass[12pt]{article}

\usepackage{amsmath}
\usepackage{amssymb}
\usepackage{amsthm}
%\usepackage{psfig}

\begin{document}
\baselineskip=16pt
\textheight=9.6in
\parindent=0pt
\def\sk {\hskip .5cm}
\def\skv {\vskip .12cm}
\def\cos {\mbox{cos}}
\def\sin {\mbox{sin}}
\def\tan {\mbox{tan}}
\def\intl{\int\limits}
\def\lm{\lim\limits}
\newcommand{\frc}{\displaystyle\frac}
\def\xbf{{\mathbf x}}
\def\fbf{{\mathbf f}}
\def\gbf{{\mathbf g}}

\def\dbA{{\mathbb A}}
\def\dbB{{\mathbb B}}
\def\dbC{{\mathbb C}}
\def\dbD{{\mathbb D}}
\def\dbE{{\mathbb E}}
\def\dbF{{\mathbb F}}
\def\dbG{{\mathbb G}}
\def\dbH{{\mathbb H}}
\def\dbI{{\mathbb I}}
\def\dbJ{{\mathbb J}}
\def\dbK{{\mathbb K}}
\def\dbL{{\mathbb L}}
\def\dbM{{\mathbb M}}
\def\dbN{{\mathbb N}}
\def\dbO{{\mathbb O}}
\def\dbP{{\mathbb P}}
\def\dbQ{{\mathbb Q}}
\def\dbR{{\mathbb R}}
\def\dbS{{\mathbb S}}
\def\dbT{{\mathbb T}}
\def\dbU{{\mathbb U}}
\def\dbV{{\mathbb V}}
\def\dbW{{\mathbb W}}
\def\dbX{{\mathbb X}}
\def\dbY{{\mathbb Y}}
\def\dbZ{{\mathbb Z}}

\def\la{{\langle}}
\def\ra{{\rangle}}

\def\Aut{{\rm Aut}}
\def\End{{\rm End}}
\def\Inn{{\rm Inn}}

\bf\centerline{Algebra-I, Fall 2011. Midterm \#1. Due Thursday, September 29th}\rm
\vskip .3cm
{\bf Directions: } Each problem is worth 10 points. The best 4 out of 5 problems
will be counted (but you are encouraged to do all 5). Provide complete arguments
(do not skip steps). State clearly any result you are referring to. Partial credit for
incorrect solutions, containing steps in the right direction, may be given.
\vskip .1cm

{\bf Rules: } You are not allowed to discuss midterm problems with each other.
You may ask me any questions about the problems (e.g. if the formulation is unclear),
but as a rule I will not provide hints. You may freely use your class notes,
previous homework assignments and the book by Dummit and Foote. The use of other books
is allowed, but not encouraged. If you happen to run across a problem
very similar or identical to one on the midterm which is solved in another book,
do not consult that solution.

\skv
{\bf 1.} Let $G$ be a group and $H$ a subgroup of $G$. Recall that $H$ is
called {\bf characteristic in $G$} if $\phi(H)=H$ for any $\phi\in\Aut(G)$.
The subgroup $H$ is said to be {\bf fully characteristic in $G$} if $\phi(H)\subseteq H$
for any  $\phi\in\End(G)$ where $\End(G)$ is the set of all homomorphisms
from $G$ to $G$.

(a) (6 pts) Suppose that $K$ is a subgroup of $H$ and $H$ is a subgroup of $G$.
\begin{itemize}
\item[(i)] Prove that if $H$ is normal in $G$ and $K$ is characteristic in $H$,
then $K$ is normal in $G$.

\item[(ii)] Give an example showing that if $H$ is characteristic in $G$ and
$K$ is normal in $H$, then $K$ may not be normal in $G$.
\end{itemize}

(b) (4 pts) Give an example of a group $G$ and a subgroup $H$ of $G$ which
is characteristic, but not fully characteristic in $G$. {\bf Hint:} You
can find such a subgroup inside a familiar group.
\skv
{\bf 2.} Let $G$ be a finite group, $H$ and $K$ subgroups of $G$, and
let $n(H,K)$ denote the number of distinct $(H,K)$-double cosets in $G$.
\begin{itemize}
\item[(a)] (5 pts) Prove that $n(H,K)=\frac{1}{|H|\cdot |K|}\sum_{g\in G}|g^{-1}Hg\cap K|.$

\item[(b)] (5 pts) Suppose now that $H=\la h\ra$ and $K=\la k \ra$ where both $h$ and $k$
have order $2$. Prove that $n(H,K)=|G|/4$ if $h$ and $k$ are not conjugate in $G$
and $n(H,K)=(|G|+|C_{G}(h)|)/4$ if $H$ and $K$ are conjugate in $G$.
\end{itemize}
{\bf 3.} A group $G$ is called {\bf divisible} if for any $g\in G$ and positive
integer $n\in\dbN$ there exists $x\in G$ s.t. $x^n=g$.
\begin{itemize}
\item[(a)] (2 pts) Give an example of a divisible group.
\item[(b)] (1 pt) Give an example of a torsion divisible group (a group is said to
be torsion if all its elements have finite order). Of course, an example
for (b) will also work for (a).
\item[(c)] (7 pts) Prove that if $G$ is divisible, then $G$ cannot have a proper subgroup
of finite index. {\bf Hint:} First prove that $G$ cannot have a normal
subgroup of finite index.
\end{itemize}
\skv

{\bf 4.} Let $G$ be a finite group and $H$ a normal subgroup of $G$. Let $\mathcal K$ be a $G$-conjugacy classes
(that is, a conjugacy class of $G$) which is contained in $H$. Note that every $H$-conjugacy class
is either contained in $\mathcal K$ or disjoint from $\mathcal K$, so $\mathcal K$ is a union
of $k$ (distinct) $H$-conjugacy classes for some $k$.

(a) (5 pts) Prove that $k=[G: H \cdot C_G(x)]$ where $x$ is an arbitrary element of $\mathcal K$

(b) (2 pts) Let $G=S_n$ and $H=A_n$, and pick some element $\sigma\in \mathcal K$. Prove that $k=1$
if $C_G(\sigma)$ contains an odd permutation and $k=2$ otherwise.

(c) (3 pts) Once again, let $G=S_n$ and $H=A_n$, let $m\leq n$ be an odd number and $\mathcal K$
the $G$-conjugacy class consisting of all $m$-cycles (note that $\mathcal K\subseteq A_n$).
Prove that $k=2$ if $n-m\leq 1$ and $k=1$ otherwise.
\skv
{\bf Hint for (a):} You may deduce (a) directly from the orbit-stablizer formula applied
to suitable group action. Alternative hint is given in Dummit and Foote (see Problem~19, page~131).
\skv
\skv
{\bf 5.} Let $p$ be prime. Let $G$ be a finite group, $K$ a normal subgroup of $G$,
and assume that $|G/K|$ is divisible by $p$. Let $P$ be a Sylow $p$-subgroup of $G$.

\begin{itemize}
\item[(a)] (4 pts) Prove that $PK/K$ is a Sylow $p$-subgroup of $G/K$

\item[(b)] (4 pts) Prove that $n_p(G/K)$ divides $n_p(G)$ where $n_p(\cdot)$ denotes
the number of Sylow $p$-subgroups. {\bf Hint:} Find a relation between $N_{G/K}(PK/K)$
and $N_G(P)$.
 
\item[(c)] (2 pts) Prove that $n_p(G/K)=n_p(G)$ if and only if $P$ is normal in $PK$.
\end{itemize}
\end{document}
