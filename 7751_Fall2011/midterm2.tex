\documentclass[12pt]{article}

\usepackage{amsmath}
\usepackage{amssymb}
\usepackage{amsthm}
%\usepackage{psfig}

\begin{document}
\baselineskip=16pt
\textheight=9.6in
\parindent=0pt
\def\sk {\hskip .5cm}
\def\skv {\vskip .12cm}
\def\cos {\mbox{cos}}
\def\sin {\mbox{sin}}
\def\tan {\mbox{tan}}
\def\intl{\int\limits}
\def\lm{\lim\limits}
\newcommand{\frc}{\displaystyle\frac}
\def\xbf{{\mathbf x}}
\def\fbf{{\mathbf f}}
\def\gbf{{\mathbf g}}

\def\dbA{{\mathbb A}}
\def\dbB{{\mathbb B}}
\def\dbC{{\mathbb C}}
\def\dbD{{\mathbb D}}
\def\dbE{{\mathbb E}}
\def\dbF{{\mathbb F}}
\def\dbG{{\mathbb G}}
\def\dbH{{\mathbb H}}
\def\dbI{{\mathbb I}}
\def\dbJ{{\mathbb J}}
\def\dbK{{\mathbb K}}
\def\dbL{{\mathbb L}}
\def\dbM{{\mathbb M}}
\def\dbN{{\mathbb N}}
\def\dbO{{\mathbb O}}
\def\dbP{{\mathbb P}}
\def\dbQ{{\mathbb Q}}
\def\dbR{{\mathbb R}}
\def\dbS{{\mathbb S}}
\def\dbT{{\mathbb T}}
\def\dbU{{\mathbb U}}
\def\dbV{{\mathbb V}}
\def\dbW{{\mathbb W}}
\def\dbX{{\mathbb X}}
\def\dbY{{\mathbb Y}}
\def\dbZ{{\mathbb Z}}

\def\la{{\langle}}
\def\ra{{\rangle}}

\def\Aut{{\rm Aut}}
\def\End{{\rm End}}
\def\Inn{{\rm Inn}}
\def\eps{{\varepsilon}}

\bf\centerline{Algebra-I, Fall 2011. Midterm \#2. Due Thursday, November 3rd}\rm
\vskip .3cm
{\bf Directions: } Each problem is worth 10 points. The best 4 out of 5 problems
will be counted (but you are encouraged to do all 5). Provide complete arguments
(do not skip steps). State clearly any result you are referring to. Partial credit for
incorrect solutions, containing steps in the right direction, may be given.
\vskip .1cm

{\bf Rules: } You are not allowed to discuss midterm problems with each other.
You may ask me any questions about the problems (e.g. if the formulation is unclear),
but as a rule I will not provide hints. You may freely use your class notes,
previous homework assignments and the book by Dummit and Foote. The use of other books
is allowed, but not encouraged. If you happen to run across a problem
very similar or identical to one on the midterm which is solved in another book,
do not consult that solution.

\skv
{\bf 1.} Given a finite group $G$ and a positive integer $n$, denote
by $a_n (G)$ the number of elements of $G$ of order $n$ and by $b_n(G)$
the  number of elements of $G$ of order dividing $n$. The goal
of this problem is to prove the following theorem:
\skv
{\bf Theorem A:} If $G$ and $H$ are finite abelian groups and $a_n(G)=a_n(H)$ for all $n$, then $G$ is isomorphic to $H$.
\skv
(a) (1 pt) Let $G$ and $H$ be finite groups. Prove that $a_n(G)=a_n(H)$ for all $n$ $\iff$ $b_n(G)=b_n(H)$ for all $n$.

(b) (2 pts) Suppose that $G=X\times Y$. Prove that $b_n(G)=b_n(X)b_n(Y)$.

(c) (4 pts)  Suppose that $G$ and $H$ are finite abelian groups s.t. $a_n(G)=a_n(H)$ for all $n$. Prove that there exists a non-trivial
group $C$ s.t. $G\cong A\times C$ and $H\cong B\times C$ for some
groups $A$ and $B$. {\bf Hint:} Use the classification theorem in
invariant factors form.
 
(d) (3 pts) Now use (a),(b) and (c) to prove Theorem~A.
\skv
\skv

{\bf 2.} Let $G$ be a finite group

(a) (5 pts) Prove that $G$ is nilpotent if and only if $G$ contains
a normal subgroup of order $m$ for any $m$ dividing $|G|$.

(b) (5 pts) Prove that $G$ is cyclic if and only if $G$ contains
a unique subgroup of order $m$ for any $m$ dividing $|G|$.

{\bf Note:} Of course, the forward direction in (b) is well known,
so you can assume it without proof. {\bf Hint:} You may find
some results in [DF,6.1] useful.
\skv
\skv

{\bf 2.} In all parts of this problem $G$ is a finite group.

(a) (3 pts) Prove that $G$ has a simple quotient (that is, $G$ has a quotient
which is a simple group).

(b) (2 pts) Suppose that $G$ is perfect, that is, $[G,G]=G$. Prove that $G$ has
a non-abelian simple quotient.

(c) (5 pts) Once again, let $G$ be an arbitrary finite group. Prove that $G$
has a unique normal subgroup $K$ such that $K$ is perfect and $G/K$
is solvable. {\bf Hint:} Such $K$ can be easily described in terms of
the derived series of $G$.
\skv
\skv

{\bf 4.} (10 pts) Prove that there are precisely 5 isomorphism classes of groups of order $20$. Include all the details. 
\skv
\skv


{\bf 5.} Let $X$ be a finite set and $F=F(X)$ the (standard) free group on $X$.
\skv
{\bf Definition:} An element $g\in F$ is called cyclically reduced (with respect to $X$) if the reduced word representing $g$ is of the form $x_1^{\eps_1}\ldots x_n^{\eps_n}$
(with $x_i\in X$, $\eps_i=\pm 1$) where $x_n^{\eps_n}\neq (x_1^{\eps_1})^{-1}$.
\skv
\begin{itemize}
\item[(a)] (1 pt) Prove that any element $g\in F$ is conjugate to a cyclically
reduced element.

\item[(b)] (3 pts) Prove that if $f\in F$ and $n\in\dbN$, then $f^n$ is cyclically reduced if and only if $f$ is cyclically reduced.

\item[(c)] (4 pts) Prove that if $f,g\in F$ and $f^n=g^n$ for some $n\in\dbN$, then $f=g$. Explain the argument in detail.
{\bf Hint:} First consider the case when $f$ is cyclically reduced
and then treat the general case. 

\item[(d)] (2 pts) Now prove that if $f,g\in F$ and $f^n=g^m$ for some $n,m\in\dbN$,
then $f$ and $g$ commute. {\bf Hint:} Use (c).
\end{itemize}
\end{document}