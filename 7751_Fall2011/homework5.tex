\documentclass[12pt]{article}

\usepackage{amsmath}
\usepackage{amssymb}
\usepackage{amsthm}
%\usepackage{psfig}

\begin{document}
\baselineskip=16pt
\textheight=9.6in
\parindent=0pt 
\def\sk {\hskip .5cm}
\def\skv {\vskip .12cm}
\def\cos {\mbox{cos}}
\def\sin {\mbox{sin}}
\def\tan {\mbox{tan}}
\def\intl{\int\limits}
\def\lm{\lim\limits}
\newcommand{\frc}{\displaystyle\frac}
\def\xbf{{\mathbf x}}
\def\fbf{{\mathbf f}}
\def\gbf{{\mathbf g}}

\def\dbA{{\mathbb A}}
\def\dbB{{\mathbb B}}
\def\dbC{{\mathbb C}}
\def\dbD{{\mathbb D}}
\def\dbE{{\mathbb E}}
\def\dbF{{\mathbb F}}
\def\dbG{{\mathbb G}}
\def\dbH{{\mathbb H}}
\def\dbI{{\mathbb I}}
\def\dbJ{{\mathbb J}}
\def\dbK{{\mathbb K}}
\def\dbL{{\mathbb L}}
\def\dbM{{\mathbb M}}
\def\dbN{{\mathbb N}}
\def\dbO{{\mathbb O}}
\def\dbP{{\mathbb P}}
\def\dbQ{{\mathbb Q}}
\def\dbR{{\mathbb R}}
\def\dbS{{\mathbb S}}
\def\dbT{{\mathbb T}}
\def\dbU{{\mathbb U}}
\def\dbV{{\mathbb V}}
\def\dbW{{\mathbb W}}
\def\dbX{{\mathbb X}}
\def\dbY{{\mathbb Y}}
\def\dbZ{{\mathbb Z}}

\def\la{{\langle}}
\def\ra{{\rangle}}

\def\Aut{{\rm Aut}}
\def\Inn{{\rm Inn}}

\bf\centerline{Homework \#5. }\rm
\vskip .1cm
{\bf Approximate plan for next three weeks: } 

Oct 4,6: Nilpotent and solvable groups (6.1)

Oct 13: Free groups (6.3)

Oct 18,20: Free groups and presentations by generators and relations.

Note: our discussion of niplotent groups and free groups will
be significantly different from the one in Dummit and Foote.

\vskip .1cm
\centerline{\bf Problems, to be submitted by Thursday, October, 6th}
\skv
{\bf 1.} Let $p$ be a prime.

(a) Let $1\leq k<p$. Prove that a Sylow $p$-subgroup of the symmetric group $S_{pk}$
has order $p^k$ and find (some) Sylow $p$-subgroup of $S_{pk}$ explicitly.

(b) Prove that a Sylow $p$-subgroup of the symmetric group $S_{p^2}$
has order $p^{p+1}$ and find (some) Sylow $p$-subgroup of $S_{p^2}$ explicitly.
{\bf Hint:} First find a subgroup of order $p^2$ inside $S_{p^2}$ (this is done
as in part (a)), call it $H$. Then find an element of order $p$ in $G\setminus H$
which normalizes $H$.
\skv
{\bf 2.} DF, Problem 6 on page 184.
\skv

{\bf 3.} DF, Problem 7(a)(c)(e) on page 185. In (e) only
prove the uniqueness part (we showed the existence in class).
Clarification for part (c): for each isomorphism class of $S$
you are asked to construct certain number of non-isomorphic groups
of order $56$ with normal $7$-Sylow and $2$-Sylow isomorphic to $S$.
You are not asked to prove that your groups cover all possible isomorphism classes
(this part is optional and can be done using a sufficient condition for two semi-direct products
$H\rtimes_{\phi} K$ and $H\rtimes_{\psi} K$ to be isomorphic proved in Lecture 10). However, you should prove 
the statement of the hint in brackets following part (c).
\skv

{\bf 4.} DF, Problem 5 on page 184. The holomorph of a group $G$ denoted by $Hol(G)$ 
is defined on page 179 of Dummit and Foote.
\end{document}