\documentclass[12pt]{article}

\usepackage{amsmath}
\usepackage{amssymb}
\usepackage{amsthm}
\usepackage{amscd}
\usepackage[all]{xy}
%\usepackage{psfig}

\begin{document}
\baselineskip=16pt
\textheight=8.5in
\textwidth=6in
\parindent=0pt
\def\sk {\hskip .5cm}
\def\skv {\vskip .12cm}
\def\cos {\mbox{cos}}
\def\sin {\mbox{sin}}
\def\tan {\mbox{tan}}
\def\intl{\int\limits}
\def\lm{\lim\limits}
\newcommand{\frc}{\displaystyle\frac}
\def\xbf{{\mathbf x}}
\def\fbf{{\mathbf f}}
\def\gbf{{\mathbf g}}

\def\dbA{{\mathbb A}}
\def\dbB{{\mathbb B}}
\def\dbC{{\mathbb C}}
\def\dbD{{\mathbb D}}
\def\dbE{{\mathbb E}}
\def\dbF{{\mathbb F}}
\def\dbG{{\mathbb G}}
\def\dbH{{\mathbb H}}
\def\dbI{{\mathbb I}}
\def\dbJ{{\mathbb J}}
\def\dbK{{\mathbb K}}
\def\dbL{{\mathbb L}}
\def\dbM{{\mathbb M}}
\def\dbN{{\mathbb N}}
\def\dbO{{\mathbb O}}
\def\dbP{{\mathbb P}}
\def\dbQ{{\mathbb Q}}
\def\dbR{{\mathbb R}}
\def\dbS{{\mathbb S}}
\def\dbT{{\mathbb T}}
\def\dbU{{\mathbb U}}
\def\dbV{{\mathbb V}}
\def\dbW{{\mathbb W}}
\def\dbX{{\mathbb X}}
\def\dbY{{\mathbb Y}}
\def\dbZ{{\mathbb Z}}

\def\la{{\langle}}
\def\ra{{\rangle}}

\def\Ker{{\rm Ker\,}}
\def\Aut{{\rm Aut}}
\def\Inn{{\rm Inn}}

\bf\centerline{Homework \#9. }\rm
\vskip .1cm
{\bf Plan for next week:} Unique factorization and irreducibility
criteria in polynomial rings (9.3 and 9.4).


\vskip .1cm
\centerline{\bf Problems, to be submitted by Thursday, November 10th}
\skv
{\bf 1.} (a) Let $G$ be a finitely generated group. Use Zorn's lemma to show that $G$ has a  
maximal subgroup (recall that a maximal subgroup of $G$ is a maximal element of the set of  
proper  subgroups of $G$ partially ordered by inclusion). {\bf Hint:} The key step is to show 
that  if $\mathcal C$ is a chain of proper subgroups of $G$, then the union of subgroups in this 
chain is not the whole $G$.

(b) (optional) If your proof in (a) is correct, a nearly identical argument should
imply that $G$ always has a maximal normal subgroup. However, the latter is true
under weaker assumptions on $G$. Can you find a natural condition on $G$
(weaker than finite generation) that guarantees the existence of a maximal normal subgroup? 
Can you give an example of a group which has a maximal normal subgroup, but no maximal subgroup?

\skv
{\bf 2.} Let $R$ be a commutative ring with $1$. The {\it nilradical} of $R$ denoted $Nil(R)$ is the set of all nilpotent elements of $R$, that is $$Nil(R)=\{a\in R : a^n=0 \mbox{ for some }n\in\dbN\}.$$ The {\it Jacobson radical} of $R$ denoted by $J(R)$ is the intersection 
of all maximal ideals of $R$. Prove that 

(a) $Nil(R)$ and $J(R)$ are ideals of $R$ 

(b) $Nil(R)\subseteq J(R)$.
\skv

{\bf 3.} Let $D$ be a positive integer such that $D\equiv 3\mod 4$, and let
$R=\dbZ[\frac{1+\sqrt{-D}}{2}]$, that is, $R$ is the minimal subring of $\dbC$ containing $\dbZ$ and $\frac{1+\sqrt{-D}}{2}$.

(a) Prove that $R=\{a+ b \frac{1+\sqrt{-D}}{2} : a,b\in\dbZ \}$. You may skip details, but it should
be clear from your argument where the assumption $D\equiv 3\mod 4$ is used (otherwise the result is simply not true).

(b) Assume that $D=3, 7$ or $11$. Prove that $R$ is a Euclidean domain.
\skv

{\bf 4.} Let $R=\dbZ[\sqrt{5}]=\{a+b\sqrt{5} : a,b\in\dbZ\}$. Find an element of $R$
which is irreducible but not prime and deduce that $R$ is not a unique factorization domain (UFD).
{\bf Note:} It is an easy exercise to prove that the property {\it prime=irreducible} holds in any UFD.
 {\bf Hint:} Consider the equality $2\cdot 2=(\sqrt{5}+1)(\sqrt{5}-1)$. In order to check
whether some element of $R$ is irreducible it is convenient to use the standard norm function
$N:R\to\dbZ_{\geq 0}$ given by $N(a+b\sqrt{5})=|a^2-5b^2|$ (note that $N(uv)=N(u)N(v)$).
\skv
{\bf 5.} (a) (practice) Problem 7.3.34. Note: in all exercises in 7.3 $R$ is assumed to be a ring  
with $1$ (this is crucial for this problem). Also note that $IJ$ is NOT defined to be the set 
$\{ij: i\in I, j\in J\}$;  by definition, $IJ$ is the set of finite sums of elements of the form 
$ij$, with $i\in I, j\in J$.  

(b) Read the section on the Chinese remainder theorem (7.6).

\skv
{\bf 6.} (practice) DF, Problem 7.1.26 and 7.1.27

\skv
{\bf 7.} (a) (practice) DF, Problem 7.2.3

(b) DF, Problem 7.2.5 (you may freely refer to 7.2.3).

(c) Let $M$ be an ideal of a commutative ring $R$ with $1$. Prove that the following conditions are equivalent:

\sk (i) $M$ is the unique maximal ideal of $R$ 

\sk (ii) every element of $R\setminus M$ is invertible. 

A commutative ring with $1$ which has the unique maximal ideal is called {\it local}.
\skv
{\bf 8.} (practice) Let $R=\dbZ_{14}$, $D=\{\bar 1,\bar 2,\bar 4,\bar 8\}$ (note that $D$ is multiplicatively closed
but it does contain zero divisors). Prove that the localization $RD^{-1}$ is isomorphic to $\dbZ_{7}$ (we gave a brief an outline of proof in class).
\end{document}