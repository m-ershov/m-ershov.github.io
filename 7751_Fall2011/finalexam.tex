\documentclass[12pt]{article}

\usepackage{amsmath}
\usepackage{amssymb}
\usepackage{amsthm}
%\usepackage{psfig}

\begin{document}
\baselineskip=16pt
\textheight=9in
\parindent=0pt
\def\sk {\hskip .5cm}
\def\skv {\vskip .12cm}
\def\cos {\mbox{cos}}
\def\sin {\mbox{sin}}
\def\tan {\mbox{tan}}
\def\intl{\int\limits}
\def\lm{\lim\limits}
\newcommand{\frc}{\displaystyle\frac}
\def\xbf{{\mathbf x}}
\def\fbf{{\mathbf f}}
\def\gbf{{\mathbf g}}

\def\dbA{{\mathbb A}}
\def\dbB{{\mathbb B}}
\def\dbC{{\mathbb C}}
\def\dbD{{\mathbb D}}
\def\dbE{{\mathbb E}}
\def\dbF{{\mathbb F}}
\def\dbG{{\mathbb G}}
\def\dbH{{\mathbb H}}
\def\dbI{{\mathbb I}}
\def\dbJ{{\mathbb J}}
\def\dbK{{\mathbb K}}
\def\dbL{{\mathbb L}}
\def\dbM{{\mathbb M}}
\def\dbN{{\mathbb N}}
\def\dbO{{\mathbb O}}
\def\dbP{{\mathbb P}}
\def\dbQ{{\mathbb Q}}
\def\dbR{{\mathbb R}}
\def\dbS{{\mathbb S}}
\def\dbT{{\mathbb T}}
\def\dbU{{\mathbb U}}
\def\dbV{{\mathbb V}}
\def\dbW{{\mathbb W}}
\def\dbX{{\mathbb X}}
\def\dbY{{\mathbb Y}}
\def\dbZ{{\mathbb Z}}

\def\la{{\langle}}
\def\ra{{\rangle}}
\def\eps{{\epsilon}}

\def\Aut{{\rm Aut}}
\def\Inn{{\rm Inn}}

\bf\centerline{Math 7751, Fall 2011. Final exam. Due Tuesday, December 6th}\rm
\vskip .7cm
{\bf Directions: } Each problem is worth 15 points. The best 5 out of 6 scores
will be counted with 100\% weight, and the lowest score will be counted with
33$\frac{1}{3}$\% weight, so the maximal possible total is 80 points. Provide complete arguments
(do not skip steps). State clearly any result you are referring to. Partial credit for
incorrect solutions, containing steps in the right direction, may be given.
\vskip .1cm

{\bf Rules: } You are not allowed to discuss midterm problems with each other.
You may ask me any questions about the problems (e.g. if the formulation is unclear),
but I may only provide minor hints. You may use freely your class notes,
previous homework assignments and the book by Dummit and Foote. The use of other books is allowed,
but not encouraged. If you happen to run across a problem very similar or identical to one on the midterm
which is solved in another book, do not consult that solution.

\skv
{\bf 1.} Let G be a group of order $pqr$ where $p$, $q$, $r$ are 
distinct primes and 
\begin{itemize}
\item[(i)] $p > q > r$
\item[(ii)] $gcd(q, p - 1) = gcd(r, q - 1) = gcd(r, p - 1) = 1$.
\end{itemize}
\begin{itemize}
\item[(a)] Prove that $G$ has a normal subgroup $P$ of order $p$.
\item[(b)] Prove that $G$ has a normal subgroup $S$ of order $pq$
and that this subgroup $S$ is cyclic.
\item[(c)] Finally prove that $G$ itself is cyclic.
\end{itemize}
\skv
{\bf 2.} Let $G$ be group. A subgroup $H$ of $G$ will be called {\it essential}
if $H\cap K\neq \{1\}$ for every non-trivial subgroup $K$ of $G$.
\begin{itemize}

\item[(a)] Let $p$ be a prime and $k\geq 2$. Prove that the group $\dbZ/p^k\dbZ$
has a proper essential subgroup.

\item[(b)] Assume that $H_1$ is an essential subgroup of $G_1$ and $H_2$
is an essential subgroup of $G_2$. Prove that $H_1\times H_2$ is an essential subgroup
of $G_1\times G_2$.

\item[(c)] Let $G$ be a finite abelian group. Prove that $G$ does not have a proper
essential subgroup if and only if $G$ is a direct product of groups of prime order.
\end{itemize}
\skv
{\bf 3.}
\begin{itemize}
\item[(a)] Find a monic irreducible polynomial of degree $4$ in $\dbF_2[x]$.
Then use it to construct a field $F$ with $|F|=16$ and {\bf find explicitly}
a generator of the multiplicative group $F^{\times}$ (recall that we proved
in class that $F^{\times}$ is cyclic).
\item[(b)] Let $p>3$ be a prime and $R=\dbF_p[x]/(x^3-1)$.
Describe the multiplicative group $R^{\times}$ as a direct product of cyclic groups.
{\bf Hint:} There will be two different cases depending on whether
$p\equiv 1$ or $2$ mod $3$.
\end{itemize}


\skv
{\bf 4.} Let $\dbR$ denote the real numbers. The purpose of this problem is to show that the ring 
$A=\dbR[x,y]/(x^2+y^2-1)$ is not a UFD. For an element $f\in \dbR[x,y]$ 
we denote its image in $A$ by $[f]$.

\begin{itemize}
\item[(a)] Show that every element of $A$ can be uniquely represented in the form $[f(x)+g(x)y]$
where $f(x),g(x)\in F[x]$. 
\item[(b)] Show that $A$ has an automorphism $\phi$ of order $2$ such that
$\phi([f(x)])=[f(x)]$ for each $f(x)\in F[x]$ and $\phi([y])= -[y]$.
\item[(c)] Use (a) and (b) to construct a function $N:A\to F[x]$ such that $N(uv)=N(u)N(v)$ for all $u,v\in A.$   
\item[(d)] Use the function $N$ from (c) to show that $[x]$ is an irreducible element of $A$
and that the only invertible elements of $A$ are (images of) nonzero constant polynomials.
{\bf Hint:} It is essential that you are working over $\dbR$, not over $\dbC$.
\item[(e)] Now show that $A$ is not a UFD.
\end{itemize}
\skv
{\bf 5.}  Let $R$ be a commutative ring with $1$, and let $\Omega$ be the set
of all ideals $I$ of $R$ such that every element of $I$ is $0$ or a zero divisor.
\begin{itemize}
\item[(a)] Prove that $\Omega$ has a maximal element (with respect to inclusion)
and moreover any element of $\Omega$ is contained in a maximal element.
\item[(b)] Let $I$ be a maximal element of $\Omega$. Prove that $I$ must be prime.
\end{itemize}
{\bf Hint:} It is a well-known fact that for any non-nilpotent element
$f\in R$ there exists a prime ideal $P$ of $R$ such that $f\not\in P$.
While this result is not directly related to Problem 5(b), its proof
(see e.g. DF, Proposition 12 on p.674) may help you with 5(b).

\skv
{\bf 6.} Let $F$ be a field and $R=F[x,y]$ the ring of polynomials over $R$ in two (commuting) variables $x$ and $y$. Let $I=xR$ be the principal ideal of $R$ generated by $x$ and $S=F+I=\{f+i: f\in F, i\in I\}$. Observe that $S$ is a subring of $R$ and $I$ is an ideal of $S$ (you need not justify these facts).
\begin{itemize}
\item[(a)] Prove that $I$ is not finitely generated as an ideal of $S$.
{\bf Hint:} Assume that $I$ is finitely generated as an ideal of $S$
and reach a contradiction by showing that there must exist a natural number $m$ 
such that any polynomial $p(x,y)\in I$ contains no monomials of the form $xy^n$, with $n>m.$
\item[(b)] Prove that $S$ is not finitely generated as a ring.
{\bf Hint:} It is possible to answer (b) using (a) without doing any computations.
\end{itemize}
\end{document}
