\documentclass[12pt]{article}

\usepackage{amsmath}
\usepackage{amssymb}
\usepackage{amsthm}
%\usepackage{psfig}

\begin{document}
\baselineskip=16pt
\textheight=9.2in
\parindent=0pt 
\def\sk {\hskip .5cm}
\def\skv {\vskip .12cm}
\def\cos {\mbox{cos}}
\def\sin {\mbox{sin}}
\def\tan {\mbox{tan}}
\def\intl{\int\limits}
\def\lm{\lim\limits}
\newcommand{\frc}{\displaystyle\frac}
\def\xbf{{\mathbf x}}
\def\fbf{{\mathbf f}}
\def\gbf{{\mathbf g}}

\def\dbA{{\mathbb A}}
\def\dbB{{\mathbb B}}
\def\dbC{{\mathbb C}}
\def\dbD{{\mathbb D}}
\def\dbE{{\mathbb E}}
\def\dbF{{\mathbb F}}
\def\dbG{{\mathbb G}}
\def\dbH{{\mathbb H}}
\def\dbI{{\mathbb I}}
\def\dbJ{{\mathbb J}}
\def\dbK{{\mathbb K}}
\def\dbL{{\mathbb L}}
\def\dbM{{\mathbb M}}
\def\dbN{{\mathbb N}}
\def\dbO{{\mathbb O}}
\def\dbP{{\mathbb P}}
\def\dbQ{{\mathbb Q}}
\def\dbR{{\mathbb R}}
\def\dbS{{\mathbb S}}
\def\dbT{{\mathbb T}}
\def\dbU{{\mathbb U}}
\def\dbV{{\mathbb V}}
\def\dbW{{\mathbb W}}
\def\dbX{{\mathbb X}}
\def\dbY{{\mathbb Y}}
\def\dbZ{{\mathbb Z}}

\def\la{{\langle}}
\def\ra{{\rangle}}

\def\Aut{{\rm Aut}}
\def\Inn{{\rm Inn}}

\bf\centerline{Homework \#3, to be submitted by Thursday, September, 15th}\rm
\vskip .1cm

{\bf 1.} An action of a group $G$ on a set $X$ is called {\it transitive} if it has
just one orbit, that is, for any $x,y\in X$ there exists $g\in G$ with $g. x=y$.

(a) Let $(G,X,.)$ be a group action. Prove that if $x,y\in X$ lie in the same orbit,
then their stabilizers $Stab_G(x)$ and $Stab_G(y)$ are conjugate, that is, there exists $g\in G$
with $g Stab_G(x) g^{-1} = Stab_G(y)$. 

(b) Suppose that $(G,X,.)$ is a transitive action and fix $x\in X$. Prove that
the kernel of this action is equal to $\bigcap\limits_{g\in G} g Stab_G(x) g^{-1}$

(c) Now suppose that $G$ and $X$ are both finite, $(G,X,.)$ is a transitive 
faithful action (where `faithful' means the kernel is trivial) and $G$ is abelian. 
Prove that for any $g\in G\setminus \{1\}$ the fixed set $Fix_X(g)$ is empty. 
Deduce that $|X|=|G|$. {\bf Hint:} Use (b).
\skv

{\bf 2.} Let $G$ be a group. For each $g\in G$ let $\iota_g: G\to G$
be the conjugation by $g$, that is, $\iota_g(x)=gxg^{-1}$.
Recall that $\iota_g\in \Aut(G)$ for any $g\in G$ and
the mapping $\iota: G\to \Aut(G)$ given by $\iota(g)=\iota_g$
is a homomorphism. Elements of the subgroup $\Inn(G)=\iota(G)$ of $\Aut(G)$
are called inner automorphisms.

(a) Prove that for any $g\in G$ and $\sigma\in \Aut(G)$
one has $\sigma \iota_g \sigma^{-1}=\iota_{\sigma(g)}$.
Deduce that $\Inn(G)$ is a normal subgroup of $\Aut(G)$.

(b) Let $H$ be a normal subgroup of $G$. Note that for each $g\in G$, 
the mapping $\iota_g$ restricted to $H$
is an automorphism of $H$. By slight abuse of notation
we denote this automorphism of $H$ by $\iota_g$ as well.
Prove that $\iota_g$ is an inner automorphism of $H$ if and
only if $g\in H\cdot C_{G}(H)$ where $C_G(H)$ is the centralizer
of $H$ in $G$.
\skv

{\bf 3.} Let $n\geq 4$ and $f=(1,2)(3,4)\in S_n$. Prove that 
$|C_{S_n}(f)|=8 (n-4)!$. Then describe elements of this 
centralizer explictly. {\bf Hint:} What is the conjugacy
class of $f$?
\skv

{\bf 4.} (optional) {\it Necklace-counting problem:}
Suppose that we want to build a necklace using $n$ beads of $k$ possible colors
(we do not have to use all available colors). Two necklaces will be considered equivalent 
if they can be obtained from each other using rotations or reflections. What is the number 
of non-equivalent necklaces one can construct?

{\it Approach using group actions.} Let $X$ be the set of all necklaces with
beads of $k$ possible colors located at the vertices of a regular $n$-gon. The dihedral
group $D_{2n}$ has a natural action on $X$, and the orbits under that action
are precisely equivalence classes of necklaces in the above sense. Use this
interpretation and Burnside's orbit-counting formula to prove that for $n=9$ the number of 
non-equivalent necklaces is
$$\frac{k^9+2k^3+6k+9k^5}{18}.$$
Then try to do the same for general $n$. 

{\bf 5.} Let $G=GL_n(\dbZ_p)$ and $P=U_n(\dbZ_p)$,
the subgroup of upper-unitriangular matrices in $GL_n(\dbZ_p)$.

(a) Prove that $P$ is a Sylow $p$-subgroup of $G$
(you may use the formula for $|G|$ established in Homework\#2).

(b) Let $B=UT_n(\dbZ_p)$, the subgroup of upper-triangular matrices in $G$.
Prove that $N_G(P)=B$ (note that the inclusion $N_G(P)\supseteq B$ holds
by Problem~6 in Homework\#2). {\bf Hint: } Consider the natural action
of $G$ on $X=\dbZ_p^n$. First show that if $g\in N_G(P)$, then $g . X^P=X^P$.
Then compute $X^P$ explicitly (using the definition of $P$) and then combine
these two results to deduce that $N_G(P)$ is contained in certain proper subgroup
of $G$. Finally, use induction on $n$ to prove that $N_G(P)\subseteq B$.

(c) Find $n_p(G)$, the number of Sylow $p$-subgroups of $G$.
\skv
{\bf 6.} Let $G$ be a group such that $G/Z(G)$ is cyclic.
Prove that $G$ is abelian.
\skv

{\bf 7.} (a) Prove that if $\phi:\dbZ\to\dbZ$ is a surjective homomorphism,
then $\phi$ must be injective. 

(b) Find a group $G$ and a homomorphism $\phi:G\to G$ which is surjective
and not injective.

(c) (optional) A group $G$ with the property that every surjective homomorphism from
$G$ to $G$ must also be injective is called {\it hopfian}, so part (a) asserts
that $\dbZ$ is hopfian. Can you find interesting sufficient conditions for
a group to be hopfian?
\skv
\end{document}
